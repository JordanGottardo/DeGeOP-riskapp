\section{Lista di controllo}
Durante l'applicazione del \glo{Walkthrough}{walkthrough} ai documenti, sono riportati di seguito gli errori più frequenti. Per migliorare l'efficienza e l'efficacia da parte dei verificatori è opportuno che si basino sui seguenti controlli:
	\begin{itemize}
		\item \textbf{lingua italiana:}
		\begin{itemize}
			\item la prima parola di una voce dell'elenco puntato inizia con la lettera maiuscola;
			\item la voce finale dell'elenco puntato non termina con il punto;
			\item una voce intermedia dell'elenco puntato non termina con il punto e virgola.
		\end{itemize}
		\item \textbf{norme stilistiche:}
		\begin{itemize}
			\item il carattere "e maiuscolo accentato" è scritto E' invece di È;
			\item i due punti in grassetto dopo un termine in grassetto.
		\end{itemize}
		\item \textbf{\LaTeX:}
		\begin{itemize}
			\item date e orari non scritti con i rispettivi comandi \texttt{\textbackslash frmdata\{GG\}\{MM\}\{YYYY\}} e \texttt{\textbackslash frmora\{hh\}\{mm\}};
			\item mancato utilizzo dei comandi personalizzati;
			\item utilizzo scorretto delle parentesi graffe dopo i comandi \LaTeX;
			\item mancato aggiornamento dell'intestazione del documento dopo una modifica;
			\item link e riferimenti non funzionanti o assenti.
		\end{itemize}
		\item \textbf{\glo{UML}{UML}:}
		\begin{itemize}
			\item casi d'uso non proporzionati correttamente tra loro;
			\item collegamenti in uscita non ad angolo retto.
		\end{itemize}
		\item \textbf{glossario:}
		\begin{itemize}
			\item mancata evidenziazione di termini presenti nel \gl;
			\item termini evidenziati impropriamente non presenti nel \gl.
		\end{itemize}
        \item \textbf{nomi dei documenti:}
        \begin{itemize}
            \item mancata indicazione della versione di riferimento di un documento.
        \end{itemize}
	\end{itemize}
