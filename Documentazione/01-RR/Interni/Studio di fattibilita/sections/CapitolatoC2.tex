
\section {Capitolato C2}


\subsection {Descrizione}
Il \glo{Capitolato}{capitolato} proposto da zero12 prevede la realizzazione di un applicativo di assistenza virtuale. Sfruttando il riconoscimento vocale verranno accolti gli ospiti dell’azienda. L’arrivo di tali ospiti verrà annunciato (sulla piattaforma di comunicazione scelta dall’azienda) al personale interessato e questi avranno la possibilità, nell’attesa, di richiedere alcuni comfort.


\subsection {Dominio applicativo}
L’applicazione che si andrà a sviluppare consisterà di un’interfaccia prevalentemente vocale da implementare sfruttando gli SDK esistenti di un assistente virtuale (es: Siri o Cortana). Tale interfaccia dovrà comunicare con un \glo{Database}{database} e avvertire le persone interessate dall’arrivo dell'ospite utilizzando la piattaforma di comunicazione aziendale (\glo{Slack}{Slack}).


\subsection {Dominio tecnologico}
Sono richieste le conoscenze di \glo{HTML}{HTML5}, \glo{CSS}{CSS3}, \glo{JavaScript}{JavaScript}. \\
È richiesto inoltre di utilizzare i servizi \glo{Amazon Web Services}{AWS}, in particolare \glo{Amazon Web Services}{AWS} Lambda, e tecnologie quali node.js (o Swift) e \glo{Database}{database} NoSQL (MongoDB o DynamoDB). \\
Sarà necessario utilizzare l'SDK di un assistente virtuale. \\
È infine implicitamente richiesto di conoscere come modificare \glo{Slack}{Slack} (o crearne un plug-in).


\subsection {Criticità}
In caso si decidesse di sviluppare l’idea di questo \glo{Capitolato}{capitolato} si renderà necessario analizzare gli SDK dei principali assistenti virtuali, stilando una lista di pro e contro. Successivamente sarà richiesto lo studio delle \glo{API}{API} dell’SDK scelto. \\
Inoltre si dovrà riuscire a modificare \glo{Slack}{Slack}, tecnologia proprietaria e quindi possibile causa di problemi non risolvibili (anche in presenza di \glo{API}{API} pubbliche per la creazione di plug-in, potrebbero sorgere problemi nella parte privata di \glo{Slack}{Slack}). \\
È poi richiesto documentare le \glo{API}{API} prodotte con swagger, tecnologia non ancora conosciuta dai membri del \glo{Gruppo}{gruppo}.


\subsection {Valutazione finale}
Questo \glo{Capitolato}{capitolato} stuzzica la curiosità del \glo{Gruppo}{gruppo}: gli assistenti virtuali sono ormai diffusi in molteplici ambiti. Conoscerli e saperli programmare rappresenterebbe certamente un valore aggiunto. Inoltre è richiesto usufruire dei servizi web di Amazon, altra tecnologia ampiamente adoperata dalle aziende. \\
Nonostante le moderne ed interessanti tecnologie richieste dal \glo{Capitolato}{capitolato}, il \glo{Gruppo}{gruppo} ritiene che il carico di lavoro necessario risulti essere troppo dispendioso in termini di tempo e denaro e perciò ha deciso di non candidarsi.
