
\section {Capitolato C5}


\subsection {Descrizione}
L’obiettivo del progetto è lo sviluppo di Monolith, un framework che permetta la creazione di bolle interattive per aumentare l’efficienza della comunicazione via chat e aggiungere nuove funzionalità. È richiesto di implementare il framework come pacchetto della piattaforma di messaggistica Rocket.chat. \\
I tipi di bolle realizzabili dovranno essere almeno uno tra:
\begin{itemize}
	\item bolle che possano riprodurre media;
	\item bolle con contenuto auto-aggiornante;
	\item bolle con contenuto modificabile.
\end{itemize}
È inoltre richiesto almeno un applicativo d'esempio che sfrutti il framework creato.


\subsection {Dominio applicativo}
Il framework dovrà essere utilizzabile nella piattaforma di messaggistica Rocket.chat. \\
I casi d'uso per le bolle create con Monolith sono vari, come l'interazione con clienti di aziende via chat, l'ausilio alla compilazione di form, l'invio di documenti o le risposte automatizzate di smart bot.


\subsection {Dominio tecnologico}
È richiesto l’utilizzo delle seguenti tecnologie:
\begin{itemize}
	\item \textbf{\glo{JavaScript}{JavaScript}}: sesta edizione (ES6);
	\item \textbf{ESLint}: strumento per l’identificazione di pattern nel codice \glo{JavaScript}{JavaScript}. Ne è richiesto l'utilizzo per aderire alla guida di stile \glo{JavaScript}{JavaScript} di Airbnb;
	\item \textbf{Heroku}: piattaforma online che permette agli sviluppatori di eseguire applicazioni interamente nel \glo{Cloud}{cloud};
	\item \textbf{\glo{Git}{Git}}: software di controllo di versione;
	\item \textbf{\glo{GitHub}{GitHub}/Bitbucket}: servizi di hosting di \glo{Repository}{repository} \glo{Git}{Git}.
\end{itemize}

È inoltre consigliato:
\begin{itemize}
	\item aderire alle linee guida “12 fattori”;
	\item utilizzare framework di frontend basato su \glo{JavaScript}{JavaScript} (Angular 2/React);
	\item utilizzare SCSS: sintassi utilizzata da SASS3 (estensione del \glo{CSS}{CSS}).
\end{itemize}


\subsection {Criticità}
Durante l’analisi del \glo{Capitolato}{capitolato} il \glo{Gruppo}{gruppo} ha rilevato le seguenti criticità:
\begin{itemize}
	\item documentazione esterna da scrivere completamente in inglese. Il \glo{Gruppo}{gruppo} ritiene che questo possa essere un ostacolo aggiuntivo;	% non mi piace tanto l'ultima frase
	\item sensazione che sia richiesta un’eccessiva rigidità per quanto concerne la parte di codifica;
	\item scarso interesse per il dominio applicativo proposto.
\end{itemize}

\subsection {Valutazione finale}
Nonostante l’utilizzo di tecnologie moderne, il \glo{Gruppo}{gruppo} ha deciso di scartare il progetto presentato nel \glo{Capitolato}{capitolato}. I motivi della scelta riguardano la lingua in cui la documentazione dev’essere realizzata e i pattern da seguire per la parte di codifica, sconosciuti a tutti i membri e considerati troppo vincolanti.
