
\section {Capitolato C1}

\subsection {Descrizione}
Il \glo{Capitolato}{capitolato} proposto da italianaSoftware prevede la realizzazione di un \glo{API}{API} Market per microservizi.\\
Il sistema dovrà essere realizzato con un'architettura a microservizi e consisterà in un applicazione web che permetta almeno di gestire le \glo{API}{API} e la relativa documentazione. Per ogni \glo{API}{API} verranno inoltre gestite le \glo{API}{API} key rilasciate per ogni utente e si potranno visualizzare dati tecnici e statistici. \\
Possibili obiettivi estesi prevedono la definizione di policy di vendita per ogni \glo{API}{API}, la gestione di eventuali SLA (Service Level Agreement), una moneta virtuale e la possibilità di inserire aspetti "social" nel market quali commenti e recensioni utente.


\subsection {Dominio applicativo}
L'azienda mira alla rivoluzione dell'IT, portando questo settore verso soluzioni distribuite a microservizi, i quali, secondo la visione futuristica di italianaSoftware, riguarderanno ogni aspetto della produzione di software: dalla progettazione alla messa in esecuzione.\\
L'\glo{API}{API} Market, consistente in un'applicazione web, dovrà essere accessibile agli utenti interessati alla compravendita, composizione e gestione di \glo{API}{API} basate su microservizi e consultazione della relativa documentazione.\\


\subsection {Dominio tecnologico}
Viene richiesto l’utilizzo delle seguenti tecnologie:
\begin{itemize}
	\item \textbf{Jolie:} linguaggio di programmazione, \glo{Open source}{open source}, orientato ai servizi; consigliato per la rappresentazione delle interfacce e per la creazione dell'\glo{API}{API} Gateway;
	\item \textbf{\glo{JavaScript}{JavaScript}, \glo{HTML}{HTML}, \glo{CSS}{CSS3}, Bootstrap:} per la parte di visualizzazione e comportamentale;
	\item \textbf{\glo{Database}{database} SQL/NoSQL:} a scelta.
	\item \textbf{il servizio Swagger}: per mostrare e standardizzare la documentazione delle \glo{API}{API}.
\end{itemize}


\subsection {Criticità}
È difficile mettere in comunicazione, manutenere e testare microservizi; il tempo necessario per l'apprendimento del paradigma a microservizi usato da Jolie è dispendioso. \\
Inoltre Jolie è poco utilizzato e diffuso nelle attuali realtà lavorative. \\
In caso il progetto sia affrontato da più gruppi dovranno essere considerati anche gli obiettivi estesi. La complessità del progetto potrebbe dunque crescere nel caso in cui gli altri gruppi scelgano questo stesso \glo{Capitolato}{capitolato} (non prevedibile a priori).


\subsection {Valutazione finale}
Il \glo{Gruppo}{gruppo} ha individuato i seguenti aspetti positivi:
\begin{itemize}
	\item{il tema di sviluppo di una piattaforma e-commerce riscontra buon interesse tra i membri del \glo{Gruppo}{gruppo};}
	\item{l’architettura a microservizi:} è un architettura emergente e viene già utilizzata da importanti aziende come Netflix e Amazon;
	\item{l’utilizzo di Jolie offre la possibilità di estendere il linguaggio stesso e di aggiungere nuove funzionalità;}
	\item{il team di italianaSoftware offre altresì una formazione minima necessaria all'apprendimento della tecnologia Jolie.}
\end{itemize}

Dopo aver valutato aspetti positivi e negativi è stato deciso di escludere il \glo{Capitolato}{capitolato}, in quanto i membri del \glo{Gruppo}{gruppo} non trovano di loro interesse apprendere Jolie, linguaggio poco utilizzato e diffuso, e il relativo paradigma a microservizi. Quest'ultimo non interessa poiché ancora emergente e quindi di incerta diffusione.
