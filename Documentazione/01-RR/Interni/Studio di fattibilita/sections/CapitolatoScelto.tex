
\section {Capitolato Scelto - C3}


\subsection {Descrizione}
Il \glo{Capitolato}{capitolato} proposto da RiskAPP richiede la realizzazione di un’interfaccia web fruibile sia da desktop che da tablet (con la possibilità di utilizzare le gesture tipiche: drag, pinch, point). L’applicazione dovrà permettere di inserire e visualizzare su mappa geografica gli impianti organizzativi e produttivi delle aziende e di disegnare possibili scenari di danno che possano colpire le stesse. I dati inseriti dovranno poi essere inviati ad un server di analisi che risponderà in maniera asincrona con i risultati ottenuti dall'elaborazione di tali dati.


\subsection {Dominio applicativo}
%pg13, C1.pdf
L'obiettivo è la creazione di un front-end che faciliti l'inserimento dei dati (soprattutto geografici) relativi ai processi produttivi di un'azienda e migliori la rappresentazione degli scenari di danno rispetto all'iterfaccia attuale, la quale dovrà dunque essere user-friendly. \\
L'applicazione sviluppata potrebbe poi essere integrata nella piattaforma di prodotto offerta da RiskAPP.


\subsection {Dominio tecnologico}
Lo stack tecnologico da utilizzare per la realizzazione dell'oggetto del \glo{Capitolato}{capitolato} è a scelta del fornitore, ma data la natura dell'applicazione da sviluppare saranno comunque indispensabili conoscenze nell'ambito web, in particolare \glo{HTML}{HTML}, \glo{CSS}{CSS}, \glo{JavaScript}{JavaScript}; \\
Possibili tecnologie utilizzabili nel progetto e già impiegate dal proponente:
\begin{itemize}
	\item Python3 e framework Django;
	\item Varie librerie per \glo{JavaScript}{JavaScript} come React, hammer.js e Yeoman;
	\item Bootstrap.
\end{itemize}
Sarà necessario inoltre acquisire dimestichezza con le \glo{API}{API} del prodotto offerto da RiskAPP con cui bisognerà interfacciarsi.


\subsection {Criticità}
Durante l'analisi del \glo{Capitolato}{capitolato} sono emerse le seguenti criticità:
\begin{enumerate}
	\item la realizzazione dell'applicazione è strettamente legata all'utilizzo di \glo{API}{API} già esistenti e questo potrebbe limitare le possibilità di sviluppo. Inoltre viene richiesto tempo aggiuntivo per lo studio delle \glo{API}{API} del prodotto attualmente esistente;
	\item la libertà di decidere le tecnologie da usare e l'ampiezza di scelta potrebbe allungare le tempistiche necessarie per scegliere la tecnologia migliore per lo sviluppo del progetto. %impiegare
\end{enumerate}


\subsection {Valutazione finale}

Il \glo{Capitolato}{capitolato} d'appalto presenta alcune caratteristiche positive che lo hanno portato ad essere la scelta finale del \glo{Gruppo}{gruppo}:
\begin{itemize}
	\item interesse nel dominio tecnologico;
	\item acquisizione di esperienza nello sviluppo di applicazioni web, con l'utilizzo di tecnologie ampiamente richieste nell’ambito lavorativo;
	\item discreto interesse nell'ambito proposto e nel prodotto finale che si presuppone supportare il lavoro di grandi aziende e ciò è certamente un valore aggiunto per il curriculum dei membri del \glo{Gruppo}{gruppo}.

\end{itemize}
