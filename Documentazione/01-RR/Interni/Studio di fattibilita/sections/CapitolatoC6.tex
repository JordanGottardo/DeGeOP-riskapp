
\section {Capitolato C6}


\subsection {Descrizione}
Il \glo{Capitolato}{capitolato} proposto da Zucchetti richiede la realizzazione di un programma capace di modellare diagrammi secondo lo standard \glo{UML}{UML} e di produrre codice, java o javascript, a partire da essi. L'idea esposta riguarda l'utilizzo di:
\begin{itemize}
	\item diagrammi di classe per definire la struttura del programma e i suoi prototipi;
	\item diagrammi delle attività per definire le implementazioni e i comportamenti delle componenti.
\end{itemize}
Il proponente ha lasciato ampia scelta su possibili cambiamenti e implementazioni.


\subsection {Dominio applicativo}
L'applicativo, secondo il \glo{Capitolato}{capitolato}, non presenta un target specifico. Si può supporre che verrà prevalentemente utilizzato da analisti e sviluppatori software, grazie alla generazione automatica del codice. Potrà anche essere usato da chiunque abbia necessità di modellare diagrammi \glo{UML}{UML}.


\subsection {Dominio tecnologico}
Il proponente ha richiesto che il prodotto sia sviluppato come applicazione web con architettura client-server. In particolare viene richiesto di utilizzare \glo{HTML}{HTML5}, \glo{CSS}{CSS} e \glo{JavaScript}{JavaScript} per il client e Tomcat(Java) o node.js per il server.


\subsection {Criticità}
Durante l’analisi del \glo{Capitolato}{capitolato}, il principale oggetto di critica è stato lo standard \glo{UML}{UML}. Questo risulta datato ed eccessivamente rigido. Inoltre, non essendo nato per la produzione di codice, i costrutti messi a disposizione non risultano adatti alla sua generazione: potrebbe essere richiesta un’estensione del linguaggio. \\
La natura del progetto implica piena conoscenza dello standard \glo{UML}{UML}, delle sue potenzialità e di esperienze
%critiche
sull'utilizzo di tale standard. Lo studio che ne seguirebbe potrebbe richiedere più tempo di quello a disposizione per lo sviluppo del progetto.


\subsection {Valutazione finale}
L’utilizzo dello standard \glo{UML}{UML} non solo come ausilio alla documentazione di un progetto ma anche come strumento di generazione del codice risulta un argomento abbastanza apprezzato ma le difficoltà previste nello studio del dominio hanno fatto desistere i membri del \glo{Gruppo}{gruppo} dallo scegliere il \glo{Capitolato}{capitolato} preso in esame.
