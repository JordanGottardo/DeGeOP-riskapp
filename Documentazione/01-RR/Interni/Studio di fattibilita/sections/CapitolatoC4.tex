
\section {Capitolato C4}

\subsection {Descrizione}
Il progetto presentato dall’azienda Mivoq S.r.l. prevede la realizzazione di un’applicazione per dispositivi mobili (smartphone e tablet). Questa deve agevolare la lettura per una persona affetta da dislessia. L’applicazione finale potrà consistere in un lettore di ebook o client di messaggistica.


\subsection {Dominio applicativo}
La sintesi vocale è una tecnologia che permette la conversione di un qualsiasi testo in audio. Negli ultimi anni si è assistito ad un rapido diffondersi di questa tecnologia in numerosi ambiti: le voci guida dei navigatori satellitari, gli annunci dei mezzi di trasporto pubblico, centralini telefonici, lettori di messaggi e, in generale, tutti quegli ambiti in cui è necessaria la lettura automatica a voce di un qualsiasi testo.


\subsection {Dominio tecnologico}
Le tecnologie da sfruttare comprendono:
\begin{itemize}
	\item FA-TTS (Flexible and Adaptive Text to Speech) ovvero un motore di sintesi vocale online;
	\item la piattaforma Android;
	\item applicazioni di terze parti per la gestione del testo (da scegliere liberamente).
\end{itemize}


\subsection {Criticità}
Il tema della dislessia risulta una tematica molto di nicchia. il progetto sviluppato risulterebbe perciò un prodotto poco vendibile rispetto ad altre proposte o prodotti maggiormente orientati alla massa. \\
Il \glo{Gruppo}{gruppo} ha mostrato qualche perplessità riguardo MiVoq la quale, essendo una start-up, non ha la stessa solidità e struttura di altre società. \\
Sono già disponibili sul mercato prodotti simili (es: Voice Dream), sviluppati da competitori che hanno un importante vantaggio iniziale soprattutto in termini di clientela; dunque queste aziende non avrebbero difficoltà a colmare in tempi brevi un eventuale divario con la nostra applicazione.


\subsection {Valutazione finale}
Il proponente mostra grande disponibilità a fornire supporto per le fasi di apprendimento delle tecnologie necessarie allo sviluppo del prodotto e la programmazione in ambiente Android riscontra un discreto successo tra i membri del \glo{Gruppo}{gruppo}. \\
Il \glo{Gruppo}{gruppo} sente inoltre uno stimolo motivazionale a creare un prodotto che possa aiutare il prossimo. \\
% Infine la possibilità di mantenere la paternità e proprietà del prodotto è un punto a favore. \\
Il \glo{Gruppo}{gruppo} ha deciso di non scegliere questo \glo{Capitolato}{capitolato} poiché i precedenti aspetti positivi non controbilanciano gli aspetti negativi.
