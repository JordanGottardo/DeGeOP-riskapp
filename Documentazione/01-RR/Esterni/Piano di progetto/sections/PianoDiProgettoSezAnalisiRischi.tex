%\section {Analisi dei rischi}
\section {Analisi dei rischi}
\label{sec:analisirischi}
	\subsection {Introduzione}
	In questa sezione vengono descritti i rischi che potrebbero verificarsi durante lo svolgimento del progetto.
	I rischi possono essere a livello:
	\begin{itemize}
	 \item {tecnologico};
	 \item {personale};
	 \item {organizzativo};
	 \item {dei requisiti}.
	\end{itemize}
	Per ogni rischio deve essere indicato:
	 \begin{itemize}
	 \item {nome};
	 \item {descrizione};
	 \item {possibili conseguenze};
	 \item {analisi};
	 	\begin{itemize}
	 	\item \textbf{probabilità di occorrenza}: (alta/media/bassa) mostra una stima preventiva della probabilità che il rischio si verifichi senza aver attuato alcuna misura di prevenzione;
	 	\item \textbf{impatto}: (alto/medio/basso) mostra una stima preventiva dell'impatto del rischio, nel caso in cui si verificasse senza aver deciso di attuare alcuna procedura di contenimento.
	 	\end{itemize}
	 \item \textbf{{prevenzione}:} spiega come si è deciso di prevenire il verificarsi del rischio;
	 \item \textbf{{identificazione ed attivazione procedura di contenimento}:} mostra come il \glo{Gruppo}{gruppo} riuscirà a capire che il rischio si sta verificando e come verrà attivata la procedura di contenimento;
	 \item \textbf{{contenimento}:} spiega come si è deciso di contenere le conseguenze del rischio una volta verificato. Oltre alle misure descritte successivamente per ogni specifico rischio, si è deciso di inserire dei tempi di \glo{Slack}{slack} nel piano di lavoro, sfruttabili nel caso in cui il \responsabilediprogetto{} lo ritenesse necessario al verificarsi dei rischi.
	 \item {analisi mitigata};
	 \begin{itemize}
	 	\item \textbf{probabilità mitigata di occorrenza}: (alta/media/bassa) mostra una stima preventiva della probabilità che il rischio si verifichi dopo aver attuato le opportune misure di prevenzione sopra descritte;
	 	\item \textbf{impatto mitigato}: (alto/medio/basso) mostra una stima preventiva dell'impatto del rischio, nel caso in cui si verificasse, decidendo di attuare le opportune misure di contenimento sopra descritte.
	 \end{itemize}

	 \item \textbf{{attualizzazione nel periodo}:} viene descritto se il rischio si è verificato, le reazioni del \glo{Gruppo}{gruppo} e le conseguenze effettivamente prodotte.
	 \end{itemize}



	\subsection {Panoramica generale}
	Viene mostrata una tabella che illustra i rischi individuati, la loro probabilità mitigata di occorrenza, il loro impatto mitigato.
	\begin{table}[H]
	\begin{center}
	\small
		\begin{tabular}{lllll}
			\toprule
				Livello & Rif. & Rischio & Probabilità & Impatto \\
			\midrule
				\multirow{3}{*}{Tecnologico}
				& (\ref{subsec:malfunzionamenttiSwHw})
				& Malfunzionamenti software e hardware
				& Bassa & Basso \\
				& (\ref{subsec:difficoltaTecnol})
				& Difficoltà nell'uso delle tecnologie
				& Media & Medio \\
				& (\ref{subsec:indisponibilitaTablet})
				& Indisponibilità tablet
				& Media & Basso \\
			\midrule
				\multirow{3}{*}{Personale}
				& (\ref{subsec:pbmDeiMembri})
				& Problemi dei membri
				& Media & Medio \\
				& (\ref{subsec:pbmTraMembri})
				& Problemi tra i membri
				& Bassa & Medio\\
				& (\ref{subsec:pbmProponente})
				& Problemi con il proponente
				& Bassa & Medio \\

			\midrule
				Organizzativo & (\ref{subsec:errataPianificazione})
				& Errata pianificazione dell'uso delle risorse
				& Bassa & Medio \\
			\midrule
				Dei Requisiti & (\ref{subsec:erratiRequisiti})
				& Errata o incompleta analisi dei requisiti
				& Bassa & Medio \\
			\bottomrule
			\end{tabular}
		\end{center}
		\caption{Rischi individuati}
		\label{tab:rischi}
	\end{table}

	\subsection {Livello tecnologico}
		\subsubsection {Malfunzionamenti software e hardware}
		\label{subsec:malfunzionamenttiSwHw}
			\paragraph{Descrizione} Ogni membro del \glo{Gruppo}{gruppo} userà i propri dispositivi (computer, smartphone, tablet). Non si escludono rotture o danneggiamenti di tali dispositivi, malfunzionamenti dei software necessari allo svolgimento del progetto o problemi di compatibilità.
			\paragraph{Possibili conseguenze} Ritardi nei lavori, perdita dei dati.
			\paragraph{Analisi}
			\begin{itemize}
			\item{probabilità di occorrenza:} bassa;
			\item{impatto:} alto.
			\end{itemize}
			\paragraph{Prevenzione} I membri del \glo{Gruppo}{gruppo} eseguiranno backup regolari nel \glo{Repository}{repository}.
			\paragraph{Identificazione e attivazione procedura di contenimento}
			I membri del \glo{Gruppo}{gruppo} dovranno controllare periodicamente i propri dispositivi, i software utilizzati ed i dati prodotti. In caso di malfunzionamenti avviseranno il \responsabilediprogetto.
			\paragraph{Contenimento}
			Il \responsabilediprogetto, dopo essersi consultato con i membri che hanno riscontrato il malfunzionamento, potrà eventualmente decidere di alleggerire o modificare il loro carico di lavoro. Eventuali dati persi a causa del malfunzionamento dovranno, se possibile, essere ripristinati dal \glo{Repository}{repository}. In caso di necessità si potranno usare i computer dei laboratori, messi a disposizione dall'università, per la continuazione del lavoro.
			\paragraph{Analisi mitigata}
			\begin{itemize}
			\item{probabilità mitigata di occorrenza:} bassa;
			\item{impatto mitigato:} basso.
			\end{itemize}
			\paragraph{Attualizzazione nel periodo}
				\begin{itemize}
				\item{An - Analisi}: il rischio non si è verificato.
				\end{itemize}

		\subsubsection {Difficoltà nell'uso delle tecnologie}
		\label{subsec:difficoltaTecnol}
			\paragraph{Descrizione}
			Per lo svolgimento del progetto sarà necessario utilizzare tecnologie con le quali non tutti i membri del \glo{Gruppo}{gruppo} hanno esperienza. Potrebbero quindi sorgere difficoltà nell'utilizzo di queste tecnologie.
			\paragraph{Possibili conseguenze} Ritardi nei lavori.
			\paragraph{Analisi}
			\begin{itemize}
			\item{probabilità di occorrenza:} alta;
			\item{impatto:} alto.
			\end{itemize}
			\paragraph{Prevenzione}
			Le tecnologie da usare verrano decise con congruo anticipo. Saranno previsti nel piano di lavoro momenti di formazione. Gli \amministratori{} forniranno la documentazione e i riferimenti necessari allo studio autonomo delle tecnologie.
			\paragraph{Identificazione e attivazione procedura di contenimento}
			Sarà compito del \responsabilediprogetto{} e degli \amministratori{} verificare il grado di conoscenza delle tecnologie richieste. I membri del \glo{Gruppo}{gruppo} che dovessero trovare difficoltà nell'uso delle tecnologie richieste durante lo svolgimento dovranno avvisare il \responsabilediprogetto.
			\paragraph{Contenimento}
			Il \responsabilediprogetto{} potrà sollevare momentaneamente il membro carente dal proprio incarico. Il piano di lavoro potrà essere variato per permettere al membro carente di aggiornarsi nel minor tempo possibile. Nel caso in cui il membro non riesca a utilizzare la tecnologia allora dovrà essere sostituito. Se possibile la tecnologia potrà essere sostituita da una tecnologia equivalente.
			\paragraph{Analisi mitigata}
			\begin{itemize}
			\item{probabilità mitigata di occorrenza:} media;
			\item{impatto mitigato:} medio.
			\end{itemize}
			\paragraph{Attualizzazione nel periodo}
				\begin{itemize}
				\item \textbf{\textbf{{An - Analisi}:}} Lo strumento di gestione di progetto scelto inzialmente si è dimostrato poco flessibile, carente di alcune funzionalità e poco integrato negli smartphone. Il \responsabilediprogetto{} e gli \amministratori{} hanno deciso di sostituire lo strumento con un altro più completo e maggiormente integrato. Ci sono stati inoltre problemi con il software di tracciamento dei requisiti. Il \responsabilediprogetto, dopo aver provato a modificare lo strumento, ha deciso di sostituirlo con uno equivalente.
				\end{itemize}

		\subsubsection{Indisponibilità tablet}
		\label{subsec:indisponibilitaTablet}
					\paragraph{Descrizione} Il progetto richiede lo sviluppo di un'interfaccia utilizzabile da tablet. Specialmente per quanto riguarda l'attività di testing sarà quindi necessario avere almeno un tablet a disposizione su cui testare il funzionamento dell'applicazione. Due membri sono dotati di tablet personale, ma è da prendere in considerazione il caso in cui non ci sia la possibilità di averli a disposizione.
					\paragraph{Possibili conseguenze} Ritardi nei lavori.
					\paragraph{Analisi}
					\begin{itemize}
					\item{probabilità di occorrenza:} alta;
					\item{impatto:} medio.
					\end{itemize}
					\paragraph{Prevenzione} I due membri del \glo{Gruppo}{gruppo} comunicano al \responsabilediprogetto{} i giorni in cui offrono la disponibilità del tablet.
					\paragraph{Identificazione}
					Nel caso in cui i due membri del \glo{Gruppo}{gruppo} non possano mettere a disposizione il tablet nei giorni stabiliti, informeranno preventivamente il \responsabilediprogetto{}.
					\paragraph{Contenimento}
					Il \responsabilediprogetto{} comunicherà il problema agli altri membri del \glo{Gruppo}{gruppo} che dovranno quindi eseguire i test su smartphone o strumenti equivalenti.
			\paragraph{Analisi mitigata}
			\begin{itemize}
			\item{probabilità mitigata di occorrenza:} media;
			\item{impatto mitigato:} basso.
			\end{itemize}
					\paragraph{Attualizzazione nel periodo}
						\begin{itemize}
						\item{An - Analisi}: Il rischio non si è verificato.
						\end{itemize}

	\subsection {Livello personale}
		\subsubsection {Problemi dei membri}
		\label{subsec:pbmDeiMembri}
			\paragraph{Descrizione} Gli impegni personali e universitari dei singoli membri possono far sì che non sempre questi siano disponibili a lavorare sul progetto o a presenziare alle riunioni.
			\paragraph{Possibili conseguenze} Ritardi nei lavori.
			\paragraph{Analisi}
			\begin{itemize}
			\item{probabilità di occorrenza:} alto;
			\item{impatto:} alto.
			\end{itemize}
			\paragraph{Prevenzione} Viene predisposto un orario settimanale condiviso su cui segnare gli impegni ricorrenti dei membri del \glo{Gruppo}{gruppo} per facilitare l'organizzazione delle riunioni. Il \responsabilediprogetto{} dovrà scegliere per le riunioni i giorni e gli orari con il maggior numero di presenze. Durante la pianificazione del lavoro si dovrà tener conto dei periodi di maggiore indisponibilità dei membri come, ad esempio, sessioni d'esame.
			\paragraph{Identificazione e attivazione procedura di contenimento}
			In caso di imprevisti i membri del \glo{Gruppo}{gruppo} dovranno informare in maniera tempestiva il \responsabilediprogetto{} sfruttando gli strumenti di comunicazione messi a disposizione.
			\paragraph{Contenimento} In caso si tratti di imprevisti sulla partecipazione ad una riunione, il \responsabilediprogetto{} potrà decidere di spostare la riunione. Al termine di ogni riunione viene stilato un verbale di cui il membro mancante potrà prendere visione. In caso si tratti di imprevisti sul lavoro da svolgere, il \responsabilediprogetto{} potrà provvedere a modificare la pianificazione e se necessario, riassegnare il lavoro.
			\paragraph{Analisi mitigata}
			\begin{itemize}
				\item{probabilità mitigata di occorrenza:} media;
				\item{impatto mitigato:} medio.
			\end{itemize}
			\paragraph{Attualizzazione nel periodo}
				\begin{itemize}
				\item \textbf{{An - Analisi}:} Il rischio si è in parte verificato in quanto un membro del \glo{Gruppo}{gruppo}, essendo impegnato in attività di stage, non ha potuto partecipare ad alcune riunioni. Il membro si è tenuto aggiornato con le decisioni prese attraverso i canali di comunicazione e prendendo visione di tutti i verbali.
				\end{itemize}

		\subsubsection {Problemi tra i membri}
		\label{subsec:pbmTraMembri}
			\paragraph{Descrizione}
			I membri del \glo{Gruppo}{gruppo} non hanno mai lavorato ad un progetto di gruppo di così grandi dimensioni. Potrebbero verificarsi contrasti tra i membri del \glo{Gruppo}{gruppo}, e quindi problemi di collaborazione.
			\paragraph{Possibili conseguenze} Ambiente non adatto al lavoro cooperativo, ritardi nei lavori.
			\paragraph{Analisi}
			\begin{itemize}
			\item{probabilità di occorrenza:} bassa;
			\item{impatto:} alto.
			\end{itemize}
			\paragraph{Prevenzione} I membri del \glo{Gruppo}{team} cercheranno di fornire critiche solo se costruttive.
			\paragraph{Identificazione e attivazione procedura di contenimento}
			Ogni membro del \glo{Gruppo}{gruppo} comunicherà al \responsabilediprogetto{} eventuali dissapori al fine di risolvere al più presto eventuali situazioni problematiche.
			\paragraph{Contenimento} Il \responsabilediprogetto{} dovrà capire il problema e appianare le divergenze. Se necessario potrà provvedere alla riorganizzazione del piano di lavoro separando i membri coinvolti.
			\paragraph{Analisi mitigata}
			\begin{itemize}
			\item{probabilità mitigata di occorrenza:} bassa;
			\item{impatto mitigato:} medio.
			\end{itemize}
			\paragraph{Attualizzazione nel periodo}
				\begin{itemize}
				\item{An - Analisi}: Il rischio non si è verificato.
				\end{itemize}

		\subsubsection {Problemi col proponente}
		\label{subsec:pbmProponente}
			\paragraph{Descrizione}
			Potrebbero verificarsi ritardi da parte del proponente nelle comunicazioni o nel fornire il materiale richiesto.
			\paragraph{Possibili conseguenze} Ritardi nei lavori.
			\paragraph{Analisi}
			\begin{itemize}
			\item{probabilità di occorrenza:} media;
			\item{impatto:} alto.
			\end{itemize}
			\paragraph{Prevenzione} Le scadenze a cui è soggetto il \glo{Gruppo}{gruppo} saranno rese note al proponente al fine di migliorare le tempistiche di collaborazione.
			\paragraph{Identificazione e attivazione procedura di contenimento}
			Nel caso in cui non si ottenessero le risposte o il materiale atteso entro una settimana dalle richieste, il \responsabilediprogetto{} provvederà ad inviare un sollecito.
			\paragraph{Contenimento} Il \responsabilediprogetto{} potrà valutare se riorganizzare il piano di lavoro per procedere con l'esecuzione di altri lavori non dipendenti dalle risposte del proponente.
			\paragraph{Analisi mitigata}
			\begin{itemize}
			\item{probabilità mitigata di occorrenza:} bassa;
			\item{impatto mitigato:} medio.
			\end{itemize}
			\paragraph{Attualizzazione nel periodo}
				\begin{itemize}
				\item \textbf{{An - Analisi}:} Il rischio si è in parte verificato in quanto il proponente non ha sempre risposto prontamente alle e-mail. Inoltre parte del materiale richiesto è pervenuto tre giorni dopo rispetto a quanto previsto.
				\end{itemize}

	\subsection{Livello organizzativo}
		\subsubsection {Errata pianificazione di uso delle risorse}
			\label{subsec:errataPianificazione}
			\paragraph{Descrizione} È possibile che le stime dei tempi e dei costi siano troppo ottimistiche.
			\paragraph{Possibili conseguenze} Ritardi nella consegna, aumento del costo preventivato.
			\paragraph{Analisi}
			\begin{itemize}
			\item{probabilità di occorrenza:} media;
			\item{impatto:} alto.
			\end{itemize}
			\paragraph{Prevenzione} La pianificazione iniziale dovrà essere essere eseguita con particolare attenzione e sottoposta a più verifiche per assicurare che non sia troppo ottimistica. I membri del \glo{Gruppo}{gruppo} dovranno utilizzare sistemi di rendicontazione del tempo di lavoro per assicurare il rispetto dei tempi previsti.
			\paragraph{Identificazione e attivazione procedura di contenimento} Dovranno essere utilizzati sistemi automatici che permettano al \responsabilediprogetto{} di controllare in maniera rapida ed efficiente lo stato dei lavori.
			\paragraph{Contenimento} Il \responsabilediprogetto{} potrà modificare il piano di lavoro al fine di recuperare il ritardo. Nel caso in cui il ritardo accumulato sia eccessivo si dovrà valutare col proponente di ritrattare la scadenza.
			\paragraph{Analisi mitigata}
			\begin{itemize}
			\item{probabilità mitigata di occorrenza:} bassa;
			\item{impatto mitigato:} medio.
			\end{itemize}
			\paragraph{Attualizzazione nel periodo}
				\begin{itemize}
				\item{An - Analisi}: Il rischio non si è verificato.
				\end{itemize}

	\subsection {Livello dei requisiti}
		\subsubsection {Errata o incompleta analisi dei requisiti}
			\label{subsec:erratiRequisiti}
			\paragraph{Descrizione} È possibile che i requisiti individuati non rispecchino in maniera completa ed esaustiva le richieste del proponente.
			\paragraph{Possibili conseguenze:} Ritardi nei lavori.
			\paragraph{Analisi}
			\begin{itemize}
			\item{probabilità di occorrenza:} media;
			\item{impatto:} alto.
			\end{itemize}
			\paragraph{Prevenzione} Saranno organizzate riunioni con il proponente durante le quali il \glo{Gruppo}{gruppo} potrà mostrare il lavoro svolto, ricevere eventuali feedback e porre domande. Verrà inoltre predisposta una chat condivisa con il proponente che potrà essere utilizzata per chiarire eventuali dubbi in maniera più immediata rispetto all'email o alle riunioni.
			\paragraph{Identificazione e attivazione procedura di contenimento}
			Durante l'intera durata del progetto ogni membro del \glo{Gruppo}{gruppo} dovrà porre particolare attenzione alle richieste del proponente. Qualsiasi dubbio sulle richieste dovrà essere esposto al \responsabilediprogetto.
			\paragraph{Contenimento} Nel caso in cui il rischio si verificasse dovrà essere fatto il possibile per riadattarsi alle esigenze del proponente. Il \responsabilediprogetto{} potrà valutare a seconda dei casi se rinegoziare i requisiti con il proponente.
			\paragraph{Analisi mitigata}
			\begin{itemize}
			\item{probabilità mitigata di occorrenza:} bassa;
			\item{impatto mitigato:} medio.
			\end{itemize}
			\paragraph{Attualizzazione nel periodo}
				\begin{itemize}
				\item{An - Analisi}: Il rischio non si è verificato.
				\end{itemize}
