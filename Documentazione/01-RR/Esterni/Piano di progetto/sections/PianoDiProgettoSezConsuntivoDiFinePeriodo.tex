%\section {Consuntivo}

\section {Consuntivo}
	\subsection {Introduzione}
	In questa sezione viene presentato il bilancio orario ed economico del progetto. Al termine di ogni \glo{Fase}{fase} verrà steso un consuntivo di periodo, al termine del progetto verrà presentato un consuntivo finale.
	Il bilancio orario può essere:
	\begin{itemize}
	\item \textbf{positivo:} il preventivo orario ha superato il consuntivo orario;
	\item \textbf{negativo:} il consuntivo orario ha superato il preventivo orario;
	\item \textbf{in pari:} il preventivo orario conincide con il consuntivo orario.
	\end{itemize}
	Il bilancio economico può essere:
	\begin{itemize}
	\item \textbf{positivo:} il preventivo economico ha superato il consuntivo economico;
	\item \textbf{negativo:} il consuntivo economico ha superato il preventivo economico;
	\item \textbf{in pari:}  il preventivo economico conincide con il consuntivo economico.
	\end{itemize}
	\subsection {Consuntivi di periodo}
		\subsubsection {Fase: An - Analisi}
			\paragraph{Consuntivo orario}
%---------------------------------------------------------------------
	\begin{table}[H] \begin{center} \begin{tabular}{llllllll}
	\toprule
	\textbf{Nominativo}	&	\textbf{Re}		&	\textbf{Am}		&	\textbf{At}		&	\textbf{Pj}		&	\textbf{Pr}		&	\textbf{Ve}		&	\textbf{Tot}		 \\
	\midrule																						 \\
	Brutesco	&	-		&	-		&	13	(+1)	&	-		&	-		&	12		&	25	(+1)	 \\
	Damo		&	-		&	9		&	16	(+1)	&	-		&	-		&	-		&	25	(+1) \\
	De Gaspari	&	-		&	-		&	13  (+1)	&	-		&	-		&	12		&	25	(+1) \\
	Gottardo	&	14		&	-		&	11	(+1)	&	-		&	-		&	-		&	25	(+1) \\
	Pasqualini	&	-		&	-		&	13	(+1)	&	-		&	-		&	12		&	25	(+1) \\
	Petenazzi	&	7		&	-		&	18	(+1)	&	-		&	-		&	-		&	25	(+1) \\
	Prete		&	-		&	10		&	15	(+1)	&	-		&	-		&	-		&	25	(+1) \\
	\midrule
	Tot in ore	&	21		&	19		&	99	(+7)	&	0		&	0		&	36		&	175	(+7)	 \\
	\bottomrule
	\end{tabular} \end{center} \caption{Prospetto orario a consuntivo per la \glo{Fase}{fase} di analisi}\label{tab:oreAnalisiCons}
    \end{table}
%---------------------------------------------------------------------
			\paragraph{Consuntivo economico}
			La differenza tra le ore a consuntivo e preventivo di questa \glo{Fase}{fase} non sono a carico del proponente e quindi saranno considerate solamente come ore di investimento non rendicontate.
%-----------------------------------------------------------------


							\begin{table}[H] \begin{center} \begin{tabular}{llllllll}
							\toprule
								&	\textbf{Re}	&	\textbf{Am}	&	\textbf{At}	&	\textbf{Pj}	&	\textbf{Pr}	&	\textbf{Ve}	&	\textbf{Tot}	 \\
							\midrule
							Tot. in ore	& 21	&	19 & 99(+7) &	-		&	-		&	36	&	182	(+7) \\
				Tot. in €	&	 €     630,00 		 & 	 €  380,00 		 & 	 €  2.475,00 		 & 	 €           -   		 & 	 €               -   		 & 	 €  540,00 		 & 	 €              4.025 		 \\
							\bottomrule
							\end{tabular} \end{center} \caption{Prospetto economico - Fase:
							An
							}\label{tab:sAnCons} \end{table}

%-----------------------------------------------------------------
			\paragraph{Conclusioni}
			Il lavoro degli \analisti{} ha richiesto più tempo di quello preventivato in quanto lo studio del dominio si è dimostrato più difficile di quanto previsto. Come si vede dalla tabella \ref{tab:oreAnalisiCons} il bilancio orario risulta negativo in quanto eccede di 7 ore rispetto a quanto pianificato.
			Come si vede dalla tabella \ref{tab:sAnCons} il bilancio economico è negativo per un importo pari a -175€.
			Queste variazioni rispetto al preventivo non avranno impatto sul costo finale in quanto le ore aggiuntive sono considerate di investimento.

	\subsection{Totale non rendicontato}
		\subsubsection{Consuntivo orario}
		Il consuntivo orario totale non rendicontato coincide col consuntivo orario per la \glo{Fase}{fase} di analisi.
		\subsubsection{Consuntivo economico}
		Il consuntivo orario totale non rendicontato coincide col consuntivo orario per la \glo{Fase}{fase} di analisi.
	\subsection{Totale complessivo}
		\subsubsection{Consuntivo orario}
        Il consuntivo orario totale complessivo coincide col consuntivo orario per la \glo{Fase}{fase} di analisi.
		\subsubsection{Consuntivo economico}
		Il consuntivo orario totale complessivo coincide col consuntivo orario per la \glo{Fase}{fase} di analisi.
	\subsection{Totale rendicontato}
		\subsubsection{Consuntivo orario}
		Il consuntivo orario totale rendicontato non viene riportato in quanto tutte le ore di questa \glo{Fase}{fase} sono non rendicontate.
		\subsubsection{Consuntivo economico}
		Il consuntivo economico totale rendicontato non viene riportato in quanto tutte le ore di questa \glo{Fase}{fase} sono non rendicontate.
