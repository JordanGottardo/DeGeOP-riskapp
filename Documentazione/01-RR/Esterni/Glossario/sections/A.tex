\section{A}

% \gref{Ad hoc}
% Indica che un programma, dei servizi o un sistema informativo sono fatti su misura, ovvero appositamente, per uno specifico problema e non per una versione generale dello stesso.
%%% %\url{http://amazon.com}. \\
%%%%\hyperref[Amazon]{http://Amazon.com} \\
%%%%\hyperref{http://google.com}{}{}{HEY} \\
%%%\href{http://amazon.com}{Amazon.com}

\gref{Amazon Web Services}
Collezione di servizi di cloud computing offerti da \href{https://aws.amazon.com/it/}{Amazon}. \\
Alcuni esempi di servizi sono:
\begin{itemize}
	\item calcolo;
	\item archiviazione;
	\item database.
\end{itemize}

\gref{API}
\textit{Application Programming Interface}. Insieme di procedure utilizzabili per interfacciarsi con un programma o un sistema informatico in modo standard. Spesso si intendono le librerie software disponibili in un certo linguaggio di programmazione.

\gref{Arco}
Collegamento orientato tra due \glo{Nodo}{nodi}.

\gref{Asset}
Fabbricato con importanza strategica per il processo produttivo di un'azienda. Un asset può contenere uno o più \glo{Nodo}{nodi}.

\gref{Astah}
Strumento per la modellazione di diagrammi \glo{UML}{UML}. Supporta la creazione dei seguenti tipi di diagrammi:
\begin{itemize}
	\item delle classi;
	\item dei casi d'uso;
	\item delle attività;
	\item di sequenza;
	\item dei packages.
\end{itemize} 
