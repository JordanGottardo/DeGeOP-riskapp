\documentclass[a4paper,11pt]{article}
\usepackage{../../../Template/zemplateVerb}

%\externaldocument{path/to/glossario}
\docVerbV{2}
\docTitle{Verbale esterno - \getdocVerbV}
\docVersion{1.0.0}
\docCreationDate{\frmdata{27}{12}{2016}}
\docLastUpdateDate{\frmdata{28}{12}{2016}}
\docStatus{Approvato}
\docEditors{Giovanni Damo}
\docVerificators{Daniel De Gaspari}
\docApprovers{Giulia Petenazzi}
\docUse{Esterno}
\docDestination{\Tullio \\ & \Cardin \\ & \zephyrus \\ & \riskapp}
\docJournal{
	1.0.0 & \frmdata{28}{12}{2016} & Giulia Petenazzi & \responsabile  & Approvazione \\
	0.1.0 & \frmdata{27}{12}{2016} & Daniel De Gaspari & \verificatore & Verifica documento\\
	0.0.1 & \frmdata{27}{12}{2016} & Giovanni Damo & \analista & Stesura documento\\
}

\begin{document}
	\section{Estremi della riunione}
	\begin{itemize}
		\item \textbf{data:} \frmdata{27}{12}{2016};
		\item \textbf{ora inizio:} \frmora{09}{30};
		\item \textbf{ora fine:} \frmora{13}{00};
		\item \textbf{luogo:} ufficio \riskapp{} con sede a Conselve - Padova;
		\item \textbf{segretario:} Giovanni Damo;
		\item \textbf{partecipanti:}
		\begin{itemize}
			 \item Daniel De Gaspari;
			 \item Federico Carturan (proponente);
			 \item Giovanni Damo;
			 \item Giovanni Prete;
			 \item Giulia Petenazzi;
			 \item Jordan Gottardo;
			 \item Leonardo Brutesco;
			 \item Marco Pasqualini;
			 \item Pierpaolo Toniolo (proponente).
		\end{itemize}
		\item \textbf{assenti:}
			\begin{itemize}
					 \item nessuno.
			\end{itemize}
	\end{itemize}
	\section{Ordine del giorno}
		\begin{itemize}
			\item esposizione e discussione dei casi d'uso;
			\item chiarimenti tecnici struttura applicazione.
		\end{itemize}
	\section{Verbale della riunione}
	\begin{itemize}
		\item viene esposto e disegnato un mock-up dell'interfaccia;
        \item il proponente commenta il mock-up evidenziando gli aspetti da rivedere o cambiare;
		\item sono chiariti i ruoli dei seguenti componenti: asset, archi, nodi e scenari;
		\item vengono discusse le differenze di rappresentazione tra le lavorazioni per processo e quelle per commessa;
		\item viene discusso come rappresentare sulla mappa gli scenari di danno;
        \item vengono esposti e discussi i casi d'uso preparati dal gruppo;
		\item assieme a Pierpaolo vengono discussi anche i seguenti aspetti tecnici:
		\begin{itemize}
			\item la realizzazione dell'applicazione in forma di web-app o stand-alone;
			\item la gestione della traduzione delle lingue;
			\item la gestione del login;
			\item la modalità di fruizione del tutorial e la sua realizzazione;
			\item possibilità di integrare il prodotto nell'applicazione del proponente e possibili complicazioni tecniche.
		\end{itemize}
	\end{itemize}
	\section{Decisioni prese}
	\begin{itemize}
		\itemVE  il proponente rinuncia al requisito, presente nel capitolato, riguardante il funzionamento off-line dell'applicazione. La realizzazione del prodotto sarà pertanto sotto forma di web-app;
		\itemVE presa in carico da parte del proponente della creazione delle API necessarie per la realizzazione del prodotto;
		\itemVE impegno da parte del gruppo a decidere lo stack tecnologico da utilizzare, in tempi brevi;
		\itemVE invio dei casi d'uso aggiornati e corretti per una revisione da parte del proponente;
		\itemVE la rappresentazione sulla mappa prenderà il considerazione un solo processo;
		\itemVE gli scenari di danno saranno rappresentati sulla mappa sotto forma di: poligono, gradiente radiale e gradiente lungo una linea;
		\itemVE visti i rapidi cambiamenti di versione dell'applicazione del proponente, è stato concordato che la versione utilizzata per lo sviluppo del progetto sarà quella disponibile all'inizio della fase di progettazione;
		\itemVE disponibilità da parte del proponente di fornire un ambiente con utente già autenticato.
		
	\end{itemize}
\end{document}
