\newpage



\section{Visione generale della strategia di gestione della qualità}
\subsection{Obiettivi di qualità}
In questa sezione vengono espressi gli obiettivi di qualità che il \glo{Gruppo}{team} si è prefissato. Data la difficoltà (e in alcuni casi l'impossibilità) nel misurare direttamente la qualità, sono stati scelti standard, modelli e metriche.
Ognuno di questi fa uso di scale differenti e fissate a priori. Per ogni criterio, il \glo{Gruppo}{team} ha fissato dei range di valori accettabili e ottimi. A prescindere dal livello raggiunto in ogni misurazione, l'obiettivo da perseguire è il miglioramento continuo della qualità, realizzata attraverso il ciclo \glo{PDCA}{PDCA}. Nel caso in cui non si raggiungesse l'obiettivo minimo, dovranno essere attuate misure correttive come previsto dalle \ndpv.	
Siccome la qualità non è una proprietà intrinseca dei processi, è fondamentale dotarsi di buoni strumenti per effettuare le misurazioni. È possibile trovare una descrizione di tali strumenti nellle \ndpv.

\subsubsection{Qualità di processo}
Garantire la qualità dei processi è fondamentale se si vogliono ottenere prodotti di qualità. L'unico modo di garantire \glo{Quality assurance}{quality assurance} è far sì che  i processi siano normati e misurati. Inoltre, è possibile anche ottenere maggiore efficienza, efficacia e ripetibilità dei risultati. Gli obiettivi relativi a questi ambiti sono illustrati nel \pdpv.
Le caratteristiche che i processi dovrebbero avere sono le seguenti:

\begin{itemize}
	\item un processo dovrebbe essere in grado di migliorare continuamente le proprie performance:
	\begin{itemize}
		\item le performance di un processo dovrebbero essere misurabili;
		\item un processo dovrebbe perseguire obiettivi quantitativi di miglioramento.
	\end{itemize}
	\item i processi e le loro attività dovrebbero rispettare i tempi e i costi stabiliti dal \pdpv.\\
\end{itemize}
Seguono gli obiettivi e le metriche riguardanti la qualità di processo che il \glo{Gruppo}{team} ha stabilito.

\paragraph{Miglioramento costante} 
\label{OMC}
Per quantificare il livello di performance raggiunto dai processi, si è deciso di adottare il modello \glo{CMM}{CMM}. L'obiettivo è migliorare costantemente tale livello, secondo quanto definito dal ciclo \glo{PDCA}{PDCA}.
\begin{itemize}
	\item \textbf{metrica utilizzata:} Livello \glo{CMM}{CMM} ($LCMM$). Fa riferimento alla scala stabilita dal modello \glo{CMM}{CMM};
	\item \textbf{soglia di accettabilità:} $LCMM \geq 2$;
	\item \textbf{soglia di ottimalità:} $LCMM \geq 4$.
\end{itemize}
Per una descrizione più dettagliata del modello \glo{CMM}{CMM}, si faccia riferimento all'\hyperref[appendice A]{appendice A}.  %\ref{appendice A}.
Per una descrizione più dettagliata della metrica consultare la sezione \ref{MMC}.


\paragraph{Rispetto della pianificazione}
\label{ORDP}
Rispettare la pianificazione del lavoro stabilita nel \pdpv{} è fondamentale per evitare ritardi e garantire la qualità del processo. Qualora non la si rispettasse, è molto probabile che il processo non abbia le caratteristiche di qualità desiderate.
\begin{itemize}
	\item \textbf{metrica utilizzata:} Schedule Variance ($SV$);
	\item \textbf{soglia di accettabilità:} $SV\leq 4${} giorni rispetto a quanto pianificato;
	\item \textbf{soglia di ottimalità:} $SV\leq 0${} giorni rispetto a quanto pianificato, ovvero essere allineato alla pianificazione o in anticipo.
\end{itemize}
Per una descrizione più dettagliata della metrica consultare la sezione \ref{MRDP}

\paragraph{Rispetto del budget}
\label{ORDB}
Rispettare il budget stabilito nel \pdpv{} è un obiettivo importante per evitare inefficienze nell'utilizzo delle risorse.
\begin{itemize}
	\item \textbf{metrica utilizzata:} Cost Variance ($CV$);
	\item \textbf{soglia di accettabilità:} $CV\leq10\%$ rispetto a quanto preventivato;
	\item \textbf{soglia di ottimalità:} $CV\leq 0\%$ rispetto a quanto preventivato.
\end{itemize}
Per una descrizione più dettagliata della metrica consultare la sezione \ref{MRDB}

\paragraph{Completezza dell'analisi dei rischi}
\label{OCDADR}
Il \glo{Gruppo}{team} desidera che il processo di analisi dei rischi sia il più completo possibile, così da ridurre la probabilità di subire danni da rischi non preventivati.
\begin{itemize}
	\item \textbf{metrica utilizzata:} contatore Rischi Non Preventivati ($RNP$) che aumenta di 1 ogni volta che si presenta un rischio non previsto nel \pdpv; 
	\item \textbf{soglia di accettabilità:} $RNP\leq2$;
	\item \textbf{soglia di ottimalità:} $RNP=0$.
\end{itemize}
Per una descrizione più dettagliata della metrica consultare la sezione \ref{MCDADR}

\subsubsection{Qualità di prodotto}
Oltre alla qualità dei processi, il \glo{Gruppo}{team} desidera anche garantire determinate caratteristiche di qualità dei prodotti. Per raggiungere questo obiettivo, è necessario che il processo con cui tale prodotto viene realizzato sia controllato e vincolato. A tal fine, è stato scelto di seguire lo standard \glo{ISO}{ISO}/\glo{IEC}{IEC} 9126:2001.

Le tipologie di prodotti che verranno realizzati sono due:
\begin{itemize}
	\item documenti;
	\item software.
\end{itemize}
\paragraph{Qualità dei documenti}
Il \glo{Gruppo}{team} si pone come obiettivo la produzione di documenti di qualità. Essi sono infatti fondamentali per la comprensione del prodotto software fin dal concepimento, sia da parte di soggetti interni che esterni.
Seguono gli obiettivi e le metriche riguardanti la qualità dei documenti che il \glo{Gruppo}{team} si è prefissato.


\subparagraph{Leggibilità e comprensibilità}
\label{OLEC}
La leggibilità e comprensibilità dei documenti sono caratteristiche fondamentali affinché essi siano utili a coloro che li leggono.
\begin{itemize}
	\item \textbf{metrica utilizzata:} \glo{Indice Gulpease}{Indice Gulpease} ($IG$);
	\item \textbf{soglia di accettabilità:} $IG\geq 40$;
	\item \textbf{soglia di ottimalità:} $IG \geq 60$.
\end{itemize}
Per una descrizione più dettagliata della metrica consultare la sezione \ref{MLEC}.

\paragraph{Adesione alle norme interne}
\label{OAANI}
Aderire alle regole di stesura dei documenti definite nelle \ndpv{} è fondamentale per assicurare l'omogeneità del testo e della terminologia. Alcuni esempi di norme riguardanti i documenti sono quelle relative ai loro nomi (e relativa versione), agli elenchi puntati, ai ruoli dei membri, ecc.
\begin{itemize}
	\item \textbf{metrica utilizzata:} numero di Errori riguardanti le Norme interne rinvenuti e Non Corretti ($ENNC$);
	\item \textbf{soglia di accettabilità:} $ENNC=0$;
	\item \textbf{soglia di ottimalità:} $ENNC=0$.
\end{itemize}
Per una descrizione più dettagliata della metrica consultare la sezione \ref{MAANI}.

\subparagraph{Correttezza ortografica}
\label{OCO}
Il \glo{Gruppo}{team} desidera che i documenti prodotti siano completamente esenti da errori ortografici rilevati e non corretti.
\begin{itemize}
	\item \textbf{metrica utilizzata:} numero di Errori Ortografici rinvenuti e Non Corretti ($EONC$);
	\item \textbf{soglia di accettabilità:} $EONC=0$;
	\item \textbf{soglia di ottimalità:} $EONC=0$.
\end{itemize}
Per una descrizione più dettagliata della metrica consultare la sezione \ref{MCO}.

\subparagraph{Correttezza concettuale}
\label{OCC}
L'obiettivo è ridurre il più possibile il numero di errori concettuali rinvenuti e non corretti.
\begin{itemize}
	\item \textbf{metrica utilizzata:} percentuale di Errori Concettuali rinvenuti e Non Corretti ($ECNC$);
	\item \textbf{soglia di accettabilità:} $ECNC\leq 5\%$;
	\item \textbf{soglia di ottimalità:} $ECNC=0\%$.
\end{itemize}
Per una descrizione più dettagliata della metrica consultare la sezione \ref{MCC}.

\paragraph{Qualità del software}
Il \glo{Gruppo}{team} desidera che il software prodotto sia di qualità.
Seguono le caratteristiche e le metriche riguardanti la qualità del software che il \glo{Gruppo}{team} si è prefissato.

\subparagraph{Implementazione delle funzionalità obbligatorie}
\label{OIDFO}
Il software deve implementare completamente le funzionalità descritte nei requisiti obbligatori.
\begin{itemize}
	\item \textbf{metrica utilizzata:} numero di requisiti obbligatori soddisfatti ($IFO$);
	\item \textbf{soglia di accettabilità:} 100\% dei requisiti obbligatori soddisfatti;
	\item \textbf{soglia di ottimalità:} 100\% dei requisiti obbligatori soddisfatti.
\end{itemize}
Per una descrizione più dettagliata della metrica consultare la sezione \ref{MIDFO}.

\subparagraph{Implementazione delle funzionalità desiderabili}
\label{OIDFD}
Il software deve implementare completamente le funzionalità descritte nei requisiti desiderabili.
\begin{itemize}
	\item \textbf{metrica utilizzata:} numero di requisiti desiderabili soddisfatti;
	\item \textbf{soglia di accettabilità:} 100\% dei requisiti desiderabili soddisfatti;
	\item \textbf{soglia di ottimalità:} 100\% dei requisiti desiderabili soddisfatti.
\end{itemize}
Per una descrizione più dettagliata della metrica consultare la sezione \ref{MIDFD}.

\medskip

\subparagraph{Manutenibilità e comprensibilità del codice}
\label{OMECDC}
L'obiettivo è fornire codice privo di incomprensioni e manutenibile nel tempo. Per cercare di soddisfare questo obiettivo, sono state selezionate varie metriche.
\begin{itemize}
	\item \textbf{metrica utilizzata:} Numero di \glo{Statement}{Statement} per Metodo ($NSM$);
	\item \textbf{soglia di accettabilità:} $30<NSM\leq60$;
	\item \textbf{soglia di ottimalità:} $NSM\leq30$.
\end{itemize}

\medskip

\begin{itemize}
	\item \textbf{metrica utilizzata:} Numero di Parametri per Metodo ($NPM$);
	\item \textbf{soglia di accettabilità:} $5<NPM \leq 10$;
	\item \textbf{soglia di ottimalità:} $NPM\leq5$.
\end{itemize}	

\medskip

\begin{itemize}
	\item \textbf{metrica utilizzata:} Numero di Campi Dati Per Classe ($NCDPC$);
	\item \textbf{soglia di accettabilità:} $10<NCDPC\leq 15$;
	\item \textbf{soglia di ottimalità:} $NCDPC\leq10$.
\end{itemize}	

\medskip

\begin{itemize}
	\item \textbf{metrica utilizzata:} Grado di Accoppiamento ($GA$);
	\item \textbf{soglia di accettabilità:} $3<GA \leq 10$;
	\item \textbf{soglia di ottimalità:} $GA\leq3$.
\end{itemize}	
$GA$=numero di dipendenze tra classi in un \glo{Package}{package}.

\bigskip

\begin{itemize}
	\item \textbf{metrica utilizzata:} Numero Ciclomatico ($NC$);
	\item \textbf{soglia di accettabilità:} $10<NC\leq20$;
	\item \textbf{soglia di ottimalità:} $NC\leq10$.
\end{itemize}	
$NC$=numero di cammini linearmente indipendenti presenti all'interno del codice.

\bigskip

\begin{itemize}
	\item \textbf{metrica utilizzata:} Numero di Variabili dichiarate e Non Utilizzate ($NVNU$);
	\item \textbf{soglia di accettabilità:} $NVNU=0$;
	\item \textbf{soglia di ottimalità:} $NVNU=0$.
\end{itemize}


\bigskip

Per una descrizione più dettagliata della metriche relative al codice, consultare le sezioni:
\begin{itemize}
	\item \nameref{MNDSPM};
	\item \nameref{MNDPPM};
	\item \nameref{MNDCDPC};
	\item \nameref{MGDA};
	\item \nameref{MCCIC};
	\item \nameref{MNDVDENU};
	\item \nameref{MDDC}.
\end{itemize}

\subparagraph{Documentazione del codice}
\label{ODDC}
Avere codice documentato è importante per garantire manutenibilità e comprensibilità dello stesso. Il mezzo con cui si intende raggiungere tale obiettivo è commentare il codice. Verrà posta particolare attenzione nello scrivere commenti comprensibili anche a eventuali manutentori, che potranno anche essere soggetti esterni.

\begin{itemize}
	\item \textbf{metrica utilizzata:} Rapporto linee di Commento e linee di Codice ($RCC$);
	\item \textbf{soglia di accettabilità:} $RCC\geq 10\%$;
	\item \textbf{soglia di ottimalità:} $RCC \geq 30\%$.
\end{itemize}
Per una descrizione più dettagliata della metrica consultare la sezione \ref{MDDC}.


\subparagraph{Validazione web}
\label{OVW}
Il \glo{Gruppo}{team} desidera che il codice \glo{HTML}{HTML} risulti valido secondo gli strumenti offerti dal \glo{W3C}{W3C}. Nonostante codice validato non implichi direttamente la sua qualità, è un buon punto di partenza. Inoltre, serve a limitare i casi in cui l'interpretazione del codice \glo{HTML}{HTML} viene affidata al browser, ad esempio a causa di \glo{Tag}{tag} non chiusi nell'ordine corretto.
\begin{itemize}
	\item \textbf{metrica utilizzata:} Numero di Errori di Validazione ($NEV$) rilevati dal validatore online del \glo{W3C}{W3C};
	\item \textbf{soglia di accettabilità:} $NEV\leq10$;
	\item \textbf{soglia di ottimalità:} NEV=0.
\end{itemize}
Per una descrizione più dettagliata della metrica consultare la sezione \ref{MVW}.

\subparagraph{Copertura dei test richiesti}
\label{OCDTR}
Assicurare la copertura dei test è fondamentale per poter verificare la corretta implementazione delle funzionalità previste dai requisiti. 
\begin{itemize}
	\item \textbf{metrica utilizzata:} Copertura dei Test Richiesti ($CTR$);
	\item \textbf{soglia di accettabilità:} $80\%\leq CTR<90\%$;
	\item \textbf{soglia di ottimalità:} $90\%\leq CTR \leq 100\%$.
\end{itemize}
$CTR$=percentuale di test passati.\\
Per una descrizione più dettagliata della metrica consultare la sezione \ref{MCDTR}.

\subparagraph{Robustezza}
\label{OR}
Il prodotto non deve interrompere il suo funzionamento al verificarsi di situazioni anomale e di errore. È preferibile la segnalazione dell'errore all'arresto improvviso.
\begin{itemize}
	\item \textbf{metrica utilizzata:} Breakdown Avoidance ($BA$);
	\item \textbf{soglia di accettabilità:} $80\%\leq BA<95\%$;
	\item \textbf{soglia di ottimalità:} $BA\geq95\%$.
\end{itemize}
Per una descrizione più dettagliata della metrica consultare la sezione \ref{MR}.


\subparagraph{Correzione delle situazioni di fallimento}
\label{OCDSDF}
Il prodotto deve superare la maggior parte dei test che provino a compromettere la sua stabilità.
\begin{itemize}
	\item \textbf{metrica utilizzata:} Failure Avoidance ($FA$);
	\item \textbf{soglia di accettabilità:} $80\% \leq FA<95\%$;
	\item \textbf{soglia di ottimalità:} $FA\geq95\%$.
\end{itemize}
$FA$=percentuale di situazioni anomale evitate su situazioni anomale prese in considerazione.\\
Per una descrizione più dettagliata della metrica consultare la sezione \ref{MCDSDF}.


\subsubsection{Scadenze temporali}
Le scadenze che il \glo{Gruppo}{team} ha deciso di rispettare sono riportate nel \pdpv.

%FINITA IL 5/12





