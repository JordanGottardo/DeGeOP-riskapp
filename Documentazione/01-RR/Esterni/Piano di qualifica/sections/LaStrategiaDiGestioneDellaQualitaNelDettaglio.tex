\newpage



\section{La strategia di gestione della qualità nel dettaglio}
		\subsection{Risorse}
		Il processo di verifica necessita di risorse per poter ottenere gli obiettivi prefissati. Il \glo{Gruppo}{team} ha identificato i seguenti tipi di risorse:
		\begin{itemize}
			\item risorse umane;
			\item risorse hardware;
			\item risorse software.
		\end{itemize}
			\subsubsection{Risorse necessarie}
				\paragraph{Risorse umane}
				Le risorse umane comprendono il \responsabilediprogetto{} e i \verificatori.
				\paragraph{Risorse hardware}
				Le risorse hardware comprendono i computer con una potenza di calcolo sufficiente a sopportare il carico di lavoro.
				\paragraph{Risorse software}
				Le risorse software comprendono gli strumenti software, sia installabili localmente che disponibili online, che permettano di eseguire controlli su documenti e codice.
				Inoltre è necessario anche un tool per il tracciamento dei requisiti e il controllo dei test.
			\subsubsection{Risorse disponibili}
				\paragraph{Risorse umane}
				Le risorse umane disponibili sono tutti i membri del \glo{Gruppo}{team}, che ricopriranno a turno i ruoli di \responsabilediprogetto{} e di \verificatore{} come previsto dal \pdpv.
				\paragraph{Risorse hardware}
				Le risorse hardware disponibili sono i personal computer dei membri del \glo{Gruppo}{team}. In caso tali non fossero disponibili, si potranno sfruttare i computer del Servizio Calcolo dell'Università di Padova.
				\paragraph{Risorse software}
				Le risorse software disponibili sono:
				\begin{itemize}
					\item editor \LaTeX{} con pacchetto lingua italiana installato e correzioni ortografiche abilitate;
					\item script per calcolare l'\glo{Indice Gulpease}{indice Gulpease};
					\item \glo{Issue}{issue} offerte da \glo{GitHub}{GitHub}.
				\end{itemize}
		\subsection{Misure e metriche}
			\subsubsection{Misure}
			Ogni volta che viene effettuata una misurazione si ottiene un valore, ovvero una misura. Per poter comparare due valori è necessario rapportarli su una scala: facendo ciò si crea una metrica. \\
			Tenendo conto degli obiettivi precedentemente stabiliti, il valore può avere giudizio:
			\begin{itemize}
				\item \textbf{negativo:} obiettivo non raggiunto. È necessario fare ulteriori verifiche o correzioni. Per alcune metriche, le azioni da intraprendere in caso di valori negativi sono descritto nelle \ndpv;
				\item \textbf{accettabile:} obiettivo raggiunto, soglia di accettabilità superata;
				 Se il criterio a cui il valore si riferisce è un obiettivo importante in ottica \glo{PDCA}{PDCA}, è necessario attuare azioni per migliorare la qualità;
				\item \textbf{ottimale:} obiettivo raggiunto, soglia di ottimalità superata. Anche in questo caso vale quanto detto nel punto precedente riguardo il miglioramento continuo.
			\end{itemize}
			\subsubsection{Metriche per i processi}
				\paragraph{Miglioramento costante}
				\label{MMC}
					Per misurare l'efficacia del miglioramento costante, il \glo{Gruppo}{team} ha deciso di utilizzare la metrica fornita dal modello \glo{CMM}{CMM}. Per ogni \glo{Fase}{fase}, verrà misurata la qualità dei processi. La scala assume valori da 1 (peggiore) a 5 (migliore).
					\begin{itemize}
						\item \textbf{metrica utilizzata:} Livello \glo{CMM}{CMM} ($LCMM$);
						\item \textbf{valore negativo:} 1;
						\item \textbf{valore accettabile:} 2 e 3;
						\item \textbf{valore ottimale:} 4 e 5.
					\end{itemize}
				Questa metrica fa riferimento all'obiettivo riportato alla sezione \ref{OMC}.
				
				\paragraph{Rispetto della pianificazione}
				\label{MRDP}
				La metrica utilizzata è la Schedule Variance. È implementata come differenza 
				tra la pianificazione dei costi del lavoro eseguito e del lavoro pianificato. Entrambi questi valori sono intesi nella loro accezione temporale (giorni) e non monetaria.
				\begin{itemize}
					\item \textbf{metrica utilizzata:} $$SV = BCWP-BCWS$$
					dove $BCWP$=Budgeted Cost of Work Performed e $BCWS$=Budgeted Cost of Work Scheduled;
					\item \textbf{valore negativo:} $SV>5$ giorni;
					\item \textbf{valore accettabile:} $0<SV\leq 4$ giorni;
					\item \textbf{valore ottimale:} $SV\leq0$ giorni.
				\end{itemize}
				Questa metrica fa riferimento all'obiettivo riportato alla sezione \ref{ORDP}.
			
				\paragraph{Rispetto del budget}
				\label{MRDB}
				La metrica utilizzata è la Cost Variance. È implementata come differenza tra costo pianificato e costo effettivo del lavoro eseguito.
				\begin{itemize}
					\item \textbf{metrica utilizzata:} $$CV = BCWP-ACWP$$
					dove $BCWP$=Budgeted Cost of Work Performed e $ACWP$=Actual Cost of Work Performed;
					\item \textbf{valore negativo:} $CV>10\%$;
					\item \textbf{valore accettabile:} $0\%<CV\leq 10\%$;
					\item \textbf{valore ottimale:} $CV\leq 0\%$.
				\end{itemize}
				Questa metrica fa riferimento all'obiettivo riportato alla sezione \ref{ORDB}.
			
			\paragraph{Completezza dell'analisi dei rischi}
			\label{MCDADR}
			\begin{itemize}
				\item \textbf{metrica utilizzata:} contatore $RNP$ (Rischi Non Preventivati) che aumenta di 1 ogni volta che si presenta un rischio non preventivato nel \pdpv. Il contatore non si resetta al cambiamento di  \glo{Fase}{fase}: è globale di progetto;
				\item \textbf{valore negativo:} $RNP>2$;
				\item \textbf{valore accettabile:} $1\leq RNP\leq 2$;
				\item \textbf{valore ottimale:} $RNP=0$.
			\end{itemize}
			Questa metrica fa riferimento all'obiettivo riportato alla sezione \ref{OCDADR}.
			
		\subsubsection{Metriche per i prodotti}
		In questa sezione sono spiegate nel dettaglio le metriche scelte per la valutazione dei prodotti.
			
			\paragraph{Metriche per i documenti}
			Le metriche che il \glo{Gruppo}{team} ha scelto per i documenti sono il più oggettive possibili. Da sole non garantiscono la qualità generale del documento, quindi è necessario un'ulteriore e accurata verifica, soprattutto per evitare gli errori concettuali e di forma.
			
			
				\subparagraph{Leggibilità e comprensibilità}
				\label{MLEC}
				La metrica utilizzata è l'\glo{Indice Gulpease}{Indice Gulpease} ($IG$), un indice di leggibilità di un testo tarato sulla lingua italiana. La scala va da 0 a 100, dove "0" indica un documento di bassa leggibilità e "100" uno di alta. Risulta che i testi con indice:
				\begin{itemize}
					\item inferiore a 80 sono difficili da leggere per chi ha la licenza elementare;
					\item inferiore a 60 sono difficili da leggere per chi ha la licenza media;
					\item inferiore a 40 sono difficili da leggere per chi ha un diploma superiore.
				\end{itemize}
				Il \glo{Gruppo}{team} desidera che i testi siano comprensibili da persone con almeno un diploma superiore.
				\begin{itemize}
					\item \textbf{metrica utilizzata:} $$IG=89 + \frac{300\cdot{}NF-10\cdot{}NL}{NP}$$
					dove $NF$ è il numero di frasi, $NL$ è il numero di lettere e $NP$ è il numero di parole presenti nel testo.
					\item \textbf{valore negativo:} $IG<40$;
					\item \textbf{valore accettabile:} $40\leq IG <60$;
					\item \textbf{valore ottimale:} $IG\geq 60$.
				\end{itemize}
				Questa metrica fa riferimento all'obiettivo riportato alla sezione \ref{OLEC}.
				
				\subparagraph{Adesione alle norme interne}
				\label{MAANI}
				La metrica utilizzata è il numero di Errori riguardanti le Norme interne rinvenuti e Non Corretti ($ENNC$). È implementata con un contatore che aumenta di 1 ogni volta che un errore riguardante le norme interne rilevato da un \verificatore{} non viene corretto. Il contatore fa riferimento ad una specifica \glo{Fase}{fase} e viene azzerato successivamente.
				\begin{itemize}
					\item \textbf{metrica utilizzata:} $ENNC$;
					\item \textbf{valore negativo:} $ENNC>0$;
					\item \textbf{valore accettabile:} $ENNC=0$;
					\item \textbf{valore ottimale:} $ENNC=0$.
				\end{itemize}
				Questa metrica fa riferimento all'obiettivo riportato alla sezione \ref{OAANI}.
				
				\subparagraph{Correttezza ortografica}
				\label{MCO}
				La metrica utilizzata è un contatore di Errori Ortografici Non Corretti ($EONC$), che aumenta di 1 ogni volta che un errore ortografico rilevato da un \verificatore{} non viene corretto. Il contatore fa riferimento ad una specifica \glo{Fase}{fase} e viene azzerato successivamente.
				\begin{itemize}
					\item \textbf{metrica utilizzata:} $EONC$;
					\item \textbf{valore negativo:} $EONC>0$;
					\item \textbf{valore accettabile:} $EONC=0$;
					\item \textbf{valore ottimale:} $EONC=0$.
				\end{itemize}
				Questa metrica fa riferimento all'obiettivo riportato alla sezione \ref{OCO}.		
						
				\subparagraph{Correttezza concettuale}
				\label{MCC}
				La metrica utilizzata è chiamata Errori Concettuali Non Corretti ($ECNC$). La formula è data dal complemento a 1 del rapporto tra errori concettuali corretti e rilevati. Tali errori possono essere rilevati dai Verificatori o dal committente.
				\begin{itemize}
					\item \textbf{metrica utilizzata:} $$ECNC=\left(1-\frac{ECC}{ECR}\right)\cdot100$$ dove $ECC$=Errori Concettuali Corretti e $ECR$=Errori Concettuali Rinvenuti;
					\item \textbf{valore negativo:} $ECNC>5\%$;
					\item \textbf{valore accettabile:} $ECNC\leq5\%$;
					\item \textbf{valore ottimale:} $ECNC=0\%$.
				\end{itemize}
				Questa metrica fa riferimento all'obiettivo riportato alla sezione \ref{OCC}.
				
			\paragraph{Metriche per il software}
				Le metriche che il \glo{Gruppo}{team} ha scelto per misurare la qualità del software si possono riferire a:
				\begin{itemize}
					\item qualità interna;
					\item qualità esterna;
					\item qualità in uso.
				\end{itemize}
				Come descritto nello standard \glo{ISO}{ISO}/\glo{IEC}{IEC} 9126:2001, ogni metrica corrisponde a determinate caratteristiche di qualità. Segue una tabella che associa ogni metrica scelta alla relativa caratteristica. Inoltre è specificato a quale tipologia di qualità essa fa riferimento.

				\begin{table}[H]
					\centering
					\small
					\begin{tabular}{l | c| c| c}
						\hline
						%						\toprule
						\textbf{Obiettivo}                         & \textbf{Metrica} & \textbf{Tipo} & \textbf{Caratteristica} \\ \hline
						%						\midrule
						\hyperref[OIDFO]{Implementazione  funzionalità obbligatorie} 						& \hyperref[MIDFO]{IFO}              & Esterna       & Funzionalità            \\
						\hyperref[OIDFD]{Implementazione funzionalità desiderabili}                         & \hyperref[MIDFD]{IFD}              & Esterna       & Funzionalità            \\
						\hyperref[OMECDC]{Numero di statement per metodo}                                   & \hyperref[MNDSPM]{NSM}              & Interna       & Manutenibilità          \\
						\hyperref[OMECDC]{Numero di parametri per metodo}                                    & \hyperref[MNDPPM]{NPM}              & Interna       & Manutenibilità          \\
						\hyperref[OMECDC]{Numero di campi dati per classe}                                   	& \hyperref[MNDCDPC]{NCD}              & Interna       & Manutenibilità          \\
						\hyperref[OMECDC]{Grado di accoppiamento}                                            	   & \hyperref[MGDA]{GA}               & Interna       & Manutenibilità          \\
						\hyperref[OMECDC]{Complessità ciclomatica}                                                   & \hyperref[MCCIC]{NC}               & Interna       & Manutenibilità          \\
						\hyperref[OMECDC]{Variabili dichiarate e non utilizzate}                                   & \hyperref[MNDVDENU]{NVNU}             & Interna       & Manutenibilità          \\
						\hyperref[ODDC]{Documentazione del codice}                                             & \hyperref[MDDC]{LCC}              & Interna       & Manutenibilità          \\
						\hyperref[OVW]{Validazione web}                                                              & \hyperref[MVW]{NEV}              & Interna       & Manutenibilità          \\
						\hyperref[OCDTR]{Copertura dei test richiesti}                                              & \hyperref[MCDTR]{CTR}              & Interna       & Affidabilità            \\
						\hyperref[OR]{Robustezza}                                                                     & \hyperref[MR]{BA}               & Esterna       & Affidabilità            \\
						\hyperref[OCDSDF]{Correzione delle situazioni di fallimento}                           & \hyperref[MCDSDF]{FA}               & Esterna       & Affidabilità            \\ \hline
						%						\midrule                                                  &
					\end{tabular}
					\caption{Mappa Metriche-Caratteristiche}
					\label{tab:mappa_metriche_caratteristiche}
				\end{table}
				
				\subparagraph{Implementazione delle funzionalità obbligatorie}
				\label{MIDFO}
				La metrica utilizzata è chiamata Implementazione delle Funzionalità Obbligatorie ($IFO$). Consiste in un rapporto tra il numero di requisiti obbligatori soddisfatti e identificati
				\begin{itemize}
					\item \textbf{metrica utilizzata:} $$IFO= \frac{ROS}{ROI}\cdot 100$$
					dove $ROS$=numero di Requisiti Obbligatori Soddisfatti e $ROI$=numero di Requisiti Obbligatori Identificati;
					\item \textbf{valore negativo:} $IFO<100\%$;
					\item \textbf{valore accettabile:} $IFO=100\%$;
					\item \textbf{valore ottimale:} $IFO=100\%$.
				\end{itemize}
				Questa metrica fa riferimento all'obiettivo riportato alla sezione \ref{OIDFO}.
				
				\subparagraph{Implementazione delle funzionalità desiderabili}
				\label{MIDFD}
				La metrica utilizzata è chiamata Implementazione delle Funzionalità Desiderabili ($IFD$). Consiste in un rapporto tra il numero di requisiti desiderabili soddisfatti e identificati
				\begin{itemize}
					\item \textbf{metrica utilizzata:} $$IFD= \frac{RDS}{RDI}\cdot 100$$
					dove $RDS$=numero di Requisiti Desiderabili Soddisfatti e $RDI$=numero di Requisiti Desiderabili Identificati;
					\item \textbf{valore negativo:} $IFD<100\%$;
					\item \textbf{valore accettabile:} $IFD=100\%$;
					\item \textbf{valore ottimale:} $IFD=100\%$.
				\end{itemize}
				Questa metrica fa riferimento all'obiettivo riportato alla sezione \ref{OIDFD}.\\
				
				Le seguenti metriche fanno riferimento all'obiettivo di \nameref{OMECDC} riportato alla sezione \ref{OMECDC}.
			
				\subparagraph{Numero di statement per metodo}
				\label{MNDSPM}
				\begin{itemize}
					\item \textbf{metrica utilizzata:} Numero di \glo{Statement}{Statement} per Metodo ($NSM$);
					\item \textbf{valore negativo:} $NSM>60$;
					\item \textbf{valore accettabile:} $30< NSM\leq 60$;
					\item \textbf{valore ottimale:} $NSM\leq 30$.
				\end{itemize}
			
				\subparagraph{Numero di parametri per metodo}
				\label{MNDPPM}
				\begin{itemize}
					\item \textbf{metrica utilizzata:} Numero di Parametri per Metodo ($NPM$);
					\item \textbf{valore negativo:} $NPM>12$;
					\item \textbf{valore accettabile:} $5<NPM \leq 12$;
					\item \textbf{valore ottimale:} $NPM\leq5$.
				\end{itemize}
	
				\subparagraph{Numero di campi dati per classe}
				\label{MNDCDPC}
				\begin{itemize}
					\item \textbf{metrica utilizzata:} Numero Campi Dati Per Classe ($NCDPC$);
					\item \textbf{valore negativo:} $NCDPC>15$;
					\item \textbf{valore accettabile:} $10<NCDPC \leq 15$;
					\item \textbf{valore ottimale:} $NCDPC\leq10$.
				\end{itemize}
			
				\subparagraph{Grado di accoppiamento}
				\label{MGDA}
				Il grado di accoppiamento ($GA$) è calcolato in base al numero di dipendenze tra classi in un \glo{Package}{package}.
				\begin{itemize}
					\item \textbf{metrica utilizzata:} Grado di Accoppiamento ($GA$);
					\item \textbf{valore negativo:} $GA > 10$;
					\item \textbf{valore accettabile:} $3<GA \leq 10$;
					\item \textbf{valore ottimale:} $GA\leq3$.
				\end{itemize}
			
				\subparagraph{Complessità ciclomatica}
				\label{MCCIC}
				La metrica è basata sul numero ciclomatico ($NC$), che rappresenta il numero di cammini linearmente indipendenti presenti all'interno del codice.
				
				\begin{itemize}
					\item \textbf{metrica utilizzata:} $$NC=e-n+2p$$ 
					con $e$=numero di archi, $n$=numero di nodi, $p$=numero di componenti connesse;
					\item \textbf{valore negativo:} $NC>20$;
					\item \textbf{valore accettabile:} $10<NC \leq 20$;
					\item \textbf{valore ottimale:} $NC\leq10$.
				\end{itemize}
			
				\subparagraph{Numero di variabili dichiarate e non utilizzate}
				\label{MNDVDENU}
				\begin{itemize}
					\item \textbf{metrica utilizzata:} Numero di Variabili dichiarate e Non Utilizzate ($NVNU$);
					\item \textbf{valore negativo:} $NVNU>0$;
					\item \textbf{valore accettabile:} $NVNU=0$;
					\item \textbf{valore ottimale:} $NVNU=0$.
				\end{itemize}
			
			
				\subparagraph{Documentazione del codice}
				\label{MDDC}
				La metrica è implementata come Rapporto tra le linee di Commento e le linee di Codice ($RCC$).
				\begin{itemize}
					\item \textbf{metrica utilizzata:} $$RCC=\left(\frac{LDCM}{LDCC}\right)\cdot 100$$
					dove $LDCC$=Linee Di Codice e $LDCM$=Linee Di Commento.
					\item \textbf{valore negativo:} $RCC<10\%$;
					\item \textbf{valore accettabile:} $RCC\geq 10\%$;
					\item \textbf{valore ottimale:} $RCC\geq 30\%$.
				\end{itemize}
				Questa metrica fa riferimento all'obiettivo riportato alla sezione \ref{ODDC}.
			
			
				\subparagraph{Validazione web}
				\label{MVW}
				La metrica è basata sul Numero di Errori di Validazione ($NEV$) rilevati dal validatore online del \glo{W3C}{W3C}. Eventuali errori causati dall'utilizzo di librerie esterne non verranno presi in considerazione.
				\begin{itemize}
					\item \textbf{metrica utilizzata:} $NEV$;
					\item \textbf{valore negativo:} $NEV>10$;
					\item \textbf{valore accettabile:} $0<NEV\leq 10$;
					\item \textbf{valore ottimale:} $NEV=0$.
				\end{itemize}
				Questa metrica fa riferimento all'obiettivo riportato alla sezione \ref{OVW}.
				
				\subparagraph{Copertura dei test richiesti}
				\label{MCDTR}
				La metrica utilizzata è la copertura dei test richiesti ($CTR$). I test presi in considerazione sono quelli necessari a verificare l'implementazione delle funzionalità previste dai requisiti.
				\begin{itemize}
					\item \textbf{metrica utilizzata:} $$CTR=\frac{TS}{TR}\cdot100$$
					dove $TS$=numero di Test Superati e $TR$=numero di Test Richiesti;
					\item \textbf{valore negativo:} $CTR<80\%$;
					\item \textbf{valore accettabile:} $80\leq CTR<90\%$;
					\item \textbf{valore ottimale:} $90\% \leq CTR \leq 100\%$.
				\end{itemize}
				Questa metrica fa riferimento all'obiettivo riportato alla sezione \ref{OCDTR}.
			
				\subparagraph{Robustezza}
				\label{MR}
				La metrica utilizzata è la Breakdown Avoidance ($BA$). 
				\begin{itemize}
					\item \textbf{metrica utilizzata:} $$BA=\left( 1-\frac{NI}{NSA} \right)\cdot 100$$ 
					dove $NI$=Numero di Interruzioni e $NSA$=Numero di Situazioni Anomale;
					\item \textbf{valore negativo:} $BA< 80\%$;
					\item \textbf{valore accettabile:} $80 \leq BA<95\%$;
					\item \textbf{valore ottimale:} $BA\geq 95\%$.
				\end{itemize}
				Questa metrica fa riferimento all'obiettivo riportato alla sezione \ref{OR}.
				
				\subparagraph{Correzione delle situazioni di fallimento}
				\label{MCDSDF}
				La metrica utilizzata è la Failure Avoidance ($FA$). 
				\begin{itemize}
					\item \textbf{metrica utilizzata:} $$FA=\left(\frac{SAE}{SAT}\right) \cdot 100$$
					con $SAE$=Situazioni Anomale Evitate e $SAT$=Situazioni Anomale Testate;
					\item \textbf{valore negativo:} $FA< 80\%$;
					\item \textbf{valore accettabile:} $80\leq FA<95\%$;
					\item \textbf{valore ottimale:} $FA\geq 95\%$.
				\end{itemize}
				Questa metrica fa riferimento all'obiettivo riportato alla sezione \ref{OCDSDF}.
			
% FINITA IL 12/12