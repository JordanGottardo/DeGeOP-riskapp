\newpage

\section{Test}
	I test, eseguiti tramite analisi dinamica, sono attività che servono a verificare che il software prodotto implementi le funzionalità richieste. Una caratteristica fondamentale dei test è la ripetibilità: i risultati che essi forniscono devono essere deterministici, in modo da eseguire azioni correttive in caso gli esiti non siano quelli attesi.
	%Da modificare in futuro
	Per tracciare i test eseguiti e i risultati ottenuti sarà necessario produrre dei log di facile consultazione.
	
	I test verranno specificati quando sarà cominciata la \glo{Fase}{fase} di progettazione.
	
	\subsection{Test di accettazione}
		I test di accettazione, eseguiti durante il collaudo finale, servono a verificare che il software soddisfi le richieste del proponente.
		
	\subsection{Test di sistema}
		I test di sistema servono a verificare il corretto funzionamento delle componenti dell'intero sistema.
		
	\subsection{Test di integrazione}
		I test di integrazione servono a verificare il corretto funzionamento di più unità. Più precisamente, l'obiettivo è quello di testare i vari \glo{Package}{package}, sia singolarmente che nel loro insieme.
		
	\subsection{Test di unità}
		I test di unità servono a verificare il corretto funzionamento della singola unità, ovvero della più piccola parte di lavoro realizzabile dal singolo programmatore.