\newpage

\section{Resoconto delle attività di verifica -Fase Analisi}
	\subsection{Verifica dei processi}
		
		Segue una tabella riassuntiva riguardante le metriche di processo. Le righe che riportano una serie di trattini orizzontali sotto la voce "Processo" sono relative all'intera \glo{Fase}{fase} e non al singolo processo.
		Per avere informazioni dettagliate sugli scopi dei processi e sulle attività che li compongono, consultare le \ndpv.
		
		
		Per avere un resoconto testuale degli obiettivi in tabella, fare click sul nome dell'obiettivo.
		
		
		Per una descrizione delle metriche in tabella, fare click sul nome della metrica.
		\begin{table}[H]
			\centering
			\small
			\begin{tabular}{c | c | c | c | c}
				\hline
				          \textbf{Processo}            & \textbf{Obiettivo}        & \textbf{Metrica}                & \textbf{Valore}                            & \textbf{Giudizio}                           \\ \hline
				          Fornitura           & \nameref{RMC}    & \hyperref[MMC]{LCMM}   & \textcolor{LimeGreen}{2}          & \textcolor{LimeGreen}{Accettabile} \\
				          Sviluppo            & \nameref{RMC}    & \hyperref[MMC]{LCMM}   & \textcolor{LimeGreen}{2}          & \textcolor{LimeGreen}{Accettabile} \\
				       Documentazione         & \nameref{RMC}    & \hyperref[MMC]{LCMM}   & \textcolor{LimeGreen}{2}          & \textcolor{LimeGreen}{Accettabile} \\
						   Verifica            & \nameref{RMC}    & \hyperref[MMC]{LCMM}   & \textcolor{LimeGreen}{2}          & \textcolor{LimeGreen}{Accettabile} \\
				Gestione delle infrastrutture & \nameref{RMC}    & \hyperref[MMC]{LCMM}   & \textcolor{LimeGreen}{2}          & \textcolor{LimeGreen}{Accettabile} \\
				    Gestione dei processi     & \nameref{RMC}    & \hyperref[MMC]{LCMM}   & \textcolor{LimeGreen}{2}          & \textcolor{LimeGreen}{Accettabile} \\
				        Apprendimento         & \nameref{RMC}    & \hyperref[MMC]{LCMM}   & \textcolor{LimeGreen}{2}          & \textcolor{LimeGreen}{Accettabile} \\
				        - - - - - - -         & \nameref{RRDP}   & \hyperref[MRDP]{SV}    & \textcolor{LimeGreen}{3 giorni}   & \textcolor{LimeGreen}{Accettabile} \\
				        - - - - - - -         & \nameref{RRDB}   & \hyperref[MRDB]{CV}    & \textcolor{ForestGreen}{0\%}      & \textcolor{ForestGreen}{Ottimale}  \\
				        - - - - - - -         & \nameref{RCDADR} & \hyperref[MCDADR]{RNP} & \textcolor{ForestGreen}{0 rischi} & \textcolor{ForestGreen}{Ottimale}  \\ \hline
			\end{tabular}
			\caption{Resoconto metriche di processo}
			\label{tab:resoconto_metriche_processo}
		\end{table}
	
		\subsubsection{Considerazioni finali}
			\paragraph{Miglioramento costante}
			\label{RMC}
				Il livello \glo{CMM}{CMM} dei processi di Fornitura, Sviluppo, Documentazione, Verifica, Gestione delle infrastrutture, Gestione di Processo e Apprendimento in questa \glo{Fase}{fase} è pari a 2. Dopo lo stato iniziale, durato quasi fino a metà della \glo{Fase}{fase}, in cui i processi si trovavano in uno stato caotico, il rispetto delle \ndpv{} e l'adozione di strumenti automatici ha portato ad un guadagno di ripetibilità. Alcuni esempi di tali strumenti sono i correttori ortografici e lo script per il calcolo dell'indice di leggibilità per quanto riguarda i processi di Documentazione e Verifica e l'utilizzo di \glo{Trender}{Trender} per quanto riguarda l'attività di analisi dei requisiti del processo di Sviluppo.
				
				
				Tutti i processi non sono standardizzati ad un livello tale da raggiungere il livello 3 della scala. Inoltre, la disciplina non è ancora molto rigorosa. L'obiettivo per le prossime \glo{Fase}{fasi} è migliorare tale livello.
				
			\paragraph{Rispetto della pianificazione}
			\label{RRDP}
				Il ritardo riscontrato nella \glo{Fase}{fase} di Analisi è pari a 3 giorni. Dato che il ritardo è all'interno della soglia di accettabilità, il \glo{Gruppo}{team} è ancora in grado di rispettare la scadenza. L'obiettivo è cercare di evitare ritardi nelle \glo{Fase}{fasi} successive.
			
			\paragraph{Rispetto del budget}
			\label{RRDB}
				Non sono state riscontrate spese aggiuntive. La metrica assume quindi un valore ottimale.
			
			\paragraph{Completezza dell'analisi dei rischi}
			\label{RCDADR}
				Dall'inizio del progetto non sono sorti rischi non preventivati, pertanto la metrica assume un valore ottimale.
			
			\subsection{Verifica dei prodotti}
				\subsubsection{Verifica dei documenti}
					I documenti sono stati analizzati principalmente tramite \glo{Walkthrough}{walkthrough} data la scarsa esperienza dei verificatori. Gli errori più ricorrenti sono stati annotati e serviranno a creare una lista per le successiva attività di verifica, da effettuare utilizzando \glo{Inspection}{inspection}.		
					
					Seguono tabelle riassuntive riguardante le metriche relative ai documenti. 
					
					Per una descrizione delle metriche in tabella, fare click sul nome della metrica.
					
					\paragraph{Leggibilità e comprensibilità}
						\begin{table}[H]
							\centering
							\small
							\begin{tabular}{c | c | c | c}
								\hline
								\textbf{Documento} & \textbf{Metrica}    & \textbf{Valore} & \textbf{Giudizio} \\ \hline
								      \pdpv        & \hyperref[MLEC]{IG} &  \textcolor{LimeGreen}{58}              & \textcolor{LimeGreen}{Accettabile} \\
								      \pdqv        & \hyperref[MLEC]{IG} &  \textcolor{LimeGreen}{58}               &  \textcolor{LimeGreen}{Accettabile} \\
								      \ndpv        & \hyperref[MLEC]{IG} &  \textcolor{ForestGreen}{61}               & \textcolor{ForestGreen}{Ottimale}\\
								      \sdfv        & \hyperref[MLEC]{IG} &  \textcolor{LimeGreen}{52}               &  \textcolor{LimeGreen}{Accettabile}\\
								      \adrv        & \hyperref[MLEC]{IG} &  \textcolor{LimeGreen}{45}               &  \textcolor{LimeGreen}{Accettabile}\\
								       \glv        & \hyperref[MLEC]{IG} &  \textcolor{LimeGreen}{56}               & \textcolor{LimeGreen}{Accettabile} \\
								      \vunoi       & \hyperref[MLEC]{IG} &  \textcolor{ForestGreen}{79}               & \textcolor{ForestGreen}{Ottimale}\\
								      \vduei       & \hyperref[MLEC]{IG} &  \textcolor{ForestGreen}{79}               & \textcolor{ForestGreen}{Ottimale}\\
								      \vtrei       & \hyperref[MLEC]{IG} &  \textcolor{ForestGreen}{79}               & \textcolor{ForestGreen}{Ottimale}\\
								    \vquattroi     & \hyperref[MLEC]{IG} &  \textcolor{ForestGreen}{79}               & \textcolor{ForestGreen}{Ottimale}\\
								      \vunoe       & \hyperref[MLEC]{IG} &   \textcolor{ForestGreen}{73}              & \textcolor{ForestGreen}{Ottimale}\\
								      \vduee       & \hyperref[MLEC]{IG} &   \textcolor{ForestGreen}{68}              & \textcolor{ForestGreen}{Ottimale}\\ \hline
							\end{tabular}
							\caption{Resoconto leggibilità e comprensibilità}
							\label{tab_resoconto_leggibilità_e_comprensibilità}
						\end{table}
					
						\subparagraph{Considerazioni finali}
						Tutti i documenti presentano un \glo{Indice Gulpease}{indice Gulpease} ad un livello almeno accettabile; ciò dovrebbe garantire una lettura non particolarmente difficoltosa da parte di soggetti con almeno licenza superiore.
						Il documento che assume il valore più basso è l'\textit{Analisi dei requisiti v1.0.0}. Questo è dovuto al fatto che esso è un documento particolarmente tecnico e i contenuti sono esposti sotto forma di tabelle.	
					\paragraph{Adesione alle norme interne}
						\begin{table}[H]
							\centering
							\small
							\begin{tabular}{c | c | c | c}
								\hline
								\textbf{Documento} & \textbf{Metrica} & \textbf{Valore} & \textbf{Giudizio} \\
								\hline
								\pdpv & \hyperref[MAANI]{ENNC} & \textcolor{ForestGreen}{0} & \textcolor{ForestGreen}{Ottimale} \\
								\pdqv & \hyperref[MAANI]{ENNC} & \textcolor{ForestGreen}{0} & \textcolor{ForestGreen}{Ottimale}\\
								\ndpv & \hyperref[MAANI]{ENNC} &\textcolor{ForestGreen}{0} & \textcolor{ForestGreen}{Ottimale}\\
								\sdfv & \hyperref[MAANI]{ENNC} & \textcolor{ForestGreen}{0} & \textcolor{ForestGreen}{Ottimale}\\
								\adrv & \hyperref[MAANI]{ENNC} & \textcolor{ForestGreen}{0} & \textcolor{ForestGreen}{Ottimale}\\
								\glv  & \hyperref[MAANI]{ENNC} & \textcolor{ForestGreen}{0} & \textcolor{ForestGreen}{Ottimale}\\
								\vunoi& \hyperref[MAANI]{ENNC} & \textcolor{ForestGreen}{0} & \textcolor{ForestGreen}{Ottimale}\\
								\vduei& \hyperref[MAANI]{ENNC} & \textcolor{ForestGreen}{0} & \textcolor{ForestGreen}{Ottimale}\\
								\vtrei & \hyperref[MAANI]{ENNC} & \textcolor{ForestGreen}{0} & \textcolor{ForestGreen}{Ottimale}\\
								\vquattroi & \hyperref[MAANI]{ENNC} & \textcolor{ForestGreen}{0} & \textcolor{ForestGreen}{Ottimale}\\
								\vunoe & \hyperref[MAANI]{ENNC} & \textcolor{ForestGreen}{0} & \textcolor{ForestGreen}{Ottimale}\\
								\vduee & \hyperref[MAANI]{ENNC} & \textcolor{ForestGreen}{0} & \textcolor{ForestGreen}{Ottimale}\\
								\hline
							\end{tabular}
							\caption{Resoconto adesione alle norme interne}
							\label{tab_resoconto_adesione_alle_norme_interne}
						\end{table}
					
						\subparagraph{Considerazioni finali}
							Per tutti i documenti non risultano errori residui che violino le norme interne, pertanto le metriche hanno un valore ottimale.
						
						
				\paragraph{Correttezza ortografica}
					\begin{table}[H]
						\centering
						\small
						\begin{tabular}{c | c | c | c}
							\hline
							\textbf{Documento} & \textbf{Metrica} & \textbf{Valore} & \textbf{Giudizio} \\
							\hline
							\pdpv & \hyperref[MCO]{EONC} & \textcolor{ForestGreen}{0} & \textcolor{ForestGreen}{Ottimale} \\
							\pdqv & \hyperref[MCO]{EONC} & \textcolor{ForestGreen}{0} & \textcolor{ForestGreen}{Ottimale}\\
							\ndpv & \hyperref[MCO]{EONC} &\textcolor{ForestGreen}{0} & \textcolor{ForestGreen}{Ottimale}\\
							\sdfv & \hyperref[MCO]{EONC} & \textcolor{ForestGreen}{0} & \textcolor{ForestGreen}{Ottimale}\\
							\adrv & \hyperref[MCO]{EONC} & \textcolor{ForestGreen}{0} & \textcolor{ForestGreen}{Ottimale}\\
							\glv  & \hyperref[MCO]{EONC} & \textcolor{ForestGreen}{0} & \textcolor{ForestGreen}{Ottimale}\\
							\vunoi& \hyperref[MAANI]{EONC} & \textcolor{ForestGreen}{0} & \textcolor{ForestGreen}{Ottimale}\\
							\vduei& \hyperref[MAANI]{EONC} & \textcolor{ForestGreen}{0} & \textcolor{ForestGreen}{Ottimale}\\
							\vtrei & \hyperref[MAANI]{EONC} & \textcolor{ForestGreen}{0} & \textcolor{ForestGreen}{Ottimale}\\
							\vquattroi & \hyperref[MAANI]{EONC} & \textcolor{ForestGreen}{0} & \textcolor{ForestGreen}{Ottimale}\\
							\vunoe & \hyperref[MAANI]{EONC} & \textcolor{ForestGreen}{0} & \textcolor{ForestGreen}{Ottimale}\\
							\vduee & \hyperref[MAANI]{EONC} & \textcolor{ForestGreen}{0} & \textcolor{ForestGreen}{Ottimale}\\
							\hline
						\end{tabular}
						\caption{Resoconto correttezza ortografica}
						\label{tab_resoconto_correttezza_ortografica}
					\end{table}
					
					\subparagraph{Considerazioni finali}
					Dopo l'analisi automatica dei correttori ortografici e quella mediante \glo{Walkthrough}{walkthrough} da parte dei \verificatori{} non sono stati rilevati ulteriori errori che violano le norme interne, pertanto le metriche assumono un valore ottimale.
					
				\paragraph{Correttezza concettuale}
				\begin{table}[H]
					\centering
					\small
					\begin{tabular}{c | c | c | c}
						\hline
						\textbf{Documento} & \textbf{Metrica} & \textbf{Valore} & \textbf{Giudizio} \\
						\hline
						\pdpv & \hyperref[MCC]{ECNC} & \textcolor{ForestGreen}{0} & \textcolor{ForestGreen}{Ottimale} \\
						\pdqv & \hyperref[MCC]{ECNC} & \textcolor{ForestGreen}{0} & \textcolor{ForestGreen}{Ottimale}\\
						\ndpv & \hyperref[MCC]{ECNC} &\textcolor{ForestGreen}{0} & \textcolor{ForestGreen}{Ottimale}\\
						\sdfv & \hyperref[MCC]{ECNC} & \textcolor{ForestGreen}{0} & \textcolor{ForestGreen}{Ottimale}\\
						\adrv & \hyperref[MCC]{ECNC} & \textcolor{ForestGreen}{0} & \textcolor{ForestGreen}{Ottimale}\\
						\glv  & \hyperref[MCC]{ECNC} & \textcolor{ForestGreen}{0} & \textcolor{ForestGreen}{Ottimale}\\
						\vunoi& \hyperref[MCC]{ECNC} & \textcolor{ForestGreen}{0} & \textcolor{ForestGreen}{Ottimale}\\
						\vduei& \hyperref[MCC]{ECNC} & \textcolor{ForestGreen}{0} & \textcolor{ForestGreen}{Ottimale}\\
						\vtrei & \hyperref[MCC]{ECNC} & \textcolor{ForestGreen}{0} & \textcolor{ForestGreen}{Ottimale}\\
						\vquattroi & \hyperref[MCC]{ECNC} & \textcolor{ForestGreen}{0} & \textcolor{ForestGreen}{Ottimale}\\
						\vunoe & \hyperref[MCC]{ECNC} & \textcolor{ForestGreen}{0} & \textcolor{ForestGreen}{Ottimale}\\
						\vduee & \hyperref[MCC]{ECNC} & \textcolor{ForestGreen}{0} & \textcolor{ForestGreen}{Ottimale}\\
						\hline
					\end{tabular}
					\caption{Resoconto correttezza concettuale}
					\label{tab_resoconto_correttezza_concettuale}
				\end{table}
				
				\subparagraph{Considerazioni finali}
					Per tutti i documenti non sono stati rilevati errori concettuali non corretti, pertanto le metriche assumono un valore ottimale.
					
				
				
					