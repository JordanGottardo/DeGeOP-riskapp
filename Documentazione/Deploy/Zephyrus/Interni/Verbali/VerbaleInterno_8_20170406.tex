\documentclass[a4paper,11pt]{article}
\usepackage{../../../../Template/zemplateVerb-deploy}

\docVerbV{8}
\docTitle{Verbale interno - \getdocVerbV}
\docVersion{1.0.0}
\docCreationDate{\frmdata{06}{04}{2017}}
\docLastUpdateDate{\frmdata{08}{04}{2017}}
\docStatus{Approvato}
\docEditors{Jordan Gottardo}
\docVerificators{Giovanni Prete}
\docApprovers{Daniel De Gaspari}
\docUse{Interno}
\docDestination{\Tullio \\ & \Cardin \\ & \zephyrus}
\docJournal{
1.0.0 & \frmdata{08}{04}{2017} & Daniel De Gaspari & \responsabile & Approvazione documento \\
0.1.0 & \frmdata{06}{04}{2017} & Giovanni Prete & \verificatore & Verifica documento \\
0.0.1 & \frmdata{06}{04}{2017} & Jordan Gottardo & \programmatore & Stesura documento \\
}

\begin{document}
	Introduzione
	Introduzione

	\section{Estremi della riunione}
	\begin{itemize}
		\item \textbf{data:} \frmdata{06}{04}{2017};
		\item \textbf{ora inizio:} \frmora{15}{00};
		\item \textbf{ora fine:} \frmora{16}{30};
		\item \textbf{luogo:} Torre Archimede - Padova;
		\item \textbf{segretario:} Jordan Gottardo;
		\item \textbf{partecipanti:}
			\begin{itemize}
				\item Daniel De Gaspari;
				\item Giovanni Damo;
				\item Giovanni Prete;
				\item Giulia Petenazzi;
				\item Jordan Gottardo;
				\item Leonardo Brutesco;
				\item Marco Pasqualini.
			\end{itemize}
		\item \textbf{assenti:}
			\begin{itemize}
				\item nessuno.
			\end{itemize}
	\end{itemize}
	\section{Ordine del giorno}
		\begin{itemize}
			\item presa d'atto dell'uscita di Giovanni Damo dal gruppo, come precedentemente pianificato;
			\item resoconto del pessimo lavoro durante l'attività di codifica;
			\item discussione riguardante il fallimento dell'applicazione del modello di sviluppo incrementale;
			\item discussione riguardante la possibilità di saltare la consegna della \revaqual{} di aprile.
		\end{itemize}
	\section{Verbale della riunione}
		\begin{itemize}
			\item Giovanni Damo lascia ufficialmente il gruppo. Tutti i membri gli augurano buona fortuna per la laurea;
			\item i \programmatori{} espongono le motivazioni per cui l'attività di codifica ha prodotto risultati di qualità pessima. Non sono state seguite le norme di codifica fissate;
			\item il \responsabile{} prende atto della mancata applicazione del modello di sviluppo incrementale per quanto riguarda l'attività di codifica e testing e conseguente decisione correttiva;
			\item il \responsabile{} indice una votazione a maggioranza per decidere se partecipare alla \revaqual{} di aprile. \\
			Motivazioni a favore della consegna:
			\begin{itemize}
				\item dato che alcuni membri del gruppo desiderano iniziare lo stage obbligatorio a breve, essi ritengono che per il benessere psicologico del gruppo sia fondamentale partecipare alla \revaqual{} di aprile. Iniziare lo stage con due revisioni da sostenere porterebbe ulteriore stress;
				\item altri membri sostengono che, data la mancanza di certezza di superare lo scritto di Ingegneria del Software del 18 aprile, per poter iniziare lo stage in tempo è fondamentale consegnare il progetto alla prima \revacc{} possibile;
				\item data la diminuzione del numero di requisiti obbligatori a causa della decisione VE\_3.2, il team ritiene di poter implementare tutte le funzionalità non sviluppate nel prossimo periodo.
			\end{itemize}
			Motivazioni contro la consegna:
			\begin{itemize}
				\item la qualità del codice è pessima; le norme di codifica, già poco approfondite, non sono nemmeno state rispettate;
				\item convenienza nello studiare per l'esame scritto del 18 aprile.
			\end{itemize}
		\end{itemize}
	\section{Decisioni prese}
		\begin{itemize}
	        \itemVI gli \amministratori{} revocano l'accesso a Giovanni Damo alle repository; 
	        \itemVI le norme di codifica verranno revisionate ed estese in accordo alle segnalazioni dei \programmatori. Verranno effettuati maggiori controlli durante l'attività di codifica del prossimo periodo;
	        \itemVI aumento delle ore del \responsabile{} come descritto dal \pdpvtre{}, per cercare di applicare con più precisione il modello di sviluppo incrementale durante il prossimo periodo;
	        \itemVI con una votazione di 5 a 1, viene deciso di partecipare alla consegna della \revaqual{} di aprile.
		\end{itemize}
\end{document}
