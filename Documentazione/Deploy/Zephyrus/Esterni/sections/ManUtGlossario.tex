\section{Glossario}
	\subsection{A}
		\mgref{Arco}
		Collegamento orientato tra due nodi.
		
		\mgref{Asset}
		Fabbricato con importanza strategica per il processo produttivo di un'azienda. Un asset può contenere uno o più nodi.
		
		\mgref{Asus}
		Azienda produttrice di dispositivi tecnologici di varia tipologia.
	\subsection{B}
		\mgref{Browser}
		Il web browser, o più semplicemente browser, è un'applicazione per il recupero, la presentazione e la navigazione di risorse web. Tali risorse (ad esempio pagine web, immagini, video) sono a disposizione sul World Wide Web su una rete locale o sullo stesso computer dove il browser è in esecuzione. 
	\subsection{C}
		\mgref{Chrome}
		\mglo{Browser}{Browser} web sviluppato da Google.
	 %\subsection{D}
%	\subsection{E}
	\subsection{F}
		\mgref{Firefox}
		\mglo{Browser}{Browser} web  libero e multipiattaforma, mantenuto da Mozilla Foundation.
	\subsection{G}
		\mgref{Gesture}
		Combinazione di movimenti e click del dispositivo di puntamento (ad esempio il mouse) che vengono riconosciuti come comandi specifici.
		
		\mgref{Gruppo}
		Componenti che fanno parte del gruppo \zephyrus.
%	\subsection{H}
	\subsection{I}
		\mgref{iOS}
		Sistema operativo sviluppato da Apple Inc.
	
		\mgref{Ipad}
		Tablet prodotto dall'azienda Apple Inc.
%	\subsection{J}
%	\subsection{K}
%	\subsection{L}
		\mgref{Linux}
		Linux è una famiglia di sistemi operativi di tipo Unix-like, rilasciati sotto varie distribuzioni, aventi la caratteristica comune di utilizzare come nucleo il kernel Linux. Ubuntu è la distribuzione di Linux più utilizzata.
	\subsection{M}
		\mgref{MacOS}
		Sistema operativo sviluppato da Apple.
	\subsection{N}
		\mgref{Nodo}
		Oggetto che fa parte dei processo produttivo aziendale. È contenuto all'interno di un asset.
	\subsection{O}
		\mgref{Offline}
		Il dispositivo non è connesso alla rete.
		
%	\subsection{P}
%	\subsection{Q}
%	\subsection{R}
	\subsection{S}
		\mgref{Safari}
		\mglo{Browser}{browser} web sviluppato dalla Apple.
%	\subsection{T}
%	\subsection{U}
%	\subsection{V}
		\mgref{VirtualBox}
		Software open source per l'esecuzione di macchine virtuali.
	\subsection{W}
		\mgref{Windows}
		Sistema operativo sviluppato da Microsoft.
%	\subsection{X}
%	\subsection{Y}
%	\subsection{Z}