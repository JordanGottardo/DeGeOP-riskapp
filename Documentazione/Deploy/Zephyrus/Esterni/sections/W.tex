

\section{W}

\gref{W3C}
\textit{World Wide Web Consortium}. Organizzazione non governativa internazionale che ha come scopo quello di sviluppare tutte le potenzialità del World Wide Web. Al fine di riuscire nel proprio intento, la principale attività svolta dal W3C consiste nello stabilire standard tecnici che riguardino sia i linguaggi di markup che i protocolli di comunicazione.


\gref{Walkthrough}
Tecnica di analisi statica che consiste nella lettura a largo spettro del documento o del codice, al fine di trovare anomalie, senza avere un'idea precisa degli errori da cercare.


\gref{Web browser}
Il web browser, o più semplicemente browser, è un'applicazione per il recupero, la presentazione e la navigazione di risorse web. Tali risorse (ad esempio pagine web, immagini, video) sono a disposizione sul World Wide Web su una rete locale o sullo stesso computer dove il browser è in esecuzione. 

\gref{WebStorm}
\glo{IDE}{IDE} che offre supporto allo sviluppo di applicazioni \glo{JavaScript}{JavaScript} sia client che server. Offre inoltre supporto a \glo{Node.js}{Node.JS}, \glo{HTML}{HTML}, \glo{CSS}{CSS} e frameworks come AngularJS e a librerie JavaScript come React.

%  IDE che offre supporto ad una moltitudine di linguaggi, quali JavaScript, Node.JS, HTML and CSS e i loro successori. Offre inoltre a frameworks come AngularJS e a librerie JavaScript come React.

%Supported frameworks include AngularJS, React, Meteor

\gref{Windows}
Sistema operativo sviluppato da Microsoft.

