\newpage

\section{Estensione delle funzionalità}
Per ragioni di tempo e competenze alcune funzionalità non sono state implementate. Inoltre lo sviluppo è stato guidato dall'architettura preesistente di \riskapp{} e dalle API disponibili al momento della presa in consegna del progetto \progetto.

\subsection{Scenari di danno}
Le funzionalità di aggiunta, visualizzazione, modifica ed eliminazione degli scenari di danno non sono state implementate. Le operazioni da svolgere per implementarle sono:
\begin{itemize}
	\item ottenere le API da \riskapp{} per interfacciarsi con il loro server;
	\item inserire in \textit{StorePkg} le classi necessarie all'implementazione della parte di Store relativa agli scenari, come descritto dal documento di \textit{Specifica Tecnica};
	\item inserire in \textit{ActionsPkg} le classi necessarie all'implementazione delle actions che operano sugli scenari, come descritto dal documento di \textit{Specifica Tecnica};
	\item inserire in \textit{ReducerPkg} le funzioni necessarie all'implementazione dei reducer che operano sugli scenari, come descritto dal documento di \textit{Specifica Tecnica};
	\item inserire in \textit{ViewPkg} le componenti React necessarie all'aggiunta, visualizzazione, modifica ed eliminazione degli scenari di danno, come descritto dal documento di \textit{Specifica Tecnica}.
	\item inserire in \textit{CallManagerPkg} le classi necessarie all'interfacciamento con il server \riskapp{}
	I campi dati relativi agli Scenari di danno, essendo informazioni di dettaglio, non sono riportati nella \textit{Specifica Tecnica}, ma sono ricavabili dal documento di \adr{}.
	Un buon esempio di come potrebbero essere implementati dal punto di vista dell'utente finale è descritto dal documento di \textit{Specifica Tecnica} nella sezione \textit{Attività}.
\end{itemize}

\subsection{Analisi di rischio}
La funzionalità di avvio ed eliminazione delle analisi di rischio non sono state implementate. Le operazioni da svolgere per implementarle sono:
\begin{itemize}
	\item ottenere le API da \riskapp{} per interfacciarsi con il loro server;
	\item inserire in \textit{StorePkg} le classi necessarie all'implementazione della parte di Store relativa alle analisi di rischio, come descritto dal documento di \textit{Specifica Tecnica};
	\item inserire in \textit{ActionsPkg} le classi necessarie all'implementazione delle actions che operano sulle analisi di rischio, come descritto dal documento di \textit{Specifica Tecnica};
	\item inserire in \textit{ReducerPkg} le funzioni necessarie all'implementazione dei reducer che operano sulle analisi di rischio, come descritto dal documento di \textit{Specifica Tecnica};
	\item inserire in \textit{ViewPkg} le componenti React necessarie all'avvio ed all'eliminazione delle analisi di rischio, come descritto dal documento di \textit{Specifica Tecnica}.
	\item inserire in \textit{CallManagerPkg} le classi necessarie all'interfacciamento con il server \riskapp{}.
	Un buon esempio di come potrebbero essere implementate dal punto di vista dell'utente finale è descritto dal documento di \textit{Specifica Tecnica} nella sezione \textit{Attività}.
\end{itemize}

\subsection{Nuova tipologia di nodo}
I nodi possono essere di diverse tipologie. In caso si volesse inserire una nuova tipologia di nodo, le operazioni da svolgere per implementarli sono:
\begin{itemize}
	\item ottenere le API da \riskapp{} per interfacciarsi con il loro server;
	\item inserire in \textit{StorePkg} la classe necessaria all'implementazione della parte di Store relativa a quel nodo. Questa classe deve estendere la classe \textit{Node};
	\item inserire in \textit{ActionsPkg} le classi necessarie all'implementazione delle actions che operano su quel nuovo tipo di nodo;
	\item inserire in \textit{ReducerPkg} le funzioni necessarie all'implementazione dei reducer che operano su quel nuovo tipo di nodo;
	\item inserire in \textit{ViewPkg} le componenti React necessarie all'avvio ed all'eliminazione delle analisi di rischio. In particolare nelle classi \textit{ViewPkg}::\textit{SidebarPkg}::\textit{ContentPkg}::\textit{InsertNodeContent} e \textit{ViewPkg}::\textit{SidebarPkg}::\textit{ContentPkg}::\textit{ViewNodeContent}, nella funzione \textit{specializedField()} che si occupa della renderizzazione dei campi dati specifici, aggiungere la sezione che renderizza i campi dati della nuova tipologia di nodo;
	\item inserire in \textit{CallManagerPkg} le classi necessarie all'interfacciamento con il server \riskapp{}.
\end{itemize}

%\medskip
%\begin{lstlisting}[caption=My Javascript Example]
%Name.prototype = {
%methodName: function(params){
%	var doubleQuoteString = "some text";
%	var singleQuoteString = 'some more text';
%	// this is a comment
%			if(this.confirmed != null && typeof(this.confirmed) == Boolean && this.confirmed == true){
%		document.createElement('h3');
%		$('#system').append("This looks great");
%		return false;
%} else {
%	throw new Error;
%}
%}
%}
%\end{lstlisting}

%Le principali funzionalità:
%multi processo
%api 
%portare le loro funzionalita in single page
%visualizzazione degli scenari e i risulatati dell'analisi su quegli scenari
%	gradiente
%	hazard
%report/statistiche -> print
%stampare schema del processo (quindi altre visualizzazioni)
%(mappa satellitare)