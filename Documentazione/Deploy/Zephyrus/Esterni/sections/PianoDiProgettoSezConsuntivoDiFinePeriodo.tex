%\section {Consuntivo}

\section {Consuntivo di periodo}
	\subsection {Introduzione}
	In questa sezione viene presentato il bilancio orario ed economico del progetto a consuntivo. Per ogni \glo{Periodo}{periodo} viene steso un consuntivo che mostra il quantitativo di ore rendicontate investite durante quel periodo e i totali (rendicontati, non rendicontati e complessivi) delle ore spese fino al termine del periodo preso in esame. Al termine del progetto verrà presentato un consuntivo finale.
	\\Il bilancio orario può essere:
	\begin{itemize}
	\item \textbf{positivo:} il preventivo orario ha superato il consuntivo orario;
	\item \textbf{negativo:} il consuntivo orario ha superato il preventivo orario;
	\item \textbf{in pari:} il preventivo orario conincide con il consuntivo orario.
	\end{itemize}
	Il bilancio economico può essere:
	\begin{itemize}
	\item \textbf{positivo:} il preventivo economico ha superato il consuntivo economico;
	\item \textbf{negativo:} il consuntivo economico ha superato il preventivo economico;
	\item \textbf{in pari:}  il preventivo economico conincide con il consuntivo economico.
	\end{itemize}
	
	\newpage
	\subsection {Dettaglio periodi}
		\subsubsection {Periodo: An - Analisi}
			\paragraph{Consuntivo di periodo orario}
			Si ricorda che le ore di questo periodo figurano come non rendicontate.
%---------------------------------------------------------------------
								\begin{table}[H] \begin{center} \begin{tabular}{llllllll}
								\toprule
								\textbf{Nominativo}	&	\textbf{Re}		&	\textbf{Am}		&	\textbf{At}		&	\textbf{Pj}		&	\textbf{Pr}		&	\textbf{Ve}		&	\textbf{Tot}		 \\
								\midrule																						 
								Brutesco	&	-		&	-		&	13	(+1)	&	-		&	-		&	12		&	25	(+1)\\
								Damo		&	-		&	9		&	16	(+1)	&	-		&	-		&	-		&	25	(+1)\\
								De Gaspari	&	-		&	-		&	13	(+1)	&	-		&	-		&	12		&	25	(+1)\\
								Gottardo	&	14		&	-		&	11	(+1)	&	-		&	-		&	-		&	25	(+1)\\
								Pasqualini	&	-		&	-		&	13	(+1)	&	-		&	-		&	12		&	25	(+1)\\
								Petenazzi	&	7		&	-		&	18	(+1)	&	-		&	-		&	-		&	25	(+1)\\
								Prete		&	-		&	10		&	15	(+1)	&	-		&	-		&	-		&	25	(+1)\\
								\midrule																					
								Tot in ore	&	21		&	19		&	99	(+7)	&	0		&	0		&	36		&	175	(+7)\\
								\bottomrule
								\end{tabular} \end{center} \caption{Prospetto orario a consuntivo per il periodo di analisi}\label{tab:oreAnalisiCons}
									\end{table}
%---------------------------------------------------------------------
			\paragraph{Consuntivo di periodo economico}
			La differenza tra le ore a consuntivo e preventivo di questo periodo non sono a carico del proponente e quindi saranno considerate solamente come ore di investimento non rendicontate.
%-----------------------------------------------------------------


								\begin{table}[H] \begin{center} \begin{tabular}{llllllll}
								\toprule
									&	\textbf{Re}	&	\textbf{Am}	&	\textbf{At}	&	\textbf{Pj}	&	\textbf{Pr}	&	\textbf{Ve}	&	\textbf{Tot}	 \\
								\midrule
								Tot in ore	&	21		&	19		&	99 (+7)	&	0		&	0		&	36		&	175	(+7)\\
								Tot in €	&	 € 630 		&	 € 380 		&	 € 2.475 		&	 € -   		&	 € -   		&	 € 540 		&	 € 4.025 	\\
								\bottomrule
								\end{tabular} \end{center} \caption{Prospetto economico - Periodo:
								An
								}\label{tab:sAnCons} \end{table}

%-----------------------------------------------------------------
			\paragraph{Conclusioni}
			Il lavoro degli \analisti{} ha richiesto più tempo di quello preventivato in quanto lo studio del dominio si è dimostrato più difficile di quanto previsto. Come si vede dalla tabella \ref{tab:oreAnalisiCons} il bilancio orario risulta negativo in quanto eccede di 7 ore rispetto a quanto pianificato.
			Come si vede dalla tabella \ref{tab:sAnCons} il bilancio economico è negativo per un importo pari a -175€.
			Queste variazioni rispetto al preventivo non avranno impatto sul costo finale in quanto le ore aggiuntive sono considerate di investimento.

%---------------------------------------------------------------------------------
	\newpage
	\subsubsection{Pl - Progettazione logica}
	\paragraph{Consuntivo di periodo orario}
%-----------------------------------------------------------------------------------
								\begin{table}[h] \begin{center} \begin{tabular}{llllllll}																						
								\toprule																						
									&	Re		&	Am		&	An		&	Pj		&	Pr		&	Ve		&	Tot	 \\ 	
								\midrule																		
								Brutesco	&	5		&	-		&	-		&	21		&	-		&	-		&	26	\\
								Damo		&	5		&	-		&	-		&	30		&	-		&	-		&	35	\\
								De Gaspari	&	-		&	-		&	-		&	6	(-4)	&	-		&	20	(+4)	&	26	\\
								Gottardo	&			&	-		&	-		&	10		&	-		&	16		&	26	\\
								Pasqualini	&	-		&	5		&	5		&	16		&	-		&	-		&	26	\\
								Petenazzi	&	-		&	4		&	5		&	17		&	-		&	-		&	26	\\
								Prete	&	-		&	-		&	-		&	14	(+4)	&	-		&	12	(-4)	&	26	\\
								\midrule																					
								Tot in ore	&	10		&	9		&	10		&	114(+0)		&	0		&	48(+0)		&	191	\\
								
								\bottomrule
								\end{tabular} \end{center} \caption{Prospetto a consuntivo orario per il periodo di																						
									Progettazione logica																						
									}\label{tab:orePl} \end{table}	
%-----------------------------------------------------------------------------------
	\paragraph{Consuntivo di periodo economico}
%-----------------------------------------------------------------------------------
								\begin{table}[H] \begin{center} \begin{tabular}{llllllll}																						
							\toprule	
								&	Re		&	Am		&	An		&	Pj		&	Pr		&	Ve		&	Tot	 \\ 	
							\midrule																		
							Tot in ore	&	10		&	9		&	10		&	114(+0)		&	0		&	48(+0)		&	191	\\
							Tot in €	&	 € 300 		 & 	 € 180 		 & 	 € 250 		 & 	 € 2.508 		 & 	 € -   		 & 	 € 720 		 & 	 € 3.958 	\\
							\bottomrule																						
							\end{tabular} \end{center} \caption{Prospetto a consuntivo economico per il periodo di																						
							Progettazione logica																						
							}\label{tab:sPl} \end{table}
	
	%----------------------------------------------------------------------------	
				\paragraph{Conclusioni}
				E’ stato necessario un ricalibro delle ore a causa di problemi personali di un membro, come indicato
				dall’analisi dei rischi. Nonostante ciò, il consuntivo orario presenta gli stessi totali esposti nel preventivo orario di questo periodo.
%---------------------------------------------------------------------------------
\newpage
\subsubsection{PCV - Progettazione Codifica Validazione}
\paragraph{Consuntivo di periodo orario}
%-----------------------------------------------------------------------------------
							\begin{table}[H] \begin{center} \begin{tabular}{llllllll}
										\toprule
										\textbf{Nominativo}	&	\textbf{Re}	&	\textbf{Am}	&	\textbf{At}	&	\textbf{Pj}	&	\textbf{Pr}	&	\textbf{Ve}	&	\textbf{Tot}\\
										\midrule
										Brutesco	&	-		&	4		&	-		&	5		&	14	(-1)&	29	&	52	\\
										Damo		&	-		&	-		&	1	(+1)&	9		&	20	(-2)&	39	&	69	\\
										De Gaspari	&	10		&	-		&	-		&	8		&	24	(-1)&	13	&	55	\\
										Gottardo	&	-		&	6		&	-		&	28		&	21	(-1)&	-	&	55	\\
										Pasqualini	&	-		&	4		&	-		&	29		&	15	(-1)&	7	&	55	\\
										Petenazzi	&	-		&	5		&	-		&	12		&	19	(-1)&	19	&	55	\\
										Prete		&	10		&	-		&	-		&	6		&	20	(-1)&	16	&	52	\\
										\midrule																				
										Tot in ore	&	20		&	19		&	1	(+1)&	97		&	133	(-8)&	123	&	393	(-7)\\
																				
										\bottomrule
									\end{tabular} \end{center} \caption{Consuntivo di periodo orario - Periodo:
									PCV
								} \end{table}
%-----------------------------------------------------------------------------------
\paragraph{Consuntivo di periodo economico}
%-----------------------------------------------------------------------------------
\begin{table}[H] \begin{center} \begin{tabular}{llllllll}																						
			\toprule	
			&	Re		&	Am		&	An		&	Pj		&	Pr		&	Ve		&	Tot	 \\ 	
			\midrule																		
			Tot in ore	&	20		&	19		&	1	(+1)&	97		&	133	(-8)&	123	&	393	(-7)\\
			Tot in €	&	 € 600 		 & 	 € 380 		 & 	 € 25 		 & 	 € 2.134 		 & 	 € 1.995 		 & 	 € 1.845 		 & 	 € 6.979 	\\
			\bottomrule																						
		\end{tabular} \end{center} \caption{Prospetto a consuntivo economico per il periodo PCV																					
	}\label{tab:s_PCV_c} \end{table}

%----------------------------------------------------------------------------	

				
	\paragraph{Conclusioni}
	A livello di pianificazione, i codificatori hanno impiegato più tempo di quanto previsto, anche a causa di un rischio non preventivato che si è verificato: non sono state seguite le norme di codifica stabilite. La codifica è stata, specialmente nei momenti iniziali poco organizzata e alquanto indisciplinata. Il \responsabile{} ha cercato di arginare il fenomeno richiamando gli amministratori in modo che stabilissero un maggior rigore specialmente per quanto riguarda :
	\begin{itemize}
		\item il controllo di configurazione;
		\item l'integrazione degli strumenti e delle tecnologie.
	\end{itemize}
	Le conseguenze principali sono:
	\begin{itemize}
		\item gli incrementi 5, 6, 7, 8 pianificati (legati a requisiti facoltativi) non sono stati svolti;
		\item negli incrementi 1, 2, 3, 4, si è:
				\begin{itemize}
					\item progettato a basso livello l'incremento;
					\item parzialmente codificato l'incremento;
					\item parzialmente effettuati i test di unita dell'incremento.
				\end{itemize}
			non effettuando alcuna validazione del prodotto, e portando di fatto a uno sviluppo di carattere sequenziale.
	\end{itemize}
	Il sorgere di alcuni problemi col proponente come indicato dall'analisi dei rischi, ha portato alla formazione di un'ora di analisi, necessaria per la modifica dell' \adr{} e a una diminuzione delle ore di codifica, anche grazie al fatto che non si sono svolti gli incrementi opzionali.
	Il consuntivo orario ed economico presenta quindi delle variazioni rispetto a quanto era stato preventivato nelle tabelle \ref{tab:h_PCV} e \ref{tab:s_PCV}. In particolare sono state impiegate 7 ore in meno diminuendo la spesa di 95€.
	Per cercare di risolvere questi problemi, è stato steso un adeguato preventivo a finire in  \ref{preventivo_a_finire}.
	\newpage
	\subsection{Totale non rendicontato}
		\subsubsection{Consuntivo di periodo orario}
%-----------------------------------------------------------------
						\begin{table}[h] \begin{center} \begin{tabular}{llllllll}																						
						\toprule
									&	Re		&	Am		&	An		&	Pj		&	Pr		&	Ve		&	Tot	 \\ 	
						\midrule																					
						Brutesco	&	2		&	3		&	14		&	2		&	2		&	14		&	37		\\
						Damo	&	2		&	12		&	17		&	2		&	2		&	2		&	37		\\
						De Gaspari	&	2		&	0		&	14		&	2		&	2		&	14		&	34		\\
						Gottardo	&	16		&	3		&	12		&	2		&	2		&	2		&	37		\\
						Pasqualini	&	0		&	3		&	14		&	2		&	2		&	14		&	35		\\
						Petenazzi	&	9		&	3		&	19		&	2		&	2		&	2		&	37		\\
						Prete	&	2		&	13		&	16		&	2	&	2		&	2		&	37		\\
						\midrule																Tot in ore	&	33		&	37		&	106		&	14		&	14		&	50		&	254		\\	
						\bottomrule																						
						\end{tabular} \end{center} \caption{Prospetto orario a consuntivo totale non rendicontato fino alla fine del periodo PCV																						
						}\label{tab:oreNonRend} \end{table}							
%-----------------------------------------------------------------
		\subsubsection{Consuntivo di periodo economico}
%-----------------------------------------------------------------
						\begin{table}[H] \begin{center} \begin{tabular}{llllllll}
						\toprule
							&	\textbf{Re}	&	\textbf{Am}	&	\textbf{At}	&	\textbf{Pj}	&	\textbf{Pr}	&	\textbf{Ve}	&	\textbf{Tot}\\
			
						\midrule
						Tot in ore	&	33		&	37		&	106		&	14		&	14		&	50		&	254		\\
						Tot in €	&	 € 990 		 & 	 € 740 		 & 	 € 2.650 		 & 	 € 308 		 & 	 € 210 		 & 	 € 750 		 & 	 € 5.648 		\\
						\bottomrule
						\end{tabular} \end{center} \caption{Prospetto economico a consuntivo totale non rendicontato fino alla fine del periodo PCV		
						}\label{tab:s_TotaleNonRendicontato} \end{table}
%-----------------------------------------------------------------
	
	\newpage
	\subsection{Totale complessivo}
		\subsubsection{Consuntivo di periodo orario}

%-----------------------------------------------------------------------------------------																					
						\begin{table}[h] \begin{center} \begin{tabular}{llllllll}																					
						\toprule																					
							&	Re		&	Am		&	An		&	Pj		&	Pr		&	Ve		&	Tot	\\
						\midrule																					
						Brutesco	&	7		&	7		&	14		&	28		&	16		&	43		&	115	\\
						Damo	&	7		&	12		&	18		&	41		&	22		&	41		&	141	\\
						De Gaspari	&	12		&	0		&	14		&	16		&	26		&	47		&	115	\\
						Gottardo	&	16		&	9		&	12		&	40		&	23		&	18		&	118	\\
						Pasqualini	&	0		&	12		&	19		&	47		&	17		&	21		&	116	\\
						Petenazzi	&	9		&	12		&	24		&	31		&	21		&	21		&	118	\\
						Prete	&	12		&	13		&	16		&	22		&	22		&	30		&	115	\\
						\midrule																					
						Tot in ore	&	63		&	65		&	117		&	225		&	147		&	221		&	838	\\
						\bottomrule																					
						\end{tabular} \end{center} \caption{Prospetto orario a consuntivo totale complessivo fino alla fine del periodo PCV															
						} \end{table}
%-----------------------------------------------------------------------------------------																					
		\subsubsection{Consuntivo di periodo economico}
%-----------------------------------------------------------------
						\begin{table}[H] \begin{center} \begin{tabular}{llllllll}
						\toprule
							&	\textbf{Re}	&	\textbf{Am}	&	\textbf{At}	&	\textbf{Pj}	&	\textbf{Pr}	&	\textbf{Ve}	&	\textbf{Tot}\\
						\midrule																					
						Tot in ore	&	63		&	65		&	117		&	225		&	147		&	221		&	838	\\
						Tot in €	&	 € 1.890 		 & 	 € 1.300 		 & 	 € 2.925 		 & 	 € 4.950 		 & 	 € 2.205 		 & 	 € 3.315 		 & 	 € 16.585 	\\
						\bottomrule			
						\end{tabular} \end{center} \caption{Prospetto economico a consuntivo totale complessivo fino alla fine del periodo PCV			
						}\label{tab:s_TotaleNonRendicontato} \end{table}
%-----------------------------------------------------------------
	
	\newpage
	\subsection{Totale rendicontato}
		\subsubsection{Consuntivo di periodo orario}
		
		%-----------------------------------------------------------------------------------------																					
		\begin{table}[h] \begin{center} \begin{tabular}{llllllll}																					
					\toprule																					
					&	Re		&	Am		&	An		&	Pj		&	Pr		&	Ve		&	Tot	\\
					\midrule																					
					Brutesco	&	5		&	4		&	0		&	26		&	14		&	29		&	78	\\
					Damo		&	5		&	0		&	1		&	39		&	20		&	39		&	104	\\
					De Gaspari	&	10		&	0		&	0		&	14		&	24		&	33		&	81	\\
					Gottardo	&	0		&	6		&	0		&	38		&	21		&	16		&	81	\\
					Pasqualini	&	0		&	9		&	5		&	45		&	15		&	7		&	81	\\
					Petenazzi	&	0		&	9		&	5		&	29		&	19		&	19		&	81	\\
					Prete		&	10		&	0		&	0		&	20		&	20		&	28		&	78	\\
					\midrule																					
					Tot in ore	&	30		&	28		&	11		&	211		&	133		&	171		&	584	\\
					\bottomrule																					
				\end{tabular} \end{center} \caption{Prospetto orario a consuntivo totale rendicontato fino alla fine del periodo PCV															
			} \end{table}
			%-----------------------------------------------------------------------------------------																					
			\subsubsection{Consuntivo di periodo economico}
			%-----------------------------------------------------------------
			\begin{table}[H] \begin{center} \begin{tabular}{llllllll}
						\toprule
						&	\textbf{Re}	&	\textbf{Am}	&	\textbf{At}	&	\textbf{Pj}	&	\textbf{Pr}	&	\textbf{Ve}	&	\textbf{Tot}\\
						\midrule																					
						Tot in ore	&	30		&	28		&	11		&	211		&	133		&	171		&	584	\\
						Tot in €	&	 € 900 		 & 	 € 560 		 & 	 € 275 		 & 	 € 4.642 		 & 	 € 1.995 		 & 	 € 2.565 		 & 	 € 10.937 	\\
						\bottomrule			
					\end{tabular} \end{center} \caption{Prospetto economico a consuntivo totale rendicontato fino alla fine il periodo PCV	
				}\label{tab:s_TotaleNonRendicontato} \end{table}
			%-----------------------------------------------------------------

.
	\subsection{Considerazioni finali sul consuntivo di periodo}
	\label{consid_cons_pl}
	Il totale rendicontato a consuntivo (e di conseguenza anche quello complessivo) presenta delle variazioni rispetto a quanto preventivato; in particolare il bilancio orario si chiude in positivo di 7 ore, mentre quello economico si chiude in positivo di 95€. Grazie a quanto emerso nell'analisi dei rischi e nel consuntivo di fine periodo è stato steso un adeguato preventivo a finire in  \ref{preventivo_a_finire}.