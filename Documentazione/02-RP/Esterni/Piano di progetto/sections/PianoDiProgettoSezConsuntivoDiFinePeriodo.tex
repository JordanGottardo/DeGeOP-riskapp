%\section {Consuntivo}

\section {Consuntivo}
	\subsection {Introduzione}
	In questa sezione viene presentato il bilancio orario ed economico del progetto a consuntivo. Per ogni \glo{Periodo}{periodo} viene steso un consuntivo che mostra il quantitativo di ore rendicontate investite durante quel periodo e i totali (rendicontati, non rendicontati e complessivi) delle ore spese fino al termine del periodo preso in esame. Al termine del progetto verrà presentato un consuntivo finale.
	Il bilancio orario può essere:
	\begin{itemize}
	\item \textbf{positivo:} il preventivo orario ha superato il consuntivo orario;
	\item \textbf{negativo:} il consuntivo orario ha superato il preventivo orario;
	\item \textbf{in pari:} il preventivo orario conincide con il consuntivo orario.
	\end{itemize}
	Il bilancio economico può essere:
	\begin{itemize}
	\item \textbf{positivo:} il preventivo economico ha superato il consuntivo economico;
	\item \textbf{negativo:} il consuntivo economico ha superato il preventivo economico;
	\item \textbf{in pari:}  il preventivo economico conincide con il consuntivo economico.
	\end{itemize}
	
	\newpage
	\subsection {Consuntivi di periodo}
		\subsubsection {Periodo: An - Analisi}
			\paragraph{Consuntivo orario}
			Si ricorda che le ore di questo periodo figurano come non rendicontate.
%---------------------------------------------------------------------
								\begin{table}[H] \begin{center} \begin{tabular}{llllllll}
								\toprule
								\textbf{Nominativo}	&	\textbf{Re}		&	\textbf{Am}		&	\textbf{At}		&	\textbf{Pj}		&	\textbf{Pr}		&	\textbf{Ve}		&	\textbf{Tot}		 \\
								\midrule																						 
								Brutesco	&	-		&	-		&	13	(+1)	&	-		&	-		&	12		&	25	(+1)\\
								Damo		&	-		&	9		&	16	(+1)	&	-		&	-		&	-		&	25	(+1)\\
								De Gaspari	&	-		&	-		&	13	(+1)	&	-		&	-		&	12		&	25	(+1)\\
								Gottardo	&	14		&	-		&	11	(+1)	&	-		&	-		&	-		&	25	(+1)\\
								Pasqualini	&	-		&	-		&	13	(+1)	&	-		&	-		&	12		&	25	(+1)\\
								Petenazzi	&	7		&	-		&	18	(+1)	&	-		&	-		&	-		&	25	(+1)\\
								Prete		&	-		&	10		&	15	(+1)	&	-		&	-		&	-		&	25	(+1)\\
								\midrule																					
								Tot in ore	&	21		&	19		&	99	(+7)	&	0		&	0		&	36		&	175	(+7)\\
								\bottomrule
								\end{tabular} \end{center} \caption{Prospetto orario a consuntivo per il periodo di analisi}\label{tab:oreAnalisiCons}
									\end{table}
%---------------------------------------------------------------------
			\paragraph{Consuntivo economico}
			La differenza tra le ore a consuntivo e preventivo di questo periodo non sono a carico del proponente e quindi saranno considerate solamente come ore di investimento non rendicontate.
%-----------------------------------------------------------------


								\begin{table}[H] \begin{center} \begin{tabular}{llllllll}
								\toprule
									&	\textbf{Re}	&	\textbf{Am}	&	\textbf{At}	&	\textbf{Pj}	&	\textbf{Pr}	&	\textbf{Ve}	&	\textbf{Tot}	 \\
								\midrule
								Tot in ore	&	21		&	19		&	99 (+7)	&	0		&	0		&	36		&	175	(+7)\\
								Tot in €	&	 € 630 		&	 € 380 		&	 € 2.475 		&	 € -   		&	 € -   		&	 € 540 		&	 € 4.025 	\\
								\bottomrule
								\end{tabular} \end{center} \caption{Prospetto economico - Periodo:
								An
								}\label{tab:sAnCons} \end{table}

%-----------------------------------------------------------------
			\paragraph{Conclusioni}
			Il lavoro degli \analisti{} ha richiesto più tempo di quello preventivato in quanto lo studio del dominio si è dimostrato più difficile di quanto previsto. Come si vede dalla tabella \ref{tab:oreAnalisiCons} il bilancio orario risulta negativo in quanto eccede di 7 ore rispetto a quanto pianificato.
			Come si vede dalla tabella \ref{tab:sAnCons} il bilancio economico è negativo per un importo pari a -175€.
			Queste variazioni rispetto al preventivo non avranno impatto sul costo finale in quanto le ore aggiuntive sono considerate di investimento.

%---------------------------------------------------------------------------------
	\newpage
	\subsubsection{Pl - Progettazione logica}
	\paragraph{Consuntivo orario}
%-----------------------------------------------------------------------------------
								\begin{table}[h] \begin{center} \begin{tabular}{llllllll}																						
								\toprule																						
									&	Re		&	Am		&	An		&	Pj		&	Pr		&	Ve		&	Tot	 \\ 	
								\midrule																		
								Brutesco	&	5		&	-		&	-		&	21		&	-		&	-		&	26	\\
								Damo		&	5		&	-		&	-		&	30		&	-		&	-		&	35	\\
								De Gaspari	&	-		&	-		&	-		&	6	(-4)	&	-		&	20	(+4)	&	26	\\
								Gottardo	&			&	-		&	-		&	10		&	-		&	16		&	26	\\
								Pasqualini	&	-		&	5		&	5		&	16		&	-		&	-		&	26	\\
								Petenazzi	&	-		&	4		&	5		&	17		&	-		&	-		&	26	\\
								Prete	&	-		&	-		&	-		&	14	(+4)	&	-		&	12	(-4)	&	26	\\
								\midrule																					
								Tot in ore	&	10		&	9		&	10		&	114(+0)		&	0		&	48(+0)		&	191	\\
								
								\bottomrule
								\end{tabular} \end{center} \caption{Prospetto a consuntivo orario per il periodo di																						
									Progettazione logica																						
									}\label{tab:orePl} \end{table}	
%-----------------------------------------------------------------------------------
	\paragraph{Consuntivo economico}
%-----------------------------------------------------------------------------------
								\begin{table}[H] \begin{center} \begin{tabular}{llllllll}																						
							\toprule	
								&	Re		&	Am		&	An		&	Pj		&	Pr		&	Ve		&	Tot	 \\ 	
							\midrule																		
							Tot in ore	&	10		&	9		&	10		&	114(+0)		&	0		&	48(+0)		&	191	\\
							Tot in €	&	 € 300 		 & 	 € 180 		 & 	 € 250 		 & 	 € 2.508 		 & 	 € -   		 & 	 € 720 		 & 	 € 3.958 	\\
							\bottomrule																						
							\end{tabular} \end{center} \caption{Prospetto a consuntivo economico per il periodo di																						
							Progettazione logica																						
							}\label{tab:sPl} \end{table}
	
	%----------------------------------------------------------------------------	
				
	\paragraph{Conclusioni}
	E' stato necessario un ricalibro delle ore a causa di problemi personali di un membro, come indicato dall'analisi dei rischi. Nonostante ciò, il consuntivo orario presenta gli stessi totali esposti nel preventivo orario di questo periodo.
	
	\newpage
	\subsection{Totale non rendicontato}
		\subsubsection{Consuntivo orario}
%-----------------------------------------------------------------
						\begin{table}[h] \begin{center} \begin{tabular}{llllllll}																						
						\toprule
									&	Re		&	Am		&	An		&	Pj		&	Pr		&	Ve		&	Tot	 \\ 	
						\midrule																					
						Brutesco	&	2		&	0		&	14	&	2		&	0		&	12		&	30	 \\ 	
						Damo		&	2		&	12		&	17	&	2		&	0		&	0		&	33	 \\ 	
						De Gaspari	&	0		&	0		&	14	&	2		&	0		&	14		&	30	 \\ 	
						Gottardo	&	16		&	0		&	12	&	2		&	0		&	2		&	32	 \\ 	
						Pasqualini	&	0		&	3		&	14	&	2		&	0		&	12		&	31	 \\ 	
						Petenazzi	&	9		&	3		&	19	&	2		&	0		&	0		&	33	 \\ 	
						Prete		&	0		&	13		&	16	&	2		&	0		&	2		&	33	 \\ 	
						\midrule																
						Tot in ore	&	29		&	31		&	106		&	14		&	0		&	42		&	222	 \\ 		
						\bottomrule																						
						\end{tabular} \end{center} \caption{Prospetto orario a consuntivo totale non rendicontato fino alla fine del periodo di																						
						Progettazione logica																						
						}\label{tab:oreNonRend} \end{table}							
%-----------------------------------------------------------------
		\subsubsection{Consuntivo economico}
%-----------------------------------------------------------------
						\begin{table}[H] \begin{center} \begin{tabular}{llllllll}
						\toprule
							&	\textbf{Re}	&	\textbf{Am}	&	\textbf{At}	&	\textbf{Pj}	&	\textbf{Pr}	&	\textbf{Ve}	&	\textbf{Tot}\\
			
						\midrule
						Tot in ore	&	29		&	31		&	106	&	14		&	0		&	42		&	222	 \\ 	
																												
						Tot in €	&	 € 870,00 		 & 	 €  620,00 		 & 	 € 2.650,00 		 & 	 €  308,00 		 & 	 €           -   		 & 	 €  630,00 		 & 	 €    5.078 	 \\ 
						\bottomrule
						\end{tabular} \end{center} \caption{Prospetto economico a consuntivo totale non rendicontato fino alla fine del periodo di																						
												Progettazione logica				
						}\label{tab:s_TotaleNonRendicontato} \end{table}
%-----------------------------------------------------------------
	
	\newpage
	\subsection{Totale complessivo}
		\subsubsection{Consuntivo orario}

%-----------------------------------------------------------------------------------------																					
						\begin{table}[h] \begin{center} \begin{tabular}{llllllll}																					
						\toprule																					
							&	Re		&	Am		&	An		&	Pj		&	Pr		&	Ve		&	Tot	\\
						\midrule																					
						Brutesco	&	7		&	0		&	14		&	23		&	0		&	12		&	56	\\
						Damo		&	7		&	12		&	17		&	32		&	0		&	0		&	68	\\
						De Gaspari	&	0		&	0		&	14		&	8		&	0		&	34		&	56	\\
						Gottardo	&	16		&	0		&	12		&	12		&	0		&	18		&	58	\\
						Pasqualini	&	0		&	8		&	19		&	18		&	0		&	12		&	57	\\
						Petenazzi	&	9		&	7		&	24		&	19		&	0		&	0		&	59	\\
						Prete		&	0		&	13		&	16		&	16		&	0		&	14		&	59	\\
						\midrule																					
						Tot in ore	&	39		&	40		&	116		&	128		&	0		&	90		&	413	\\
						\bottomrule																					
						\end{tabular} \end{center} \caption{Prospetto orario a consuntivo totale complessivo fino alla fine del periodo di																						
						Progettazione logica																						
						}\label{tab:h_		Pl			} \end{table}
%-----------------------------------------------------------------------------------------																					
		\subsubsection{Consuntivo economico}
%-----------------------------------------------------------------
						\begin{table}[H] \begin{center} \begin{tabular}{llllllll}
						\toprule
							&	\textbf{Re}	&	\textbf{Am}	&	\textbf{At}	&	\textbf{Pj}	&	\textbf{Pr}	&	\textbf{Ve}	&	\textbf{Tot}\\
						\midrule																					
						Tot in ore	&	39		&	40		&	116		&	128		&	0		&	90		&	413	\\
						Tot in €	&	 € 1.170 		 & 	 € 800 		 & 	 € 2.900 		 & 	 € 2.816 		 & 	 € -   		 & 	 € 1.350 		 & 	 € 9.036 	\\
						\bottomrule			
						\end{tabular} \end{center} \caption{Prospetto economico a consuntivo totale non rendicontato fino alla fine il periodo di																						
												Progettazione logica				
						}\label{tab:s_TotaleNonRendicontato} \end{table}
%-----------------------------------------------------------------
	
	\newpage
	\subsection{Totale rendicontato}
		\subsubsection{Consuntivo orario}
		Il consuntivo orario totale rendicontato fino alla fine del periodo di Progettazione logica coincide con il consuntivo orario del periodo di Progettazione logica presentato in tabella \ref{tab:orePl}.
		\subsubsection{Consuntivo economico}
		Il consuntivo economico totale rendicontato fino alla fine del periodo di Progettazione logica conincide con il consuntivo economico del periodo di Progettazione logica in tabella \ref{tab:sPl}.
	\subsection{Considerazioni finali sul consuntivo}
	\label{consid_cons_pl}
	Il totale rendicontato non presenta variazioni orarie o economiche rispetto al preventivo. I bilanci (orario ed economico) si concludono quindi in pareggio.\\
	Le 7 ore non rendicontate in eccedenza rispetto al preventivo nel primo periodo di Analisi (tabella \ref{tab:oreAnalisiCons}), sono state recuperate; infatti le ore impegate degli analisti (uscenti dal progetto con la fine di questo periodo) sono state 106, uguali alle ore degli analisti che erano state preventivate (tabella \ref{tab:h_TotaleNonRendicontato}).
	\\ Fino alla fine del periodo di Progettazione logica sono state impiegate 222 ore (tabella \ref{tab:oreNonRend}) non rendicontate \textit{(di cui 175 per il periodo di Analisi)} su 259 (tabella \ref{tab:h_TotaleNonRendicontato}) ore non rendicontate preventivate per l'intero sviluppo del progetto.
	