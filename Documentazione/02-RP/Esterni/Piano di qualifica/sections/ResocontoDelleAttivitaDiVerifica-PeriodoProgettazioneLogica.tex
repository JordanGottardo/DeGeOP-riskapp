\newpage

\renewcommand\VRule[1][\arrayrulewidth]{\color{black} \vrule width 2pt}

\section{Resoconto delle attività di verifica - Periodo Progettazione Logica}
\subsection{Verifica dei processi}

Segue una tabella riassuntiva riguardante le metriche di processo. Le righe che riportano una serie di trattini orizzontali sotto la voce "Processo" sono relative all'intero \glo{Periodo}{periodo} e non al singolo processo.
Per avere informazioni dettagliate sugli scopi dei processi e sulle attività che li compongono, consultare le \ndpvdue.


Per avere un resoconto testuale degli obiettivi in tabella, fare click sul nome dell'obiettivo.


Per una descrizione delle metriche in tabella, fare click sul nome della metrica.
\begin{table}[H]
\centering
\footnotesize\setlength{\tabcolsep}{7pt}
\small
\resizebox{\textwidth}{!}{
\rowcolors{2}{white}{white}
\begin{tabular}{c | c | c | c | c}
\hline
\textbf{Processo}            & \textbf{Obiettivo}        & \textbf{Metrica}                & \textbf{Valore}                            & \textbf{Giudizio}                           \\ \hline
Sviluppo            & \nameref{R2MC}    & \hyperref[MMC]{LCMM}   & \textcolor{LimeGreen}{2}          & \textcolor{LimeGreen}{Accettabile} \\
Documentazione         & \nameref{R2MC}    & \hyperref[MMC]{LCMM}   & \textcolor{LimeGreen}{3}          & \textcolor{LimeGreen}{Accettabile} \\
Verifica            & \nameref{R2MC}    & \hyperref[MMC]{LCMM}   & \textcolor{LimeGreen}{3}          & \textcolor{LimeGreen}{Accettabile} \\
Gestione dei processi     & \nameref{R2MC}    & \hyperref[MMC]{LCMM}   & \textcolor{LimeGreen}{3}          & \textcolor{LimeGreen}{Accettabile} \\
Gestione delle infrastrutture & \nameref{R2MC}    & \hyperref[MMC]{LCMM}   & \textcolor{LimeGreen}{2}          & \textcolor{LimeGreen}{Accettabile} \\
Gestione della configurazione & \nameref{R2MC}    & \hyperref[MMC]{LCMM}   & \textcolor{LimeGreen}{3}          & \textcolor{LimeGreen}{Accettabile} \\
Apprendimento         & \nameref{R2MC}    & \hyperref[MMC]{LCMM}   & \textcolor{LimeGreen}{2}          & \textcolor{LimeGreen}{Accettabile} \\
Documentazione         & \nameref{R2IND}    & \hyperref[MMC]{RDPO}   & \textcolor{LimeGreen}{14}          & \textcolor{LimeGreen}{Accettabile} \\
Documentazione         & \nameref{R2QDT}    & \hyperref[MMC]{NCR}   & \textcolor{ForestGreen}{0}          & \textcolor{ForestGreen}{Ottimale} \\
Documentazione         & \nameref{R2QDI}    & \hyperref[MMC]{RV}   & \textcolor{LimeGreen}{722}          & \textcolor{LimeGreen}{Accettabile} \\
Documentazione            & \nameref{R2TDM}    & \hyperref[MMC]{PTM}   & \textcolor{ForestGreen}{100\%}          & \textcolor{ForestGreen}{Ottimale} \\

Sviluppo            & \nameref{R2ASP}    & \hyperref[MMC]{UCSP}   & \textcolor{ForestGreen}{0}          & \textcolor{ForestGreen}{Ottimale} \\
Sviluppo            & \nameref{R2CRO}    & \hyperref[MMC]{PROC}   & \textcolor{ForestGreen}{100\%}          & \textcolor{ForestGreen}{Ottimale} \\
Sviluppo            & \nameref{R2BGDA}    & \hyperref[MMC]{GA}   & \textcolor{Red}{26}          & \textcolor{Red}{Non accettabile} \\
Sviluppo            & \nameref{R2AGDU}    & \hyperref[MMC]{GU}   & \textcolor{Red}{0}          & \textcolor{Red}{Non accettabile} \\



- - - - - - -         & \nameref{R2RDP}   & \hyperref[MRDP]{SV}    & \textcolor{LimeGreen}{4 giorni}   & \textcolor{LimeGreen}{Accettabile} \\
- - - - - - -         & \nameref{R2RDB}   & \hyperref[MRDB]{CV}    & \textcolor{ForestGreen}{0\%}      & \textcolor{ForestGreen}{Ottimale}  \\
- - - - - - -         & \nameref{R2CDADR} & \hyperref[MCDADR]{RNP} & \textcolor{ForestGreen}{0 rischi} & \textcolor{ForestGreen}{Ottimale}  \\ \hline
\end{tabular}}
\caption{Resoconto metriche di processo}
\label{tab:resoconto_metriche_processo}
\end{table}

\subsubsection{Considerazioni finali}
\paragraph{Miglioramento costante}
\label{R2MC}
Il livello \glo{CMM}{CMM} dei processi di Sviluppo, Gestione delle infrastrutture e Apprendimento in questo periodo è pari a 2.
Per quanto riguarda il processo di Sviluppo, il livello CMM è rimasto invariato rispetto al periodo precedente in quanto è stata introdotta l'attività di progettazione ad alto livello. Essendo un'attività mai svolta prima dai membri del \glo{Gruppo}{team}, la disciplina pur essendo standardizzata non è ancora rigorosa.
I processi di Gestione delle infrastrutture e di Apprendimento, non risultano ancora completamente ripetibili.
Il livello CMM dei processi di Documentazione, Verifica, Gestione dei processi e Gestione della configurazione in questo periodo è pari a 3.
La standardizzazione dei processi sopracitati risulta maggiore a quella rilevata nel periodo precedente, con una disciplina rigorosa.
Tutti i processi non sono standardizzati ad un livello tale da raggiungere il livello 3 della scala. Inoltre, la disciplina non è ancora molto rigorosa. L'obiettivo per i prossimi periodi è migliorare tale livello.

\paragraph{Rispetto della pianificazione}
\label{R2RDP}
Il ritardo riscontrato nel periodo di Progettazione Logica è pari a 4 giorni. Il ritardo rilevato si pone ai limiti della soglia di accettabilità. Il team è ancora in grado di rispettare la scadenza ma l'obiettivo è cercare di evitare ritardi aggiuntivi nei periodi successivi.

\paragraph{Rispetto del budget}
\label{R2RDB}
Non sono state riscontrate spese aggiuntive. La metrica assume quindi un valore ottimale.

\paragraph{Completezza dell'analisi dei rischi}
\label{R2CDADR}
Nel periodo di Progettazione Logica non sono sorti rischi non preventivati, pertanto la metrica assume un valore ottimale.

\paragraph{Impegno nella documentazione}
\label{R2IND}
Nel periodo di Progettazione Logica la metrica assume un valore accettabile, nonostante si possa ancora migliorare la produttività.

\paragraph{Qualità del template}
\label{R2QDT}
Nel periodo di Progettazione Logica la metrica assume un valore ottimale. Non sono stati richiesti comandi aggiuntivi, il che denota la presenza di un template in grado di soddisfare le necessità.

\paragraph{Qualità delle immagini}
\label{R2QDI}
Nel periodo di Progettazione Logica la metrica assume un valore accettabile. La qualità dell'immagine risulta essere sufficiente per una chiara visualizzazione. Il team si impegna comunque a migliorare ulteriormente la qualità delle immagini.

\paragraph{Tracciamento delle modifiche}
\label{R2TDM}
Nel periodo di Progettazione Logica la metrica assume un valore ottimale. Tutte le modifiche effettuate ai documenti sono state tracciate nell'apposito registro.

\paragraph{Assegnazione scenari principali}
\label{R2ASP}
Nel periodo di Progettazione Logica la metrica assume un valore ottimale. Tutti gli use case hanno uno scenario.

\paragraph{Copertura requisiti obbligatori}
\label{R2CRO}
Nel periodo di Progettazione Logica la metrica assume un valore ottimale. Tutti i requisti obbligatori sono stati assegnati alle componenti progettate.

\paragraph{Basso grado di accoppiamento}
\label{R2BGDA}
La metrica assume un valore non accettabile. La componente FactorySidebarPkg risulta avere un grado di accoppiamento pari a 26. Il team ha tuttavia ritenuto necessario mantenere tale componente. Il grado di accoppiamento così elevato è dato dal fatto che la famiglia di componenti che la factory deve gestire è numerosa.

\paragraph{Alto grado di utilità}
\label{R2AGDU}
La metrica assume un valore non accettabile. La componente CallManagerPkg risulta avere un grado di utilità 0. Il team ha tuttavia ritenuto necessario mantenere tale componente. Tale grado di utilità è dovuto al fatto che CallManagerPkg si sottoscrive allo store per mantenere aggiornato il server con i dati contenuti nel primo. Questo tipo di interazione non genera una dipendenza entrante in CallManagerPkg.

\subsection{Verifica dei prodotti}
\subsubsection{Verifica dei documenti}
	I documenti sono stati analizzati principalmente tramite \glo{Walkthrough}{walkthrough} data la scarsa esperienza dei verificatori. Gli errori più ricorrenti sono stati annotati e serviranno a creare una lista per le successiva attività di verifica, da effettuare utilizzando \glo{Inspection}{inspection}.		
	
	Seguono tabelle riassuntive riguardante le metriche relative ai documenti. 
	
	Per una descrizione delle metriche in tabella, fare click sul nome della metrica.
	
	\paragraph{Leggibilità e comprensibilità}
		\begin{table}[H]
			\centering
			\small
			\rowcolors{2}{white}{white}
			\begin{tabular}{c | c | c | c}
				\hline
				\textbf{Documento} & \textbf{Metrica}    & \textbf{Valore} & \textbf{Giudizio} \\ \hline
					\pdpvdue        & \hyperref[MLEC]{IG} & \textcolor{ForestGreen}{61} & \textcolor{ForestGreen}{Ottimale} \\
					\pdqvdue        & \hyperref[MLEC]{IG} & \textcolor{LimeGreen}{57} & \textcolor{LimeGreen}{Accettabile} \\
					\ndpvdue        & \hyperref[MLEC]{IG} & \textcolor{ForestGreen}{63} & \textcolor{ForestGreen}{Ottimale} \\
					\adrvdue        & \hyperref[MLEC]{IG}  & \textcolor{LimeGreen}{48} & \textcolor{LimeGreen}{Accettabile} \\
					\stvuno		& \hyperref[MLEC]{IG}  & \textcolor{LimeGreen}{51} & \textcolor{LimeGreen}{Accettabile} \\
					\glvdue        & \hyperref[MLEC]{IG} & \textcolor{LimeGreen}{56} & \textcolor{LimeGreen}{Accettabile} \\
					\vcinquei       & \hyperref[MLEC]{IG}& \textcolor{ForestGreen}{62} & \textcolor{ForestGreen}{Ottimale} \\
					\vseii       & \hyperref[MLEC]{IG} &  \textcolor{ForestGreen}{66} & \textcolor{ForestGreen}{Ottimale} \\
					\vtree       & \hyperref[MLEC]{IG}& \textcolor{ForestGreen}{66} & \textcolor{ForestGreen}{Ottimale} \\
			\end{tabular}
			\caption{Resoconto leggibilità e comprensibilità}
			\label{tab_resoconto_leggibilità_e_comprensibilità2}
		\end{table}
	
		\subparagraph{Considerazioni finali}
		Tutti i documenti presentano un \glo{Indice Gulpease}{indice Gulpease} ad un livello almeno accettabile; ciò dovrebbe garantire una lettura non particolarmente difficoltosa da parte di soggetti con almeno licenza superiore.
		Il documento che assume il valore più basso è l'\adrvdue. Questo è dovuto al fatto che esso è un documento particolarmente tecnico e i contenuti sono esposti sotto forma di tabelle.	
	\paragraph{Adesione alle norme interne}
		\begin{table}[H]
			\centering
			\small
			\rowcolors{2}{white}{white}
			\begin{tabular}{c | c | c | c}
				\hline
				\textbf{Documento} & \textbf{Metrica} & \textbf{Valore} & \textbf{Giudizio} \\
				\hline
					\pdpvdue     & \hyperref[MLEC]{ENNC}& \textcolor{ForestGreen}{0} & \textcolor{ForestGreen}{Ottimale} \\
					\pdqvdue     & \hyperref[MLEC]{ENNC}& \textcolor{ForestGreen}{0} & \textcolor{ForestGreen}{Ottimale} \\
					\ndpvdue     & \hyperref[MLEC]{ENNC} & \textcolor{ForestGreen}{0} & \textcolor{ForestGreen}{Ottimale} \\
					\adrvdue     & \hyperref[MLEC]{ENNC} & \textcolor{ForestGreen}{0} & \textcolor{ForestGreen}{Ottimale} \\
					\stvuno		& \hyperref[MLEC]{ENNC} & \textcolor{ForestGreen}{0} & \textcolor{ForestGreen}{Ottimale} \\
					\glvdue     & \hyperref[MLEC]{ENNC} & \textcolor{ForestGreen}{0} & \textcolor{ForestGreen}{Ottimale} \\
					\vcinquei       & \hyperref[MLEC]{ENNC} & \textcolor{ForestGreen}{0} & \textcolor{ForestGreen}{Ottimale} \\
					\vseii       & \hyperref[MLEC]{ENNC}& \textcolor{ForestGreen}{0} & \textcolor{ForestGreen}{Ottimale} \\
					\vtree       & \hyperref[MLEC]{ENNC} & \textcolor{ForestGreen}{0} & \textcolor{ForestGreen}{Ottimale} \\
				\hline
			\end{tabular}
			\caption{Resoconto adesione alle norme interne}
			\label{tab_resoconto_adesione_alle_norme_interne2}
		\end{table}
	
		\subparagraph{Considerazioni finali}
			Per tutti i documenti non risultano errori residui che violino le norme interne, pertanto le metriche hanno un valore ottimale.
		
		
\paragraph{Correttezza ortografica}
	\begin{table}[H]
		\centering
		\small
		\rowcolors{2}{white}{white}
		\begin{tabular}{c | c | c | c}
			\hline
			\textbf{Documento} & \textbf{Metrica} & \textbf{Valore} & \textbf{Giudizio} \\
			\hline
				\pdpvdue  &     \hyperref[MCO]{EONC} & \textcolor{ForestGreen}{0} & \textcolor{ForestGreen}{Ottimale} \\
				\pdqvdue  &      \hyperref[MCO]{EONC} & \textcolor{ForestGreen}{0} & \textcolor{ForestGreen}{Ottimale} \\
				\ndpvdue   &   \hyperref[MCO]{EONC} & \textcolor{ForestGreen}{0} & \textcolor{ForestGreen}{Ottimale} \\
				\adrvdue   &   \hyperref[MCO]{EONC} & \textcolor{ForestGreen}{0} & \textcolor{ForestGreen}{Ottimale} \\
				\stvuno &	\hyperref[MCO]{EONC} & \textcolor{ForestGreen}{0} & \textcolor{ForestGreen}{Ottimale} \\
				\glvdue &     \hyperref[MCO]{EONC} & \textcolor{ForestGreen}{0} & \textcolor{ForestGreen}{Ottimale} \\
				\vcinquei&      \hyperref[MCO]{EONC} & \textcolor{ForestGreen}{0} & \textcolor{ForestGreen}{Ottimale} \\
				\vseii   &  \hyperref[MCO]{EONC} & \textcolor{ForestGreen}{0} & \textcolor{ForestGreen}{Ottimale} \\
				\vtree   &  \hyperref[MCO]{EONC} & \textcolor{ForestGreen}{0} & \textcolor{ForestGreen}{Ottimale} \\
			\hline
		\end{tabular}
		\caption{Resoconto correttezza ortografica}
		\label{tab_resoconto_correttezza_ortografica}
	\end{table}
	
	\subparagraph{Considerazioni finali}
	Dopo l'analisi automatica dei correttori ortografici e quella mediante walkthrough da parte dei \verificatori{} non sono stati rilevati ulteriori errori che violano le norme interne, pertanto le metriche assumono un valore ottimale.
	
\paragraph{Correttezza concettuale}
\begin{table}[H]
	\centering
	\small
	\rowcolors{2}{white}{white}
	\begin{tabular}{c | c | c | c}
		\hline
		\textbf{Documento} & \textbf{Metrica} & \textbf{Valore} & \textbf{Giudizio} \\
		\hline
			\pdpvdue        &\hyperref[MCC]{ECNC} & \textcolor{ForestGreen}{0} & \textcolor{ForestGreen}{Ottimale} \\
			\pdqvdue        &\hyperref[MCC]{ECNC} & \textcolor{ForestGreen}{0} & \textcolor{ForestGreen}{Ottimale} \\
			\ndpvdue        &\hyperref[MCC]{ECNC} & \textcolor{ForestGreen}{0} & \textcolor{ForestGreen}{Ottimale} \\
			\adrvdue        &\hyperref[MCC]{ECNC} & \textcolor{ForestGreen}{0} & \textcolor{ForestGreen}{Ottimale} \\
			\stvuno		& \hyperref[MCC]{ECNC} & \textcolor{ForestGreen}{0} & \textcolor{ForestGreen}{Ottimale} \\
			\glvdue        & \hyperref[MCC]{ECNC} & \textcolor{ForestGreen}{0} & \textcolor{ForestGreen}{Ottimale} \\
			\vcinquei       &\hyperref[MCC]{ECNC} & \textcolor{ForestGreen}{0} & \textcolor{ForestGreen}{Ottimale} \\
			\vseii       & \hyperref[MCC]{ECNC} & \textcolor{ForestGreen}{0} & \textcolor{ForestGreen}{Ottimale} \\
			\vtree       & \hyperref[MCC]{ECNC} & \textcolor{ForestGreen}{0} & \textcolor{ForestGreen}{Ottimale} \\
		\hline
	\end{tabular}
	\caption{Resoconto correttezza concettuale}
	\label{tab_resoconto_correttezza_concettuale}
\end{table}

\subparagraph{Considerazioni finali}
	Per tutti i documenti non sono stati rilevati errori concettuali non corretti, pertanto le metriche assumono un valore ottimale.
	
\paragraph{Basso livello di annidamento dell'indice}
\begin{table}[H]
	\centering
	\small
	\rowcolors{2}{white}{white}
	\begin{tabular}{c | c | c | c}
		\hline
		\textbf{Documento} & \textbf{Metrica} & \textbf{Valore} & \textbf{Giudizio} \\
		\hline
		\pdpvdue        & \hyperref[MLEC]{LA} & \textcolor{LimeGreen}{4} & \textcolor{LimeGreen}{Accettabile} \\
		\pdqvdue        & \hyperref[MLEC]{LA} & \textcolor{LimeGreen}{4} & \textcolor{LimeGreen}{Accettabile} \\
		\ndpvdue        & \hyperref[MLEC]{LA} & \textcolor{LimeGreen}{5} & \textcolor{LimeGreen}{Accettabile} \\
		\adrvdue        & \hyperref[MLEC]{LA}& \textcolor{ForestGreen}{2} & \textcolor{ForestGreen}{Ottimale} \\
		\stvuno		& \hyperref[MLEC]{LA} & \textcolor{LimeGreen}{5} & \textcolor{LimeGreen}{Accettabile} \\
		\glvdue        & \hyperref[MLEC]{LA} & \textcolor{ForestGreen}{1} & \textcolor{ForestGreen}{Ottimale} \\
		\vcinquei       &\hyperref[MLEC]{LA} & \textcolor{ForestGreen}{1} & \textcolor{ForestGreen}{Ottimale} \\
		\vseii       & \hyperref[MLEC]{LA} & \textcolor{ForestGreen}{1} & \textcolor{ForestGreen}{Ottimale} \\
		\vtree       & \hyperref[MLEC]{LA} & \textcolor{ForestGreen}{1} & \textcolor{ForestGreen}{Ottimale} \\
		\hline
	\end{tabular}
	\caption{Resoconto basso livello di annidamento dell'indice}
	\label{tab_resoconto_basso_livello_di_annidamento_indice}
\end{table}

\subparagraph{Considerazioni finali}
Per tutti i documenti il livello di annidamento dell'indice risulta accettabile. In molti casi è stata preferita una struttura tabellare come alternativa all'eccessivo annidamento.