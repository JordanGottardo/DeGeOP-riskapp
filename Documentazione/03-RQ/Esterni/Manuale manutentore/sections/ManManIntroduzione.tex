\newpage

\section{Introduzione}
	\subsection{Scopo del documento}
	Il presente documento rappresenta il Manuale del Manutentore per il progetto \progetto{} sviluppato dal gruppo \zephyrus{} per il proponente \riskapp. 
	All'interno di esso vengono descritte:
	\begin{itemize}
		\item le tecnologie utilizzate per lo sviluppo;
		\item gli strumenti utilizzati e consigliati;
		\item l'architettura del software con i relativi componenti;
		\item le funzionalità presenti.
	\end{itemize}
	Tutto ciò viene descritto per aiutare lo sviluppatore a comprendere a fondo l'applicazione per eventualmente apportarvi delle modifiche o estensioni, assumendo e richiedendo che lo sviluppatore conosca già le tecnologie utilizzate nel progetto e brevemente illustrate nella prossima sezione.
	%
	\subsection{Scopo del prodotto}
	\introScopo
	%
	\subsection{Glossario}
	Allo scopo di rendere più semplice e chiara la comprensione del manuale, nel presente documento viene incluso un breve glossario.
	Per evidenziare un termine presente in tale documento, esso verrà marcato con il pedice \textsubscript{G}. Solo la prima occorrenza del termine in ogni sezione sarà marcata per non appesantire la lettura del documento.
	Tutti i termini del glossario evidenziati sono link ipertestuali al termine corrispettivo presente nel glossario.
	%
	\subsection{Riferimenti}
	%
%		\subsubsection{Riferimenti normativi}
%		\begin{itemize}
%			\item \ndpv;
%			\item \textbf{\glo{Capitolato}{capitolato} d'appalto C3:} \href{http://www.math.unipd.it/~tullio/IS-1/2016/Progetto/C3.pdf}{http://www.math.unipd.it/~tullio/IS-1/2016/Progetto/C3.pdf};
%		\end{itemize}
		%
		\subsubsection{Riferimenti informativi}
		\begin{itemize}
			\item \textbf{Introduzione a CSS3: }
			\href{https://www.w3schools.com/css/css3_intro.asp}{https://www.w3schools.com/css/css3_intro.asp};
			\item \textbf{introduzione a HTML5:}
			\href{https://www.w3schools.com/html/html5_intro.asp}{https://www.w3schools.com/html/html5_intro.asp};
			\item \textbf{guida a ES6:}
			\href{http://es6-features.org/\#Constants}{http://es6-features.org/\#Constants};
			\item \textbf{introduzione a JSON:}
			\href{https://www.w3schools.com/js/js_json_intro.asp}{https://www.w3schools.com/js/js_json_intro.asp};
			\item \textbf{introduzione a JSX:}
			\href{https://facebook.github.io/react/docs/introducing-jsx.html}{https://facebook.github.io/react/docs/introducing-jsx.html};
			\item \textbf{Node.js:}
			\href{https://nodejs.org/it/}{https://nodejs.org/it/};
			\item \textbf{guida a OpenLayers:}
			\href{https://openlayersbook.github.io/}{https://openlayersbook.github.io/};
			\item \textbf{guida a Open Street Map:}
			\href{http://wiki.openstreetmap.org/wiki/Beginners\%27_guide}{http://wiki.openstreetmap.org/wiki/Beginners\%27_guide};
			\item \textbf{guida a React:}
			\href{https://facebook.github.io/react/}{https://facebook.github.io/react/};
			\item \textbf{React Color:}
			\href{https://casesandberg.github.io/react-color/}{https://casesandberg.github.io/react-color/};
			\item \textbf{React-Redux:}
			\href{https://github.com/reactjs/react-redux}{https://github.com/reactjs/react-redux};
			\item \textbf{React Toolbox:}
			\href{http://react-toolbox.com/#/}{http://react-toolbox.com/\#/};
			\item \textbf{Redux:}
			\href{http://redux.js.org/docs/basics/}{http://redux.js.org/docs/basics/};
			\item \textbf{SVG tutorial:}
			\href{https://www.w3schools.com/graphics/svg_intro.asp}{https://www.w3schools.com/graphics/svg_intro.asp}
		\end{itemize}
	
