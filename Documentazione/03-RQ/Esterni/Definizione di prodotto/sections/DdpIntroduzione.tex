\newpage
\section{Introduzione}
	\subsection {Scopo del documento}
	Lo scopo del documento è definire la progettazione sia ad alto livello e sia di dettaglio del prodotto DeGeOP. 
	\\Si precisa che viene riportato solo ed esclusivamente quanto verrà sviluppato effettivamente (tecnologie, componenti, classi, metodi, campi dati, tracciamento, design pattern).
	
	\subsection {Scopo del prodotto}
	\introScopo
	\subsection {Glossario}
	\introGlossario
	\subsection {Riferimenti}
	\subsubsection{Riferimenti normativi}
	\begin{itemize}
		\item \ndpv.
	\end{itemize}
	\subsubsection{Riferimenti informativi}
	\begin{itemize}
		\item \textbf{\glo{Capitolato}{capitolato} d'appalto C3:} \progetto: A Designer and Geo-localizer Web App for Organizational Plants. Reperibile all'indirizzo:\\ \url{http://www.math.unipd.it/~tullio/IS-1/2016/Progetto/C3.pdf} ;
		\item \adrv;
		\item \textbf{Guide to the Software Engineering Body of Knowledge: IEEE Computer Society. Software Engineering Coordinating Committee (Versione 2004):}
		\begin{itemize}
			\item \textbf{Chapter 3:} Software Design;
		\end{itemize}
		\item \textbf{slide del corso di Ingegneria del Software}\\
		\url{http://www.math.unipd.it/~tullio/IS-1/2016/};
		\item \textbf{Design Patterns} - Elementi per il riuso di software a oggetti - Gamma, Helm, Johnson, Vlissides;
		\item \textbf{Documentazione di Redux. Reperibile all'indirizzo: \\}\url{http://redux.js.org/};
		\item \textbf{Documentazione di React. Reperibile all'indirizzo: \\}\url{https://facebook.github.io/react/};
		\item \textbf{Documentazione di REST. Reperibile all'indirizzo: \\}\url{https://en.wikipedia.org/wiki/Representational_state_transfer}.
	\end{itemize}
