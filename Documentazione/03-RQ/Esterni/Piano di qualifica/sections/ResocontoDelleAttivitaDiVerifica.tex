\newpage

\section{Resoconto delle attività di verifica}
	Il resoconto delle attività di verifica è stato suddiviso per obiettivo. All'interno di ogni obiettivo è presente un resoconto per ogni periodo. Per avere una lista dei periodi, consultare il \pdpv.
	\subsection{Verifica dei processi}
		Seguono gli esiti delle verifiche riguardanti i processi. Per avere informazioni dettagliate sugli scopi dei processi e sulle attività che li compongono, consultare le \ndpv.
		\subsubsection{Tutti i processi}
			\paragraph{Miglioramento costante}
				\subparagraph{Periodo An}
		\begin{table}[H]
			\centering
			\small
			\rowcolors{2}{white}{white}
			\begin{tabular}{c | c | c | c}
				\hline
				          \textbf{Processo}                 & \textbf{Metrica}                & \textbf{Valore}                            & \textbf{Giudizio}                           \\ \hline
				          Fornitura              & \hyperref[MMC]{LCMM}   & \textcolor{LimeGreen}{2}          & \textcolor{LimeGreen}{Accettabile} \\
				          Sviluppo                & \hyperref[MMC]{LCMM}   & \textcolor{LimeGreen}{2}          & \textcolor{LimeGreen}{Accettabile} \\
				       Documentazione            & \hyperref[MMC]{LCMM}   & \textcolor{LimeGreen}{2}          & \textcolor{LimeGreen}{Accettabile} \\
						   Verifica              & \hyperref[MMC]{LCMM}   & \textcolor{LimeGreen}{2}          & \textcolor{LimeGreen}{Accettabile} \\
				Gestione delle infrastrutture    & \hyperref[MMC]{LCMM}   & \textcolor{LimeGreen}{2}          & \textcolor{LimeGreen}{Accettabile} \\
				    Gestione dei processi        & \hyperref[MMC]{LCMM}   & \textcolor{LimeGreen}{2}          & \textcolor{LimeGreen}{Accettabile} \\
				        Apprendimento             & \hyperref[MMC]{LCMM}   & \textcolor{LimeGreen}{2}          & \textcolor{LimeGreen}{Accettabile} \\
				    \hline
			\end{tabular}
			\caption{Resoconto miglioramento costante - periodo An}
			\label{tab:resoconto_obiettivo_miglioramento_costante_AN}
		\end{table}
	
		Il livello \glo{CMM}{CMM} dei processi di Fornitura, Sviluppo, Documentazione, Verifica, Gestione delle infrastrutture, Gestione di Processo e Apprendimento in questo periodo è pari a 2. Dopo lo stato iniziale, durato quasi fino a metà periodo, in cui i processi si trovavano in uno stato caotico, il rispetto delle \ndpvuno{} e l'adozione di strumenti automatici ha portato ad un guadagno di ripetibilità. Alcuni esempi di tali strumenti sono i correttori ortografici e lo script per il calcolo dell'indice di leggibilità per quanto riguarda i processi di Documentazione e Verifica e l'utilizzo di \glo{Trender}{Trender} per quanto riguarda l'attività di analisi dei requisiti del processo di Sviluppo.
	
	
		Tutti i processi non sono standardizzati ad un livello tale da raggiungere il livello 3 della scala. Inoltre, la disciplina non è ancora molto rigorosa. L'obiettivo per i prossimi periodi è migliorare tale livello.
	\subparagraph{Periodo Pl}
	
	
	\begin{table}[H]
		\centering


			\rowcolors{2}{white}{white}
			\begin{tabular}{c | c | c | c | c}
				\hline
				\textbf{Processo}          & \textbf{Metrica}                & \textbf{Valore}                            & \textbf{Giudizio}                           \\ \hline
				Sviluppo             & \hyperref[MMC]{LCMM}   & \textcolor{LimeGreen}{2}          & \textcolor{LimeGreen}{Accettabile} \\
				Documentazione       & \hyperref[MMC]{LCMM}   & \textcolor{LimeGreen}{3}          & \textcolor{LimeGreen}{Accettabile} \\
				Verifica            & \hyperref[MMC]{LCMM}   & \textcolor{LimeGreen}{3}          & \textcolor{LimeGreen}{Accettabile} \\
				Gestione dei processi       & \hyperref[MMC]{LCMM}   & \textcolor{LimeGreen}{3}          & \textcolor{LimeGreen}{Accettabile} \\
				Gestione delle infrastrutture  & \hyperref[MMC]{LCMM}   & \textcolor{LimeGreen}{2}          & \textcolor{LimeGreen}{Accettabile} \\
				Gestione della configurazione   & \hyperref[MMC]{LCMM}   & \textcolor{LimeGreen}{3}          & \textcolor{LimeGreen}{Accettabile} \\
				Apprendimento             & \hyperref[MMC]{LCMM}   & \textcolor{LimeGreen}{2}          & \textcolor{LimeGreen}{Accettabile} \\
				
				
\hline
		\end{tabular}
		\caption{Resoconto miglioramento costante - periodo Pl}
		\label{tab:resoconto_obiettivo_miglioramento_costante_PPL}
	\end{table}

Il livello \glo{CMM}{CMM} dei processi di Sviluppo, Gestione delle infrastrutture e Apprendimento in questo periodo è pari a 2.
Per quanto riguarda il processo di Sviluppo, il livello CMM è rimasto invariato rispetto al periodo precedente in quanto è stata introdotta l'attività di progettazione ad alto livello. Essendo un'attività mai svolta prima dai membri del \glo{Gruppo}{team}, la disciplina pur essendo standardizzata non è ancora rigorosa.
I processi di Gestione delle infrastrutture e di Apprendimento, non risultano ancora completamente ripetibili.\\
Il livello CMM dei processi di Documentazione, Verifica, Gestione dei processi e Gestione della configurazione in questo periodo è pari a 3.
La standardizzazione dei processi sopracitati risulta maggiore a quella rilevata nel periodo precedente, con una disciplina rigorosa.
L'obiettivo per i prossimi periodi è migliorare il livello CMM.

\subparagraph{Periodo PCV}


\begin{table}[H]
	\centering
	
	
	\rowcolors{2}{white}{white}
	\begin{tabular}{c | c | c | c | c}
		\hline
		\textbf{Processo}          & \textbf{Metrica}                & \textbf{Valore}                            & \textbf{Giudizio}                           \\ \hline
		Sviluppo             & \hyperref[MMC]{LCMM}   & \textcolor{Red}{1}          & \textcolor{Red}{Non accettabile} \\
		Documentazione       & \hyperref[MMC]{LCMM}   & \textcolor{LimeGreen}{3}          & \textcolor{LimeGreen}{Accettabile} \\
		Verifica            & \hyperref[MMC]{LCMM}   & \textcolor{LimeGreen}{3}          & \textcolor{LimeGreen}{Accettabile} \\
		Validazione & \hyperref[MMC]{LCMM}   & \textcolor{LimeGreen}{3}          & \textcolor{LimeGreen}{Accettabile} \\
		Gestione dei processi       & \hyperref[MMC]{LCMM}   & \textcolor{LimeGreen}{3}          & \textcolor{LimeGreen}{Accettabile} \\
		Gestione delle infrastrutture  & \hyperref[MMC]{LCMM}   & \textcolor{LimeGreen}{2}          & \textcolor{LimeGreen}{Accettabile} \\
		Gestione della configurazione   & \hyperref[MMC]{LCMM}   & \textcolor{LimeGreen}{3}          & \textcolor{LimeGreen}{Accettabile} \\
		Apprendimento             & \hyperref[MMC]{LCMM}   & \textcolor{LimeGreen}{2}          & \textcolor{LimeGreen}{Accettabile} \\
		
		
		\hline
	\end{tabular}
	\caption{Resoconto miglioramento costante - periodo PCV}
	\label{tab:resoconto_obiettivo_miglioramento_costante_PCV}
\end{table}

Il livello \glo{CMM}{CMM} dei processi di Gestione delle infrastrutture e Apprendimento in questo periodo è pari a 2.
Per quanto riguarda il processo di Sviluppo, il livello CMM è rimasto invariato rispetto al periodo precedente in quanto è stata introdotta l'attività di codifica. Essendo un'attività mai svolta prima dai membri del \glo{Gruppo}{team}, la disciplina pur essendo standardizzata non è ancora rigorosa.
I processi di Gestione delle infrastrutture e di Apprendimento, non risultano ancora completamente ripetibili.\\
Il livello CMM dei processi di Documentazione, Verifica, Gestione dei processi e Gestione della configurazione in questo periodo è pari a 3.\\
Il livello CMM del processo di Sviluppo è 1. Le norme di codifica sono estremamente superficiali e la loro applicazione è quasi nulla. I verificatori ritengono che il codice prodotto durante questo periodo sia di qualità inaccettabile. Sono già state prese decisioni per cercare di risolvere la situazione.\\
Tutti i processi non sono ancora ad un livello di ripetibilità tale da poter essere riadattati senza perdite sostanziali di qualità, pertanto nessuno raggiunge il livello 4.

\newpage

			\paragraph{Rispetto della pianificazione}
			Per una descrizione dell'obiettivo, consultare \nameref{ORDP}.
			\label{RRDP}
			
				\subparagraph{Periodo An}
					\begin{table}[H]
						\centering
						\rowcolors{2}{white}{white}
						\begin{tabular}{  c | c | c}
							\hline
							\textbf{Metrica} & \textbf{Valore} & \textbf{Giudizio} \\
							\hline
							\hyperref[MRDB]{SV}    & \textcolor{LimeGreen}{3 giorni}      & \textcolor{LimeGreen}{Accettabile}  \\\hline
						\end{tabular}
					\end{table}
				
					Il ritardo riscontrato nel periodo di Analisi è pari a 3 giorni. Dato che il ritardo è all'interno della soglia di accettabilità, il \glo{Gruppo}{team} è ancora in grado di rispettare la scadenza. L'obiettivo è cercare di evitare ritardi nei periodi successivi.
			
			
				\subparagraph{Periodo Pl}
				
				\begin{table}[H]
					\centering
					\rowcolors{2}{white}{white}
					\begin{tabular}{  c | c | c}
						\hline
						\textbf{Metrica} & \textbf{Valore} & \textbf{Giudizio} \\
						\hline
						\hyperref[MRDB]{SV}    & \textcolor{LimeGreen}{4 giorni}      & \textcolor{LimeGreen}{Accettabile}  \\\hline
					\end{tabular}
				\end{table}
			
					Il ritardo riscontrato nel periodo di Progettazione Logica è pari a 4 giorni. Il ritardo rilevato si pone ai limiti della soglia di accettabilità. Il team è ancora in grado di rispettare la scadenza ma l'obiettivo è cercare di evitare ritardi aggiuntivi nei periodi successivi.
					
					\subparagraph{Periodo PCV}
					
					\begin{table}[H]
						\centering
						\rowcolors{2}{white}{white}
						\begin{tabular}{  c | c | c}
							\hline
							\textbf{Metrica} & \textbf{Valore} & \textbf{Giudizio} \\
							\hline
							\hyperref[MRDB]{SV}    & \textcolor{Red}{10 giorni}      & \textcolor{Red}{Non accettabile}  \\\hline
						\end{tabular}
					\end{table}
					
					Il ritardo riscontrato nel periodo di Progettazione di dettaglio, Codifica e Validazione è pari a 10 giorni. La metrica assume un valore non accettabile.
					I motivi del ritardo sono dovuti principalmente ai problemi con le tecnologie riscontrati in questo periodo e alla scarsa applicazione delle norme di codifica.
					
					
					L'obiettivo per il prossimo periodo è cercare di recuperare il tempo perso, concentrandosi maggiormente sulle funzionalità obbligatorie e lasciando in secondo piano quelle opzionali.
					
					\newpage
			\paragraph{Rispetto del budget}
			\label{RRDB}
			Per una descrizione dell'obiettivo, consultare \nameref{ORDB}.
				\subparagraph{Periodo An}
					\begin{table}[H]
						\centering
												\rowcolors{2}{white}{white}
							\begin{tabular}{  c | c | c}
								\hline
									\textbf{Metrica} & \textbf{Valore} & \textbf{Giudizio} \\
								\hline
								    \hyperref[MRDB]{CV}    & \textcolor{ForestGreen}{0\%}      & \textcolor{ForestGreen}{Ottimale}  \\\hline
							\end{tabular}
					\end{table}
			
					Non sono state riscontrate spese aggiuntive. La metrica assume quindi un valore ottimale.
					
				\subparagraph{Periodo Pl}
					\begin{table}[H]
						\centering
						\rowcolors{2}{white}{white}
						\begin{tabular}{  c | c | c}
							\hline
							\textbf{Metrica} & \textbf{Valore} & \textbf{Giudizio} \\
							\hline
							\hyperref[MRDB]{CV}    & \textcolor{ForestGreen}{0\%}      & \textcolor{ForestGreen}{Ottimale}  \\\hline
						\end{tabular}
					\end{table}
%				
					Non sono state riscontrate spese aggiuntive. La metrica assume quindi un valore ottimale.
					
					\subparagraph{Periodo PCV}
					\begin{table}[H]
						\centering
						\rowcolors{2}{white}{white}
						\begin{tabular}{  c | c | c}
							\hline
							\textbf{Metrica} & \textbf{Valore} & \textbf{Giudizio} \\
							\hline
							\hyperref[MRDB]{CV}    & \textcolor{ForestGreen}{-1.34\%}      & \textcolor{ForestGreen}{Ottimale}  \\\hline
						\end{tabular}
					\end{table}
					%				
					In questo periodo è stato speso meno di quanto preventivato, quindi la metrica assume un valore ottimale.
					\newpage
					
			\paragraph{Completezza dell'analisi dei rischi}
			\label{RCDADR}
				Per una descrizione dell'obiettivo, consultare \nameref{OCDADR}.
			
				\subparagraph{Periodo An}
				\begin{table}[H]
					\centering
					\rowcolors{2}{white}{white}
					\begin{tabular}{  c | c | c}
						\hline
						\textbf{Metrica} & \textbf{Valore} & \textbf{Giudizio} \\
						\hline
						\hyperref[MRDB]{RNP}    & \textcolor{ForestGreen}{0 rischi}      & \textcolor{ForestGreen}{Ottimale}  \\\hline
					\end{tabular}
				\end{table}
					Dall'inizio del progetto non sono sorti rischi non preventivati, pertanto la metrica assume un valore ottimale.
				
				\subparagraph{Periodo Pl}
				
				\begin{table}[H]
					\centering
					\rowcolors{2}{white}{white}
					\begin{tabular}{  c | c | c}
						\hline
						\textbf{Metrica} & \textbf{Valore} & \textbf{Giudizio} \\
						\hline
						\hyperref[MRDB]{RNP}    & \textcolor{ForestGreen}{0 rischi}      & \textcolor{ForestGreen}{Ottimale}  \\\hline
					\end{tabular}
				\end{table}
					Nel periodo di Progettazione Logica non sono sorti rischi non preventivati, pertanto la metrica assume un valore ottimale.
					
					\subparagraph{Periodo PCV}
					
					\begin{table}[H]
						\centering
						\rowcolors{2}{white}{white}
						\begin{tabular}{  c | c | c}
							\hline
							\textbf{Metrica} & \textbf{Valore} & \textbf{Giudizio} \\
							\hline
							\hyperref[MRDB]{RNP}    & \textcolor{ForestGreen}{0 rischi}      & \textcolor{ForestGreen}{Ottimale}  \\\hline
						\end{tabular}
					\end{table}
					Nel periodo preso in esame non sono sorti rischi non preventivati, pertanto la metrica assume un valore ottimale.
					\newpage
					
		\subsubsection{Processo di documentazione}
			Alcuni obiettivi relativi al processo di documentazione non hanno un resoconto per il periodo di Analisi in quanto sono stati stabiliti alla fine di esso.
			
			\paragraph{Impegno nella documentazione}
				Per una descrizione dell'obiettivo, consultare \nameref{OIND}.
				\subparagraph{Periodo Pl}
				\begin{table}[H]
					\centering
					\rowcolors{2}{white}{white}
					\begin{tabular}{  c | c | c}
						\hline
						\textbf{Metrica} & \textbf{Valore} & \textbf{Giudizio} \\
						\hline
						 \hyperref[MMC]{RDPO}   & \textcolor{LimeGreen}{14}          & \textcolor{LimeGreen}{Accettabile} \\ \hline
					\end{tabular}
				\end{table}
					La metrica assume un valore accettabile; nonostante ciò si può ancora migliorare la produttività.
					
					\subparagraph{Periodo PCV}
					\begin{table}[H]
						\centering
						\rowcolors{2}{white}{white}
						\begin{tabular}{  c | c | c}
							\hline
							\textbf{Metrica} & \textbf{Valore} & \textbf{Giudizio} \\
							\hline
							\hyperref[MMC]{RDPO}   & \textcolor{LimeGreen}{18}          & \textcolor{LimeGreen}{Accettabile} \\ \hline
						\end{tabular}
					\end{table}
					La metrica assume un valore ottimale; la produttività dei documenti è aumentata rispetto al periodo sufficiente grazie alla maggiore esperienza dei membri del team.
					
					
					\newpage
			\paragraph{Qualità del template}
				Per una descrizione dell'obiettivo, consultare \nameref{OQDT}.
				\subparagraph{Periodo Pl}
				\begin{table}[H]
					\centering
					\rowcolors{2}{white}{white}
					\begin{tabular}{  c | c | c}
						\hline
						\textbf{Metrica} & \textbf{Valore} & \textbf{Giudizio} \\
						\hline
						\hyperref[MMC]{NCR}   & \textcolor{ForestGreen}{0}          & \textcolor{ForestGreen}{Ottimale} \\ \hline
					\end{tabular} 
				\end{table}
					La metrica assume un valore ottimale. Non sono stati richiesti comandi aggiuntivi nel periodo in esame, il che denota la presenza di un template in grado di soddisfare le necessità.
					
					
					\subparagraph{Periodo PCV}
					\begin{table}[H]
						\centering
						\rowcolors{2}{white}{white}
						\begin{tabular}{  c | c | c}
							\hline
							\textbf{Metrica} & \textbf{Valore} & \textbf{Giudizio} \\
							\hline
							\hyperref[MMC]{NCR}   & \textcolor{ForestGreen}{0}          & \textcolor{ForestGreen}{Ottimale} \\ \hline
						\end{tabular} 
					\end{table}
					La metrica assume un valore ottimale. Non sono stati richiesti comandi aggiuntivi nel periodo in esame, il che denota la presenza di un template in grado di soddisfare le necessità.
					
					\newpage
					
			\paragraph{Qualità delle immagini}
				Per una descrizione dell'obiettivo, consultare \nameref{OQDI}.
				\subparagraph{Periodo Pl}
				\begin{table}[H]
					\centering
					\rowcolors{2}{white}{white}
					\begin{tabular}{  c | c | c}
						\hline
						\textbf{Metrica} & \textbf{Valore} & \textbf{Giudizio} \\
						\hline
						 \hyperref[MMC]{RV}   & \textcolor{LimeGreen}{722}          & \textcolor{LimeGreen}{Accettabile} \\ \hline
					\end{tabular} 
				\end{table}
					La metrica assume un valore accettabile. La qualità dell'immagine risulta essere sufficiente per una chiara visualizzazione. Il team si impegna comunque a migliorare ulteriormente la qualità delle immagini.
				
				\subparagraph{Periodo PCV}
				\begin{table}[H]
					\centering
					\rowcolors{2}{white}{white}
					\begin{tabular}{  c | c | c}
						\hline
						\textbf{Metrica} & \textbf{Valore} & \textbf{Giudizio} \\
						\hline
						\hyperref[MMC]{RV}   & \textcolor{LimeGreen}{722}          & \textcolor{LimeGreen}{Accettabile} \\ \hline
					\end{tabular} 
				\end{table}
				La metrica assume un valore accettabile. La qualità dell'immagine risulta essere sufficiente per una chiara visualizzazione. Il team si impegna comunque a migliorare ulteriormente la qualità delle immagini.
				
				\newpage
				
			\paragraph{Tracciamento delle modifiche}
				Per una descrizione dell'obiettivo, consultare \nameref{OTDM}.
				\subparagraph{Periodo Pl} 
				\begin{table}[H]
					\centering
					\rowcolors{2}{white}{white}
					\begin{tabular}{  c | c | c}
						\hline
						\textbf{Metrica} & \textbf{Valore} & \textbf{Giudizio} \\
						\hline
						\hyperref[MMC]{PTM}   & \textcolor{ForestGreen}{100\%}          & \textcolor{ForestGreen}{Ottimale} \\ \hline
					\end{tabular} 
				\end{table}
			
			La metrica assume un valore ottimale. Tutte le modifiche effettuate ai documenti sono state tracciate nell'apposito registro.
			
			\subparagraph{Periodo PCV} 
			\begin{table}[H]
				\centering
				\rowcolors{2}{white}{white}
				\begin{tabular}{  c | c | c}
					\hline
					\textbf{Metrica} & \textbf{Valore} & \textbf{Giudizio} \\
					\hline
					\hyperref[MMC]{PTM}   & \textcolor{ForestGreen}{100\%}          & \textcolor{ForestGreen}{Ottimale} \\ \hline
				\end{tabular} 
			\end{table}
			
					La metrica assume un valore ottimale. Tutte le modifiche effettuate ai documenti sono state tracciate nell'apposito registro.
					\newpage
					
		\subsubsection{Processo di sviluppo}
			Alcuni obiettivi relativi al processo di documentazione non hanno un resoconto per il periodo di Analisi in quanto sono stati stabiliti alla fine di esso.
				

				
				\paragraph{Impegno nella codifica}
					Per una descrizione dell'obiettivo, consultare \nameref{OINC}.
					
					\subparagraph{Periodo PCV} 
					\begin{table}[H]
						\centering
						\rowcolors{2}{white}{white}
						\begin{tabular}{  c | c | c}
							\hline
							\textbf{Metrica} & \textbf{Valore} & \textbf{Giudizio} \\
							\hline
							\hyperref[MMC]{RCPO}   & \textcolor{LimeGreen}{4}          & \textcolor{LimeGreen}{Accettabile} \\ \hline
						\end{tabular} 
					\end{table}
				
				La metrica assume un valore accettabile. L'impegno nella codifica è relativamente basso in quanto tutti i programmatori sono inesperti. L'obiettivo per il prossimo periodo è di aumentare il livello di produttività nella codifica. 
				\newpage
				
				\paragraph{Assegnazione scenari principali}
					Per una descrizione dell'obiettivo, consultare \nameref{OASP}.
					\subparagraph{Periodo Pl}
					
					\begin{table}[H]
						\centering
						\rowcolors{2}{white}{white}
						\begin{tabular}{  c | c | c}
							\hline
							\textbf{Metrica} & \textbf{Valore} & \textbf{Giudizio} \\
							\hline
						\hyperref[MMC]{UCSP}   & \textcolor{ForestGreen}{0}          & \textcolor{ForestGreen}{Ottimale} \\ \hline
						\end{tabular} 
					\end{table}
				
					La metrica assume un valore ottimale. Tutti gli use case hanno uno scenario principale assegnato.
					
					\subparagraph{Periodo PCV}
					
					\begin{table}[H]
						\centering
						\rowcolors{2}{white}{white}
						\begin{tabular}{  c | c | c}
							\hline
							\textbf{Metrica} & \textbf{Valore} & \textbf{Giudizio} \\
							\hline
							\hyperref[MMC]{UCSP}   & \textcolor{ForestGreen}{0}          & \textcolor{ForestGreen}{Ottimale} \\ \hline
						\end{tabular} 
					\end{table}
					
					La metrica continua ad assumere un valore ottimale. Tutti gli use case hanno uno scenario principale assegnato.
				
				\newpage
				\paragraph{Copertura requisiti obbligatori}
					Per una descrizione dell'obiettivo, consultare \nameref{OCRO}.
					\subparagraph{Periodo Pl}
					
					\begin{table}[H]
						\centering
						\rowcolors{2}{white}{white}
						\begin{tabular}{  c | c | c}
							\hline
							\textbf{Metrica} & \textbf{Valore} & \textbf{Giudizio} \\
							\hline
							\hyperref[MMC]{PROC}   & \textcolor{ForestGreen}{100\%}          & \textcolor{ForestGreen}{Ottimale} \\ \hline
						\end{tabular} 
					\end{table}
				
				La metrica assume un valore ottimale. Tutti i requisiti obbligatori sono stati assegnati alle componenti progettate.
				
				\subparagraph{Periodo PCV}
				
				\begin{table}[H]
					\centering
					\rowcolors{2}{white}{white}
					\begin{tabular}{  c | c | c}
						\hline
						\textbf{Metrica} & \textbf{Valore} & \textbf{Giudizio} \\
						\hline
						\hyperref[MMC]{PROC}   & \textcolor{ForestGreen}{100\%}          & \textcolor{ForestGreen}{Ottimale} \\ \hline
					\end{tabular} 
				\end{table}
				
				La metrica continua ad assumere un valore ottimale. Tutti i requisiti obbligatori sono stati assegnati alle componenti progettate.
				
				\newpage
				\paragraph{Basso grado di accoppiamento}
					Per una descrizione dell'obiettivo, consultare \nameref{OBGDA}.
					\subparagraph{Periodo Pl}
					
					\begin{table}[H]
						\centering
						\rowcolors{2}{white}{white}
						\begin{tabular}{  c | c | c}
							\hline
							\textbf{Metrica} & \textbf{Valore} & \textbf{Giudizio} \\
							\hline
							 \hyperref[MMC]{GA}   & \textcolor{Red}{26}          & \textcolor{Red}{Non accettabile} \\ \hline
						\end{tabular} 
					\end{table}
				
				La metrica assume un valore non accettabile. La componente FactorySidebarPkg risulta avere un grado di accoppiamento pari a 26. Il team ha tuttavia ritenuto necessario mantenere tale componente. Il grado di accoppiamento così elevato è dato dal fatto che la famiglia di componenti che la factory deve gestire è numerosa.
				
					\subparagraph{Periodo PCV}
				
				\begin{table}[H]
					\centering
					\rowcolors{2}{white}{white}
					\begin{tabular}{  c | c | c}
						\hline
						\textbf{Metrica} & \textbf{Valore} & \textbf{Giudizio} \\
						\hline
						\hyperref[MMC]{GA}   & \textcolor{Red}{26}          & \textcolor{Red}{Non accettabile} \\ \hline
					\end{tabular} 
				\end{table}
				
				La metrica assume un valore non accettabile. Le motivazioni rimangono le stesse del periodo precedente.
				
				\newpage
				\paragraph{Alto grado di utilità}
					Per una descrizione dell'obiettivo, consultare \nameref{OAGDU}.
					\subparagraph{Periodo Pl}
					
					\begin{table}[H]
						\centering
						\rowcolors{2}{white}{white}
						\begin{tabular}{  c | c | c}
							\hline
							\textbf{Metrica} & \textbf{Valore} & \textbf{Giudizio} \\
							\hline
						 \hyperref[MMC]{GU}   & \textcolor{Red}{0}          & \textcolor{Red}{Non accettabile}  \\ \hline
						\end{tabular} 
					\end{table}
				
		La metrica assume un valore non accettabile. La componente CallManagerPkg risulta avere un
		grado di utilità pari a 0. Il team ha tuttavia ritenuto necessario mantenere tale componente. Tale
		grado di utilità è dovuto al fatto che CallManagerPkg si sottoscrive allo store per mantenere
		aggiornato il server con i dati contenuti nel primo. Questo tipo di interazione non genera una
		dipendenza entrante in CallManagerPkg.
				
	
				\subparagraph{Periodo PCV}
				
				\begin{table}[H]
					\centering
					\rowcolors{2}{white}{white}
					\begin{tabular}{  c | c | c}
						\hline
						\textbf{Metrica} & \textbf{Valore} & \textbf{Giudizio} \\
						\hline
						\hyperref[MMC]{GU}   & \textcolor{Red}{0}          & \textcolor{Red}{Non accettabile}  \\ \hline
					\end{tabular} 
				\end{table}
				
				La metrica assume un valore non accettabile. Le motivazioni rimangono le stesse del periodo precedente.
				
				\newpage
			\subsection{Verifica dei prodotti}
				\subsubsection{Verifica dei documenti}
					Nel periodo di Analisi, i documenti sono stati analizzati principalmente tramite \glo{Walkthrough}{walkthrough} data la scarsa esperienza dei verificatori. Gli errori più ricorrenti sono stati annotati e serviranno a creare una lista per le successiva attività di verifica, da effettuare utilizzando \glo{Inspection}{inspection}.		

					\paragraph{Leggibilità e comprensibilità}
						Per una descrizione dell'obiettivo, consultare \nameref{OLEC}.
						\subparagraph{Periodo An}
							\begin{table}[H]
								\centering
								\small
								\rowcolors{2}{white}{white}
								\begin{tabular}{c | c | c | c}
									\hline
									\textbf{Documento} & \textbf{Metrica}    & \textbf{Valore} & \textbf{Giudizio} \\ \hline
									      \pdpvuno        & \hyperref[MLEC]{IG} &  \textcolor{LimeGreen}{58}              & \textcolor{LimeGreen}{Accettabile} \\
									      \pdqvuno        & \hyperref[MLEC]{IG} &  \textcolor{LimeGreen}{58}               &  \textcolor{LimeGreen}{Accettabile} \\
									      \ndpvuno        & \hyperref[MLEC]{IG} &  \textcolor{ForestGreen}{61}               & \textcolor{ForestGreen}{Ottimale}\\
									      \sdfv        & \hyperref[MLEC]{IG} &  \textcolor{LimeGreen}{52}               &  \textcolor{LimeGreen}{Accettabile}\\
									      \adrvuno        & \hyperref[MLEC]{IG} &  \textcolor{LimeGreen}{45}               &  \textcolor{LimeGreen}{Accettabile}\\
									       \glvuno        & \hyperref[MLEC]{IG} &  \textcolor{LimeGreen}{56}               & \textcolor{LimeGreen}{Accettabile} \\
									      \vunoi       & \hyperref[MLEC]{IG} &  \textcolor{ForestGreen}{79}               & \textcolor{ForestGreen}{Ottimale}\\
									      \vduei       & \hyperref[MLEC]{IG} &  \textcolor{ForestGreen}{79}               & \textcolor{ForestGreen}{Ottimale}\\
									      \vtrei       & \hyperref[MLEC]{IG} &  \textcolor{ForestGreen}{79}               & \textcolor{ForestGreen}{Ottimale}\\
									    \vquattroi     & \hyperref[MLEC]{IG} &  \textcolor{ForestGreen}{79}               & \textcolor{ForestGreen}{Ottimale}\\
									      \vunoe       & \hyperref[MLEC]{IG} &   \textcolor{ForestGreen}{73}              & \textcolor{ForestGreen}{Ottimale}\\
									      \vduee       & \hyperref[MLEC]{IG} &   \textcolor{ForestGreen}{68}              & \textcolor{ForestGreen}{Ottimale}\\ \hline
								\end{tabular}
								\caption{Resoconto leggibilità e comprensibilità - periodo An}
								\label{tab_resoconto_leggibilità_e_comprensibilità_PA}
							\end{table}
			
						Tutti i documenti presentano un \glo{Indice Gulpease}{indice Gulpease} ad un livello almeno accettabile; ciò dovrebbe garantire una lettura non particolarmente difficoltosa da parte di soggetti con almeno licenza superiore.
						Il documento che assume il valore più basso è l'\adrvuno{}
						 Questo è dovuto al fatto che esso è un documento particolarmente tecnico e i contenuti sono esposti sotto forma di tabelle.	
						
						\subparagraph{Periodo Pl}
						
						\begin{table}[H]
							\centering
							\small
							\rowcolors{2}{white}{white}
							\begin{tabular}{c | c | c | c}
								\hline
								\textbf{Documento} & \textbf{Metrica}    & \textbf{Valore} & \textbf{Giudizio} \\ \hline
								\pdpvdue        & \hyperref[MLEC]{IG} & \textcolor{ForestGreen}{61} & \textcolor{ForestGreen}{Ottimale} \\
								\pdqvdue        & \hyperref[MLEC]{IG} & \textcolor{LimeGreen}{57} & \textcolor{LimeGreen}{Accettabile} \\
								\ndpvdue        & \hyperref[MLEC]{IG} & \textcolor{ForestGreen}{63} & \textcolor{ForestGreen}{Ottimale} \\
								\adrvdue        & \hyperref[MLEC]{IG}  & \textcolor{LimeGreen}{48} & \textcolor{LimeGreen}{Accettabile} \\
								\stvuno		& \hyperref[MLEC]{IG}  & \textcolor{LimeGreen}{51} & \textcolor{LimeGreen}{Accettabile} \\
								\glvdue        & \hyperref[MLEC]{IG} & \textcolor{LimeGreen}{56} & \textcolor{LimeGreen}{Accettabile} \\
								\vcinquei       & \hyperref[MLEC]{IG}& \textcolor{ForestGreen}{62} & \textcolor{ForestGreen}{Ottimale} \\
								\vseii       & \hyperref[MLEC]{IG} &  \textcolor{ForestGreen}{66} & \textcolor{ForestGreen}{Ottimale} \\
								\vtree       & \hyperref[MLEC]{IG}& \textcolor{ForestGreen}{66} & \textcolor{ForestGreen}{Ottimale} \\
							\end{tabular}
							\caption{Resoconto leggibilità e comprensibilità - periodo Pl}
							\label{tab_resoconto_leggibilità_e_comprensibilità_PPL}
						\end{table}
						
						Tutti i documenti presentano un \glo{Indice Gulpease}{indice Gulpease} ad un livello almeno accettabile; ciò dovrebbe garantire una lettura non particolarmente difficoltosa da parte di soggetti con almeno licenza superiore.
						Il documento che assume il valore più basso è l'\adrvdue. Questo è dovuto al fatto che esso è un documento particolarmente tecnico e i contenuti sono esposti sotto forma di tabelle.	
						
							\subparagraph{Periodo PCV}
						
						\begin{table}[H]
							\centering
							\small
							\rowcolors{2}{white}{white}
							\begin{tabular}{c | c | c | c}
								\hline
								\textbf{Documento} & \textbf{Metrica}    & \textbf{Valore} & \textbf{Giudizio} \\ \hline
								\pdpvtre        & \hyperref[MLEC]{IG} & \textcolor{LimeGreen}{57} & \textcolor{LimeGreen}{Ottimale} \\
								\pdqvtre        & \hyperref[MLEC]{IG} & \textcolor{ForestGreen}{65} & \textcolor{ForestGreen}{Accettabile} \\
								\ndpvtre        & \hyperref[MLEC]{IG} & \textcolor{ForestGreen}{64} & \textcolor{ForestGreen}{Ottimale} \\
								\adrvtre        & \hyperref[MLEC]{IG}  & \textcolor{LimeGreen}{47} & \textcolor{LimeGreen}{Accettabile} \\
								\stvdue		& \hyperref[MLEC]{IG}  & \textcolor{LimeGreen}{51} & \textcolor{LimeGreen}{Accettabile} \\
								\glvdue        & \hyperref[MLEC]{IG} & \textcolor{LimeGreen}{56} & \textcolor{LimeGreen}{Accettabile} \\
								\ddpvuno        & \hyperref[MLEC]{IG} & \textcolor{LimeGreen}{57} & \textcolor{LimeGreen}{Accettabile} \\
								\manutvuno        & \hyperref[MLEC]{IG} & \textcolor{LimeGreen}{58} & \textcolor{LimeGreen}{Accettabile} \\
								\manmanvuno        & \hyperref[MLEC]{IG} & \textcolor{LimeGreen}{55} & \textcolor{LimeGreen}{Accettabile} \\
								\vsesettei       & \hyperref[MLEC]{IG}& \textcolor{ForestGreen}{66} & \textcolor{ForestGreen}{Ottimale} \\
								\vottoi       & \hyperref[MLEC]{IG} &  \textcolor{ForestGreen}{62} & \textcolor{ForestGreen}{Ottimale} \\
								\vquattroe       & \hyperref[MLEC]{IG}& \textcolor{ForestGreen}{61} & \textcolor{ForestGreen}{Ottimale} \\
							\end{tabular}
							\caption{Resoconto leggibilità e comprensibilità - periodo PCV}
							\label{tab_resoconto_leggibilità_e_comprensibilità_PPCV}
						\end{table}
						
						%TODO
						
						\newpage
					\paragraph{Adesione alle norme interne}
						Per una descrizione dell'obiettivo, consultare \nameref{OAANI}.
					
						\subparagraph{Periodo An}
							\begin{table}[H]
								\centering
								\small
								\rowcolors{2}{white}{white}
								\begin{tabular}{c | c | c | c}
									\hline
									\textbf{Documento} & \textbf{Metrica} & \textbf{Valore} & \textbf{Giudizio} \\
									\hline
									\pdpvuno & \hyperref[MAANI]{ENNC} & \textcolor{ForestGreen}{0} & \textcolor{ForestGreen}{Ottimale} \\
									\pdqvuno & \hyperref[MAANI]{ENNC} & \textcolor{ForestGreen}{0} & \textcolor{ForestGreen}{Ottimale}\\
									\ndpvuno & \hyperref[MAANI]{ENNC} &\textcolor{ForestGreen}{0} & \textcolor{ForestGreen}{Ottimale}\\
									\sdfv & \hyperref[MAANI]{ENNC} & \textcolor{ForestGreen}{0} & \textcolor{ForestGreen}{Ottimale}\\
									\adrvuno & \hyperref[MAANI]{ENNC} & \textcolor{ForestGreen}{0} & \textcolor{ForestGreen}{Ottimale}\\
									\glvuno  & \hyperref[MAANI]{ENNC} & \textcolor{ForestGreen}{0} & \textcolor{ForestGreen}{Ottimale}\\
									\vunoi& \hyperref[MAANI]{ENNC} & \textcolor{ForestGreen}{0} & \textcolor{ForestGreen}{Ottimale}\\
									\vduei& \hyperref[MAANI]{ENNC} & \textcolor{ForestGreen}{0} & \textcolor{ForestGreen}{Ottimale}\\
									\vtrei & \hyperref[MAANI]{ENNC} & \textcolor{ForestGreen}{0} & \textcolor{ForestGreen}{Ottimale}\\
									\vquattroi & \hyperref[MAANI]{ENNC} & \textcolor{ForestGreen}{0} & \textcolor{ForestGreen}{Ottimale}\\
									\vunoe & \hyperref[MAANI]{ENNC} & \textcolor{ForestGreen}{0} & \textcolor{ForestGreen}{Ottimale}\\
									\vduee & \hyperref[MAANI]{ENNC} & \textcolor{ForestGreen}{0} & \textcolor{ForestGreen}{Ottimale}\\
									\hline
								\end{tabular}
								\caption{Resoconto adesione alle norme interne - periodo An}
								\label{tab_resoconto_adesione_alle_norme_interne_PA}
							\end{table}
						
							Per tutti i documenti non risultano errori residui che violino le norme interne, pertanto le metriche hanno un valore ottimale.
						
						
						\subparagraph{Periodo Pl}
						
						\begin{table}[H]
							\centering
							\small
							\rowcolors{2}{white}{white}
							\begin{tabular}{c | c | c | c}
								\hline
								\textbf{Documento} & \textbf{Metrica} & \textbf{Valore} & \textbf{Giudizio} \\
								\hline
								\pdpvdue     & \hyperref[MLEC]{ENNC}& \textcolor{ForestGreen}{0} & \textcolor{ForestGreen}{Ottimale} \\
								\pdqvdue     & \hyperref[MLEC]{ENNC}& \textcolor{ForestGreen}{0} & \textcolor{ForestGreen}{Ottimale} \\
								\ndpvdue     & \hyperref[MLEC]{ENNC} & \textcolor{ForestGreen}{0} & \textcolor{ForestGreen}{Ottimale} \\
								\adrvdue     & \hyperref[MLEC]{ENNC} & \textcolor{ForestGreen}{0} & \textcolor{ForestGreen}{Ottimale} \\
								\stvuno		& \hyperref[MLEC]{ENNC} & \textcolor{ForestGreen}{0} & \textcolor{ForestGreen}{Ottimale} \\
								\glvdue     & \hyperref[MLEC]{ENNC} & \textcolor{ForestGreen}{0} & \textcolor{ForestGreen}{Ottimale} \\
								\vcinquei       & \hyperref[MLEC]{ENNC} & \textcolor{ForestGreen}{0} & \textcolor{ForestGreen}{Ottimale} \\
								\vseii       & \hyperref[MLEC]{ENNC}& \textcolor{ForestGreen}{0} & \textcolor{ForestGreen}{Ottimale} \\
								\vtree       & \hyperref[MLEC]{ENNC} & \textcolor{ForestGreen}{0} & \textcolor{ForestGreen}{Ottimale} \\
								\hline
							\end{tabular}
							\caption{Resoconto adesione alle norme interne - periodo Pl}
							\label{tab_resoconto_adesione_alle_norme_interne_PPL}
						\end{table}

						Per tutti i documenti non risultano errori residui che violino le norme interne, pertanto le metriche hanno un valore ottimale.
						
						\subparagraph{Periodo PCV}
						
						\begin{table}[H]
							\centering
							\small
							\rowcolors{2}{white}{white}
						\begin{tabular}{c | c | c | c}
							\hline
							\textbf{Documento} & \textbf{Metrica}    & \textbf{Valore} & \textbf{Giudizio} \\ \hline
							\pdpvtre        & \hyperref[MLEC]{ENNC}  & \textcolor{ForestGreen}{0} & \textcolor{ForestGreen}{Ottimale} \\
							\pdqvtre        & \hyperref[MLEC]{ENNC} &  \textcolor{ForestGreen}{0} & \textcolor{ForestGreen}{Ottimale} \\
							\ndpvtre        & \hyperref[MLEC]{ENNC}  & \textcolor{ForestGreen}{0} & \textcolor{ForestGreen}{Ottimale} \\
							\adrvtre        & \hyperref[MLEC]{ENNC}   & \textcolor{ForestGreen}{0} & \textcolor{ForestGreen}{Ottimale} \\
							\stvdue		& \hyperref[MLEC]{ENNC}  & \textcolor{ForestGreen}{0} & \textcolor{ForestGreen}{Ottimale} \\
							\glvdue        & \hyperref[MLEC]{ENNC}  & \textcolor{ForestGreen}{0} & \textcolor{ForestGreen}{Ottimale} \\
							\ddpvuno        & \hyperref[MLEC]{ENNC}  & \textcolor{ForestGreen}{0} & \textcolor{ForestGreen}{Ottimale} \\
							\manutvuno        & \hyperref[MLEC]{ENNC}  & \textcolor{ForestGreen}{0} & \textcolor{ForestGreen}{Ottimale} \\
							\manmanvuno        & \hyperref[MLEC]{ENNC}  & \textcolor{ForestGreen}{0} & \textcolor{ForestGreen}{Ottimale} \\
							\vsesettei       & \hyperref[MLEC]{ENNC} & \textcolor{ForestGreen}{0} & \textcolor{ForestGreen}{Ottimale} \\
							\vottoi       & \hyperref[MLEC]{ENNC}  & \textcolor{ForestGreen}{0} & \textcolor{ForestGreen}{Ottimale} \\
							\vquattroe       & \hyperref[MLEC]{ENNC} & \textcolor{ForestGreen}{0} & \textcolor{ForestGreen}{Ottimale} \\
						\end{tabular}
							\caption{Resoconto adesione alle norme interne - periodo PCV}
							\label{tab_resoconto_adesione_alle_norme_interne_PCV}
						\end{table}
						
						Per tutti i documenti non risultano errori residui che violino le norme interne, pertanto le metriche hanno un valore ottimale.
						
						\newpage
				\paragraph{Correttezza ortografica}
					Per una descrizione dell'obiettivo, consultare \nameref{OCO}.
				
					\subparagraph{Periodo An}
				
					\begin{table}[H]
						\centering
						\rowcolors{2}{white}{white}
						\small
						\begin{tabular}{c | c | c | c}
							\hline
							\textbf{Documento} & \textbf{Metrica} & \textbf{Valore} & \textbf{Giudizio} \\
							\hline
							\pdpvuno & \hyperref[MCO]{EONC} & \textcolor{ForestGreen}{0} & \textcolor{ForestGreen}{Ottimale} \\
							\pdqvuno & \hyperref[MCO]{EONC} & \textcolor{ForestGreen}{0} & \textcolor{ForestGreen}{Ottimale}\\
							\ndpvuno & \hyperref[MCO]{EONC} &\textcolor{ForestGreen}{0} & \textcolor{ForestGreen}{Ottimale}\\
							\sdfv & \hyperref[MCO]{EONC} & \textcolor{ForestGreen}{0} & \textcolor{ForestGreen}{Ottimale}\\
							\adrvuno & \hyperref[MCO]{EONC} & \textcolor{ForestGreen}{0} & \textcolor{ForestGreen}{Ottimale}\\
							\glvuno  & \hyperref[MCO]{EONC} & \textcolor{ForestGreen}{0} & \textcolor{ForestGreen}{Ottimale}\\
							\vunoi& \hyperref[MAANI]{EONC} & \textcolor{ForestGreen}{0} & \textcolor{ForestGreen}{Ottimale}\\
							\vduei& \hyperref[MAANI]{EONC} & \textcolor{ForestGreen}{0} & \textcolor{ForestGreen}{Ottimale}\\
							\vtrei & \hyperref[MAANI]{EONC} & \textcolor{ForestGreen}{0} & \textcolor{ForestGreen}{Ottimale}\\
							\vquattroi & \hyperref[MAANI]{EONC} & \textcolor{ForestGreen}{0} & \textcolor{ForestGreen}{Ottimale}\\
							\vunoe & \hyperref[MAANI]{EONC} & \textcolor{ForestGreen}{0} & \textcolor{ForestGreen}{Ottimale}\\
							\vduee & \hyperref[MAANI]{EONC} & \textcolor{ForestGreen}{0} & \textcolor{ForestGreen}{Ottimale}\\
							\hline
						\end{tabular}
						\caption{Resoconto correttezza ortografica - periodo An}
						\label{tab_resoconto_correttezza_ortografica_PA}
					\end{table}
				
					Dopo l'analisi automatica dei correttori ortografici e quella mediante walkthrough da parte dei \verificatori{} non sono stati rilevati ulteriori errori che violano le norme interne, pertanto le metriche assumono un valore ottimale.
					
				\subparagraph{Periodo Pl}
				
						\begin{table}[H]
						\centering
						\small
						\rowcolors{2}{white}{white}
						\begin{tabular}{c | c | c | c}
							\hline
							\textbf{Documento} & \textbf{Metrica} & \textbf{Valore} & \textbf{Giudizio} \\
							\hline
							\pdpvdue  &     \hyperref[MCO]{EONC} & \textcolor{ForestGreen}{0} & \textcolor{ForestGreen}{Ottimale} \\
							\pdqvdue  &      \hyperref[MCO]{EONC} & \textcolor{ForestGreen}{0} & \textcolor{ForestGreen}{Ottimale} \\
							\ndpvdue   &   \hyperref[MCO]{EONC} & \textcolor{ForestGreen}{0} & \textcolor{ForestGreen}{Ottimale} \\
							\adrvdue   &   \hyperref[MCO]{EONC} & \textcolor{ForestGreen}{0} & \textcolor{ForestGreen}{Ottimale} \\
							\stvuno &	\hyperref[MCO]{EONC} & \textcolor{ForestGreen}{0} & \textcolor{ForestGreen}{Ottimale} \\
							\glvdue &     \hyperref[MCO]{EONC} & \textcolor{ForestGreen}{0} & \textcolor{ForestGreen}{Ottimale} \\
							\vcinquei&      \hyperref[MCO]{EONC} & \textcolor{ForestGreen}{0} & \textcolor{ForestGreen}{Ottimale} \\
							\vseii   &  \hyperref[MCO]{EONC} & \textcolor{ForestGreen}{0} & \textcolor{ForestGreen}{Ottimale} \\
							\vtree   &  \hyperref[MCO]{EONC} & \textcolor{ForestGreen}{0} & \textcolor{ForestGreen}{Ottimale} \\
							\hline
						\end{tabular}
						\caption{Resoconto correttezza ortografica - periodo Pl}
						\label{tab_resoconto_correttezza_ortografica_PPL}
					\end{table}

					Dopo l'analisi automatica dei correttori ortografici e quella mediante walkthrough da parte dei \verificatori{} non sono stati rilevati ulteriori errori che violano le norme interne, pertanto le metriche assumono un valore ottimale.
						
						\subparagraph{Periodo PCV}
						
						\begin{table}[H]
							\centering
							\small
							\rowcolors{2}{white}{white}
							\begin{tabular}{c | c | c | c}
								\hline
								\textbf{Documento} & \textbf{Metrica}    & \textbf{Valore} & \textbf{Giudizio} \\ \hline
								\pdpvtre        & \hyperref[MLEC]{EONC}  & \textcolor{ForestGreen}{0} & \textcolor{ForestGreen}{Ottimale} \\
								\pdqvtre        & \hyperref[MLEC]{EONC} &  \textcolor{ForestGreen}{0} & \textcolor{ForestGreen}{Ottimale} \\
								\ndpvtre        & \hyperref[MLEC]{EONC}  & \textcolor{ForestGreen}{0} & \textcolor{ForestGreen}{Ottimale} \\
								\adrvtre        & \hyperref[MLEC]{EONC}   & \textcolor{ForestGreen}{0} & \textcolor{ForestGreen}{Ottimale} \\
								\stvdue		& \hyperref[MLEC]{EONC}  & \textcolor{ForestGreen}{0} & \textcolor{ForestGreen}{Ottimale} \\
								\glvdue        & \hyperref[MLEC]{EONC}  & \textcolor{ForestGreen}{0} & \textcolor{ForestGreen}{Ottimale} \\
								\ddpvuno        & \hyperref[MLEC]{EONC}  & \textcolor{ForestGreen}{0} & \textcolor{ForestGreen}{Ottimale} \\
								\manutvuno        & \hyperref[MLEC]{EONC}  & \textcolor{ForestGreen}{0} & \textcolor{ForestGreen}{Ottimale} \\
								\manmanvuno        & \hyperref[MLEC]{EONC}  & \textcolor{ForestGreen}{0} & \textcolor{ForestGreen}{Ottimale} \\
								\vsesettei       & \hyperref[MLEC]{EONC} & \textcolor{ForestGreen}{0} & \textcolor{ForestGreen}{Ottimale} \\
								\vottoi       & \hyperref[MLEC]{EONC}  & \textcolor{ForestGreen}{0} & \textcolor{ForestGreen}{Ottimale} \\
								\vquattroe       & \hyperref[MLEC]{EONC} & \textcolor{ForestGreen}{0} & \textcolor{ForestGreen}{Ottimale} \\
							\end{tabular}
							\caption{Resoconto correttezza ortografica - periodo PCV}
							\label{tab_resoconto_correttezza_ortografica_PCV}
						\end{table}
						
						Dopo l'analisi automatica dei correttori ortografici e quella mediante walkthrough da parte dei \verificatori{} non sono stati rilevati ulteriori errori che violano le norme interne, pertanto le metriche assumono un valore ottimale.
						
						\newpage
				\paragraph{Correttezza concettuale}
					Per una descrizione dell'obiettivo, consultare \nameref{OCC}.
					\subparagraph{Periodo An}
				
						\begin{table}[H]
							\centering
							\rowcolors{2}{white}{white}
							\small
							\begin{tabular}{c | c | c | c}
								\hline
								\textbf{Documento} & \textbf{Metrica} & \textbf{Valore} & \textbf{Giudizio} \\
								\hline
								\pdpvuno & \hyperref[MCC]{ECNC} & \textcolor{ForestGreen}{0} & \textcolor{ForestGreen}{Ottimale} \\
								\pdqvuno & \hyperref[MCC]{ECNC} & \textcolor{ForestGreen}{0} & \textcolor{ForestGreen}{Ottimale}\\
								\ndpvuno & \hyperref[MCC]{ECNC} &\textcolor{ForestGreen}{0} & \textcolor{ForestGreen}{Ottimale}\\
								\sdfv & \hyperref[MCC]{ECNC} & \textcolor{ForestGreen}{0} & \textcolor{ForestGreen}{Ottimale}\\
								\adrvuno & \hyperref[MCC]{ECNC} & \textcolor{ForestGreen}{0} & \textcolor{ForestGreen}{Ottimale}\\
								\glvuno  & \hyperref[MCC]{ECNC} & \textcolor{ForestGreen}{0} & \textcolor{ForestGreen}{Ottimale}\\
								\vunoi& \hyperref[MCC]{ECNC} & \textcolor{ForestGreen}{0} & \textcolor{ForestGreen}{Ottimale}\\
								\vduei& \hyperref[MCC]{ECNC} & \textcolor{ForestGreen}{0} & \textcolor{ForestGreen}{Ottimale}\\
								\vtrei & \hyperref[MCC]{ECNC} & \textcolor{ForestGreen}{0} & \textcolor{ForestGreen}{Ottimale}\\
								\vquattroi & \hyperref[MCC]{ECNC} & \textcolor{ForestGreen}{0} & \textcolor{ForestGreen}{Ottimale}\\
								\vunoe & \hyperref[MCC]{ECNC} & \textcolor{ForestGreen}{0} & \textcolor{ForestGreen}{Ottimale}\\
								\vduee & \hyperref[MCC]{ECNC} & \textcolor{ForestGreen}{0} & \textcolor{ForestGreen}{Ottimale}\\
								\hline
							\end{tabular}
							\caption{Resoconto correttezza concettuale - periodo An}
							\label{tab_resoconto_correttezza_concettuale_PA}
						\end{table}

							Per tutti i documenti non sono stati rilevati errori concettuali non corretti, pertanto le metriche assumono un valore ottimale.
							
							
							\subparagraph{Periodo Pl}
							
							\begin{table}[H]
								\centering
								\rowcolors{2}{white}{white}
								\small
							\begin{tabular}{c | c | c | c}
								\hline
								\textbf{Documento} & \textbf{Metrica} & \textbf{Valore} & \textbf{Giudizio} \\
								\hline
								\pdpvdue  &     \hyperref[MCO]{ECNC} & \textcolor{ForestGreen}{0} & \textcolor{ForestGreen}{Ottimale} \\
								\pdqvdue  &      \hyperref[MCO]{ECNC} & \textcolor{ForestGreen}{0} & \textcolor{ForestGreen}{Ottimale} \\
								\ndpvdue   &   \hyperref[MCO]{ECNC} & \textcolor{ForestGreen}{0} & \textcolor{ForestGreen}{Ottimale} \\
								\adrvdue   &   \hyperref[MCO]{ECNC} & \textcolor{ForestGreen}{0} & \textcolor{ForestGreen}{Ottimale} \\
								\stvuno &	\hyperref[MCO]{ECNC} & \textcolor{ForestGreen}{0} & \textcolor{ForestGreen}{Ottimale} \\
								\glvdue &     \hyperref[MCO]{ECNC} & \textcolor{ForestGreen}{0} & \textcolor{ForestGreen}{Ottimale} \\
								\vcinquei&      \hyperref[MCO]{ECNC} & \textcolor{ForestGreen}{0} & \textcolor{ForestGreen}{Ottimale} \\
								\vseii   &  \hyperref[MCO]{ECNC} & \textcolor{ForestGreen}{0} & \textcolor{ForestGreen}{Ottimale} \\
								\vtree   &  \hyperref[MCO]{ECNC} & \textcolor{ForestGreen}{0} & \textcolor{ForestGreen}{Ottimale} \\
								\hline
							\end{tabular}
								\caption{Resoconto correttezza concettuale - periodo Pl}
								\label{tab_resoconto_correttezza_concettuale_PPL}
							\end{table}
							
							\subparagraph{Periodo PCV}
							Per tutti i documenti non sono stati rilevati errori concettuali non corretti, pertanto le metriche assumono un valore ottimale.
							
							\begin{table}[H]
								\centering
								\rowcolors{2}{white}{white}
								\small
								\begin{tabular}{c | c | c | c}
									\hline
									\textbf{Documento} & \textbf{Metrica}    & \textbf{Valore} & \textbf{Giudizio} \\ \hline
									\pdpvtre        & \hyperref[MLEC]{ECNC}  & \textcolor{ForestGreen}{0} & \textcolor{ForestGreen}{Ottimale} \\
									\pdqvtre        & \hyperref[MLEC]{ECNC} &  \textcolor{ForestGreen}{0} & \textcolor{ForestGreen}{Ottimale} \\
									\ndpvtre        & \hyperref[MLEC]{ECNC}  & \textcolor{ForestGreen}{0} & \textcolor{ForestGreen}{Ottimale} \\
									\adrvtre        & \hyperref[MLEC]{ECNC}   & \textcolor{ForestGreen}{0} & \textcolor{ForestGreen}{Ottimale} \\
									\stvdue		& \hyperref[MLEC]{ECNC}  & \textcolor{ForestGreen}{0} & \textcolor{ForestGreen}{Ottimale} \\
									\glvdue        & \hyperref[MLEC]{ECNC}  & \textcolor{ForestGreen}{0} & \textcolor{ForestGreen}{Ottimale} \\
									\ddpvuno        & \hyperref[MLEC]{ECNC}  & \textcolor{ForestGreen}{0} & \textcolor{ForestGreen}{Ottimale} \\
									\manutvuno        & \hyperref[MLEC]{ECNC}  & \textcolor{ForestGreen}{0} & \textcolor{ForestGreen}{Ottimale} \\
									\manmanvuno        & \hyperref[MLEC]{ECNC}  & \textcolor{ForestGreen}{0} & \textcolor{ForestGreen}{Ottimale} \\
									\vsesettei       & \hyperref[MLEC]{ECNC} & \textcolor{ForestGreen}{0} & \textcolor{ForestGreen}{Ottimale} \\
									\vottoi       & \hyperref[MLEC]{ECNC}  & \textcolor{ForestGreen}{0} & \textcolor{ForestGreen}{Ottimale} \\
									\vquattroe       & \hyperref[MLEC]{ECNC} & \textcolor{ForestGreen}{0} & \textcolor{ForestGreen}{Ottimale} \\
								\end{tabular}
								\caption{Resoconto correttezza concettuale - periodo PCV}
								\label{tab_resoconto_correttezza_concettuale_PPCV}
							\end{table}
							
							Per tutti i documenti non sono stati rilevati errori concettuali non corretti, pertanto le metriche assumono un valore ottimale.
				
				\newpage
					\paragraph{Basso livello di annidamento dell'indice}
						Per una descrizione dell'obiettivo, consultare \nameref{OBLDAI}.
						\subparagraph{Periodo Pl}
						\begin{table}[H]
							\centering
							\small
							\rowcolors{2}{white}{white}
							\begin{tabular}{c | c | c | c}
								\hline
								\textbf{Documento} & \textbf{Metrica} & \textbf{Valore} & \textbf{Giudizio} \\
								\hline
								\pdpvdue        & \hyperref[MLEC]{LA} & \textcolor{LimeGreen}{4} & \textcolor{LimeGreen}{Accettabile} \\
								\pdqvdue        & \hyperref[MLEC]{LA} & \textcolor{LimeGreen}{4} & \textcolor{LimeGreen}{Accettabile} \\
								\ndpvdue        & \hyperref[MLEC]{LA} & \textcolor{LimeGreen}{5} & \textcolor{LimeGreen}{Accettabile} \\
								\adrvdue        & \hyperref[MLEC]{LA}& \textcolor{ForestGreen}{2} & \textcolor{ForestGreen}{Ottimale} \\
								\stvuno		& \hyperref[MLEC]{LA} & \textcolor{LimeGreen}{5} & \textcolor{LimeGreen}{Accettabile} \\
								\glvdue        & \hyperref[MLEC]{LA} & \textcolor{ForestGreen}{1} & \textcolor{ForestGreen}{Ottimale} \\
								\vcinquei       &\hyperref[MLEC]{LA} & \textcolor{ForestGreen}{1} & \textcolor{ForestGreen}{Ottimale} \\
								\vseii       & \hyperref[MLEC]{LA} & \textcolor{ForestGreen}{1} & \textcolor{ForestGreen}{Ottimale} \\
								\vtree       & \hyperref[MLEC]{LA} & \textcolor{ForestGreen}{1} & \textcolor{ForestGreen}{Ottimale} \\
								\hline
							\end{tabular}
							\caption{Resoconto basso livello di annidamento dell'indice - periodo Pl}
							\label{tab_resoconto_basso_livello_di_annidamento_indice_PPL}
						\end{table}
						
					Per tutti i documenti il livello di annidamento dell'indice risulta accettabile. In molti casi è stata preferita una struttura tabellare come alternativa all'eccessivo annidamento.
					\subparagraph{Periodo PCV}
						\begin{table}[H]
						\centering
						\small
						\rowcolors{2}{white}{white}
						\begin{tabular}{c | c | c | c}
							\hline
							\textbf{Documento} & \textbf{Metrica} & \textbf{Valore} & \textbf{Giudizio} \\
							\hline
							\pdpvtre        & \hyperref[MLEC]{LA} & \textcolor{LimeGreen}{5} & \textcolor{LimeGreen}{Accettabile} \\
							\pdqvtre        & \hyperref[MLEC]{LA} & \textcolor{LimeGreen}{4} & \textcolor{LimeGreen}{Accettabile} \\
							\ndpvtre        & \hyperref[MLEC]{LA} & \textcolor{LimeGreen}{5} & \textcolor{LimeGreen}{Accettabile} \\
							\adrvtre        & \hyperref[MLEC]{LA}& \textcolor{ForestGreen}{2} & \textcolor{ForestGreen}{Ottimale} \\
							\stvdue		& \hyperref[MLEC]{LA} & \textcolor{LimeGreen}{5} & \textcolor{LimeGreen}{Accettabile} \\
							\glvdue        & \hyperref[MLEC]{LA} & \textcolor{ForestGreen}{1} & \textcolor{ForestGreen}{Ottimale} \\
							\ddpvuno        & \hyperref[MLEC]{LA}  & \textcolor{LimeGreen}{5} & \textcolor{LimeGreen}{Ottimale} \\
							\manutvuno        & \hyperref[MLEC]{LA}  & \textcolor{ForestGreen}{2} & \textcolor{ForestGreen}{Ottimale} \\
							\manmanvuno        & \hyperref[MLEC]{LA}  & \textcolor{LimeGreen}{4} & \textcolor{LimeGreen}{Accettabile} \\
							\vsesettei       & \hyperref[MLEC]{LA} & \textcolor{ForestGreen}{1} & \textcolor{ForestGreen}{Ottimale} \\
							\vottoi       & \hyperref[MLEC]{LA}  & \textcolor{ForestGreen}{1} & \textcolor{ForestGreen}{Ottimale} \\
							\vquattroe       & \hyperref[MLEC]{LA} & \textcolor{ForestGreen}{1} & \textcolor{ForestGreen}{Ottimale} \\
						\end{tabular}
						\caption{Resoconto basso livello di annidamento dell'indice - periodo PCV}
						\label{tab_resoconto_basso_livello_di_annidamento_indice_PPCV}
					\end{table}
					
					Per tutti i documenti il livello di annidamento dell'indice risulta accettabile. In molti casi è stata preferita una struttura tabellare come alternativa all'eccessivo annidamento.
					
		\subsubsection{Verifica del software}
			Gli obiettivi riguardanti il software sono stati verificati solamente dal periodo Progettazione di dettaglio, Codifica e Validazione in poi. Nei periodi precedenti non esisteva ancora codice, quindi la verifica di tali obiettivi non è stata possibile.
			
			\paragraph{Implementazione delle funzionalità obbligatorie}
			Per una descrizione dell'obiettivo, consultare \nameref{OIDFO}.
				\subparagraph{Periodo PCV}
			\begin{table}[H]
				\centering
				\rowcolors{2}{white}{white}
				\begin{tabular}{  c | c | c}
					\hline
					\textbf{Metrica} & \textbf{Valore} & \textbf{Giudizio} \\
					\hline
					\hyperref[MMC]{IFO}   & \textcolor{Red}{0}          & \textcolor{Red}{Non accettabile}  \\ \hline
				\end{tabular} 
			\end{table}
					Non sono state ancora implementate funzionalità obbligatorie: dato che i test di sistema non sono stati ancora implementati e superati, non è possibile confermare l'effettiva implementazione delle funzionalità. Questo rivela un grave ritardo nei lavori. Tuttavia, i test di unità sono già stati pianificati e superati parzialmente. 
					
					L'obiettivo per il prossimo periodo è implementare completamente le funzionalità obbligatorie.
			
			\newpage
			\paragraph{Implementazione delle funzionalità desiderabili}
			Per una descrizione dell'obiettivo, consultare \nameref{OIDFD}.
				\subparagraph{Periodo PCV}
				
				\begin{table}[H]
					\centering
					\rowcolors{2}{white}{white}
					\begin{tabular}{  c | c | c}
						\hline
						\textbf{Metrica} & \textbf{Valore} & \textbf{Giudizio} \\
						\hline
						\hyperref[MMC]{IFD}   & \textcolor{Red}{0}          & \textcolor{Red}{Non accettabile}  \\ \hline
					\end{tabular} 
				\end{table}
			
			Non sono state ancora implementate funzionalità desiderabili: dato che i test di sistema non sono stati ancora implementati e superati, non è possibile confermare l'effettiva implementazione delle funzionalità. Questo rivela un grave ritardo nei lavori. Tuttavia, i test di unità sono già stati pianificati e superati parzialmente. 
			
			
			L'obiettivo per il prossimo periodo è implementare le funzionalità desiderabili per raggiungere l'obiettivo associato.
			
			\newpage
			
			\paragraph{Basso numero di statement per metodo}
			Per una descrizione dell'obiettivo, consultare \nameref{OBNDSPM}.
				\subparagraph{Periodo PCV}
			
			\begin{table}[H]
				\centering
				\rowcolors{2}{white}{white}
				\begin{tabular}{  c | c | c}
					\hline
					\textbf{Metrica} & \textbf{Valore} & \textbf{Giudizio} \\
					\hline
					\hyperref[MMC]{NSM}   & \textcolor{ForestGreen}{16}          & \textcolor{ForestGreen}{Ottimale}  \\ \hline
				\end{tabular} 
			\end{table}
			Il metodo con massimo numero di statement è il constructor della classe Asset. Tale valore è all'interno della soglia di ottimalità.
			
			L'obiettivo per il prossimo periodo è cercare di non codificare metodi con numero di statement maggiore.
			
		\newpage
			\paragraph{Basso numero di parametri per metodo}
			Per una descrizione dell'obiettivo, consultare \nameref{OBNDPPM}.
				\subparagraph{Periodo PCV}
				
				\begin{table}[H]
					\centering
					\rowcolors{2}{white}{white}
					\begin{tabular}{  c | c | c}
						\hline
						\textbf{Metrica} & \textbf{Valore} & \textbf{Giudizio} \\
						\hline
						\hyperref[MMC]{NPM}   & \textcolor{ForestGreen}{3}          & \textcolor{ForestGreen}{Ottimale}  \\ \hline
					\end{tabular} 
				\end{table}
			Il metodo con più alto numero di parametri è handleChange della classe DeGeOPView. Il valore è ottimale
			
			L'obiettivo per il prossimo periodo è cercare di non codificare metodi con un numero di parametri maggiore.
			
			\newpage
			\paragraph{Basso numero di campi dati per classe}
			Per una descrizione dell'obiettivo, consultare \nameref{OBNDCDPC}.
				\subparagraph{Periodo PCV}
				
				\begin{table}[H]
					\centering
					\rowcolors{2}{white}{white}
					\begin{tabular}{  c | c | c}
						\hline
						\textbf{Metrica} & \textbf{Valore} & \textbf{Giudizio} \\
						\hline
						\hyperref[MMC]{NCDPC}   & \textcolor{Red}{19}          & \textcolor{Red}{Non accettabile}  \\ \hline
					\end{tabular} 
				\end{table}
				La classe con il massimo numero di campi dati è mapWrapper. Non è stato ritenuto sensato dividere le funzionalità offerte da questa classe in quanto esse sono fortemente coese. È stato preferito mantenere basso il numero di dipendenze per mantenere un basso livello di accoppiamento.
				
				\newpage
			\paragraph{Bassa complessità ciclomatica}
			Per una descrizione dell'obiettivo, consultare \nameref{OBCC}.
				\subparagraph{Periodo PCV}
				
				\begin{table}[H]
					\centering
					\rowcolors{2}{white}{white}
					\begin{tabular}{  c | c | c}
						\hline
						\textbf{Metrica} & \textbf{Valore} & \textbf{Giudizio} \\
						\hline
						\hyperref[MMC]{NC}   & \textcolor{ForestGreen}{5}          & \textcolor{ForestGreen}{Ottimale}  \\ \hline
					\end{tabular} 
				\end{table}
			Il metodo con il massimo numero di complessità ciclomatica è componentDidUpdate della classe DeGeOPView. Il valore è ottimale.
			
			L'obiettivo per il prossimo periodo è cercare di non codificare metodi con complessità ciclomatica maggiore in modo da poter ottenere un branch coverage elevato più facilmente durante l'implementazione dei test.
			
			\newpage
			\paragraph{Assenza di variabili dichiarate e non utilizzate}
			Per una descrizione dell'obiettivo, consultare \nameref{OADVDENU}.
				\subparagraph{Periodo PCV}
			
			\begin{table}[H]
				\centering
				\rowcolors{2}{white}{white}
				\begin{tabular}{  c | c | c}
					\hline
					\textbf{Metrica} & \textbf{Valore} & \textbf{Giudizio} \\
					\hline
					\hyperref[MMC]{NVNU}   & \textcolor{ForestGreen}{0}          & \textcolor{ForestGreen}{Ottimale}  \\ \hline
				\end{tabular} 
			\end{table}
		
			Non risultano esserci variabili dichiarate e non utilizzate. La metrica assume quindi un valore ottimale.
		
			\newpage
			\paragraph{Documentazione del codice}
			Per una descrizione dell'obiettivo, consultare \nameref{ODDC}.
				\subparagraph{Periodo PCV}
				
				\begin{table}[H]
					\centering
					\rowcolors{2}{white}{white}
					\begin{tabular}{  c | c | c}
						\hline
						\textbf{Metrica} & \textbf{Valore} & \textbf{Giudizio} \\
						\hline
						\hyperref[MMC]{RCC}   & \textcolor{ForestGreen}{76\%}          & \textcolor{ForestGreen}{Ottimale}  \\ \hline
					\end{tabular} 
				\end{table}	
			La metrica assume un valore ottimale. Il codice risulta essere documentato in maniera ottimale.
			
			\newpage
			\paragraph{Superamento dei test pianificati}
			Per una descrizione dell'obiettivo, consultare \nameref{OSDTP}.
				\subparagraph{Periodo PCV}
				
				\begin{table}[H]
					\centering
					\rowcolors{2}{white}{white}
					\begin{tabular}{  c | c | c}
						\hline
						\textbf{Metrica} & \textbf{Valore} & \textbf{Giudizio} \\
						\hline
						\hyperref[MMC]{STP}   & \textcolor{Red}{13.93\%}          & \textcolor{Red}{Non accettabile}  \\ \hline
					\end{tabular} 
				\end{table}
				La metrica assume un valore fortemente negativo. Ciò denota un grave ritardo nei lavori.
				
				L'obiettivo per il prossimo periodo è implementare e superare i test mancanti.
				
				\newpage
			\paragraph{Robustezza}
			Per una descrizione dell'obiettivo, consultare \nameref{OR}.
				\subparagraph{Periodo PCV}
				
				\begin{table}[H]
					\centering
					\rowcolors{2}{white}{white}
					\begin{tabular}{  c | c | c}
						\hline
						\textbf{Metrica} & \textbf{Valore} & \textbf{Giudizio} \\
						\hline
						\hyperref[MMC]{BA}   & \textcolor{ForestGreen}{100\%}          & \textcolor{ForestGreen}{Ottimale}  \\ \hline
					\end{tabular} 
				\end{table}
			I test effettuati non hanno rilevato situazioni anomale che comportino l'interruzione del funzionamento del prodotto. La metrica assume quindi un valore ottimale
			
			\newpage
			\paragraph{Correzione delle situazioni di fallimento}
			Per una descrizione dell'obiettivo, consultare \nameref{OCDSDF}.
				\subparagraph{Periodo PCV}
				
				\begin{table}[H]
					\centering
					\rowcolors{2}{white}{white}
					\begin{tabular}{  c | c | c}
						\hline
						\textbf{Metrica} & \textbf{Valore} & \textbf{Giudizio} \\
						\hline
						\hyperref[MMC]{FA}   & \textcolor{ForestGreen}{100\%}          & \textcolor{ForestGreen}{Ottimale}  \\ \hline
					\end{tabular} 
				\end{table}
			
			Tutte le situazioni di fallimento rilevate dai test sono state corrette, pertanto la metrica assume un valore ottimale.
			
			L'obiettivo è continuare a risolvere tutte le situazioni di fallimento rilevate durante l'esecuzione dei test.
			
			\newpage
			\paragraph{Copertura degli statement}
			Per una descrizione dell'obiettivo, consultare \nameref{OCDS}.
				\subparagraph{Periodo PCV}
				
				\begin{table}[H]
					\centering
					\rowcolors{2}{white}{white}
					\begin{tabular}{  c | c | c}
						\hline
						\textbf{Metrica} & \textbf{Valore} & \textbf{Giudizio} \\
						\hline
						\hyperref[MMC]{SC}   & \textcolor{ForestGreen}{100\%}          & \textcolor{ForestGreen}{Ottimale}  \\ \hline
					\end{tabular} 
				\end{table}
			I test effettuati eseguono almeno una volta ogni statement del codice che testano. 
			
			La metrica assume quindi un valore ottimale.
			\newpage
			\paragraph{Copertura dei branch}
			Per una descrizione dell'obiettivo, consultare \nameref{OCDB}.
				\subparagraph{Periodo PCV}
				
				\begin{table}[H]
					\centering
					\rowcolors{2}{white}{white}
					\begin{tabular}{  c | c | c}
						\hline
						\textbf{Metrica} & \textbf{Valore} & \textbf{Giudizio} \\
						\hline
						\hyperref[MMC]{BC}   & \textcolor{ForestGreen}{100\%}          & \textcolor{ForestGreen}{Ottimale}  \\ \hline
					\end{tabular} 
				\end{table}
			
			I test effettuati eseguono almeno una volta ogni branch del codice che testano. 
			
			La metrica assume quindi un valore ottimale.
			
			
			
			
			
			
				
					