\newpage

\section{Stime di fattibilità e bisogno di risorse}
\label{sec:sdf}

L'architettura definita precedentemente ha raggiunto un livello di dettaglio sufficiente a fornire una stima sulla fattibilità e di bisogno di risorse.
Durante la progettazione iniziale del prodotto e la scelta delle tecnologie da utilizzare, si sono riscontrati diversi potenziali limiti legati agli stessi.
Il \glo{Gruppo}{gruppo} si è impegnato a coprire le parti carenti come descritto nei paragrafi sottostanti, al fine di rendere le tecnologie completamente adeguate per la realizzazione del prodotto.

\subsection{JavaScript}

L'utilizzo ottimale di \glo{Libreria}{librerie} come \glo{React}{React} e Redux prevedono l'uso di costrutti e sintassi presenti solo dalla versione ES6 di \glo{JavaScript}{JavaScript} in poi. Tale versione non è ancora totalmente definita e quindi molti browser odierni non supportano nativamente alcune feature. Specificatamente ES6 ha inserito la sintassi relativa al costrutto delle classi (class, construct, ecc.) e alcuni operatori utili all'implementazione dei \glo{Reducer}{reducer} in Redux (per esempio lo spread operator). Questa situazione ha portato ad una serie di conseguenze, fra le quali:
\begin{itemize}
	\item utilizzo di Babel, una \glo{Componente}{componente} \js{} che agisce come compilatore (o meglio, come un refactor di codice) trasformando codice, che utilizza le feature definite in ES6 o superiore, in codice completamente compatibile alla versione ES5, che quindi è supportato nativamente dai browser moderni;
	\item bassa presenza di codice e librerie scritte in ES6. Per esempio la libreria OpenLayer e la sua documentazione sono scritte in codice ES5. Ciò si ripercuote sul prodotto da definire in due modi:
	\begin{itemize}
		\item necessità di convertire il codice ES5 in codice ES6 per mantenere uniforme il codice prodotto;
		\item necessità di importare librerie secondarie che operano tale conversione liberando il team dalla scrittura di ulteriore codice.
	\end{itemize}
\end{itemize}
Il \glo{Package}{package} Babel è una componente popolare e molto usata quindi il rischio che  non funzioni correttamente è molto basso, ma comunque da non trascurare.

\subsection{React}

La libreria React è relativamente giovane, ma risulta molto usata. Risulta essere soprattutto una libreria stabile, anche a causa del numero basso di \glo{Issue}{issue} correttamente aperti sulla relativa pagina \glo{Github}{GitHub}.
\\Il requisito di funzionamento del prodotto su dispositivi mobile, nello specifico su tablet, implica che il prodotto stesso dovrebbe risultare di piccole dimensioni e computazionalmente leggero, per favorire un uso veloce e fluido anche su questa famiglia di dispositivi. React risulta una libreria adatta allo scopo in quanto leggera e pienamente supportata su dispositivi mobili. 