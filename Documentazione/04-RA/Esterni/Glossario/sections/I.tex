\section{I}

\gref{IDE}
\textit{Integrated Development Enviroment}. Software utilizzato per la scrittura di codice sorgente. Spesso aiuta il programmatore segnalando errori di sintassi, oltre a tutta una serie di strumenti e funzionalità di supporto allo sviluppo.


\gref{IEC}
\textit{International Electrotechnical Commission}. Organizzazione internazionale per la definizione di standard in materia di elettricità, elettronica e tecnologie correlate. Molti dei suoi standard sono definiti in collaborazione con l'\glo{ISO}{ISO}.

\gref{Indice Gulpease}
Indice di leggibilità di un testo tarato sulla lingua italiana. Rispetto ad altri ha il vantaggio di utilizzare la lunghezza delle parole in lettere anziché in sillabe, semplificandone il calcolo automatico. \\
I valori dell'Indice Gulpease sono compresi tra 0 e 100, dove il valore 100 indica la leggibilità più alta e 0 la leggibilità più bassa. In generale risulta che testi con un indice:
\begin{itemize}
	\item inferiore a 80 sono difficili da leggere per chi ha la licenza elementare;
	\item inferiore a 60 sono difficili da leggere per chi ha la licenza media;
	\item inferiore a 40 sono difficili da leggere per chi ha un diploma superiore.
\end{itemize}


\gref{Inspection}
Tecnica di analisi statica che consiste in una lettura dettagliata e mirata dei documenti o del codice, cercando errori specifici. Viene spesso abbinata ad una lista di controllo.

\gref{Ipertestuale}
Relativo a un ipertesto. Un ipertesto è un documento elettronico contenente un insieme di informazioni di natura per lo più testuale e grafica, ma anche integrabili con risorse di altri tipi. \\
Le informazioni sono raccolte in diverse aree collegate tra loro secondo una configurazione a grafo.

\gref{ISO}
\textit{International Organization for Standardization}. Organizzazione non governativa internazionale per la definizione di norme tecniche. 
L'uso degli standard aiuta la creazione di prodotti e servizi sicuri, affidabili e di buona qualità. Gli standard aiutano le imprese ad aumentare la produttività diminuendo gli errori e gli sprechi. Essi servono anche per salvaguardare i consumatori e gli utenti finali di prodotti e servizi, garantendo che i prodotti certificati siano conformi agli standard minimi internazionali. \\
Le norme ISO sono numerate e hanno un formato del tipo ISO <nnnn>:<yyyy> - <titolo>, dove:
\begin{itemize}
	\item <nnnn> è il numero della norma;
	\item <yyyy> l'anno di pubblicazione;
	\item <titolo> è una breve descrizione della norma.
\end{itemize}

\gref{Issue}
Un problema o questione importante da discutere.
Nei documenti riguardanti il progetto \progetto{} indica un qualche tipo di problema o errore.

\gref{Istanbul}
Strumento \glo{JavaScript}{JavaScript} per la rilevazione del code coverage del codice JavaScript.
