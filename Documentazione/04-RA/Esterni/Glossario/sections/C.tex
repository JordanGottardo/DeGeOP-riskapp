
\section{C}

\gref{Callback}
Funzione passata come argomento ad un'altra funzione. Ci si aspetta che quest'ultima esegua la prima.

\gref{Camel case}
Notazione di scrittura nata durante gli anni settanta. Le parole composte o le frasi vengono scritte unendo i termini tra di loro, lasciando le iniziali maiuscole. Viene spesso utilizzato per i nomi di variabili e metodi all'interno di codice sorgente.

\gref{Capitolato}
Atto allegato a un contratto d'appalto in cui vengono indicate modalità, costi e tempi di realizzazione dell'opera oggetto del contratto.

%\gref{Chai}
%Libreria \glo{JavaScript}{JavaScript} per la gestione delle asserzioni.

\gref{Ciclo di vita}
Insieme degli stati di maturità che un prodotto software assume dal suo concepimento al suo ritiro.

\gref{Cloud}
Paradigma di erogazione di risorse informatiche, come archiviazione, elaborazione e trasmissione di dati. Tali risorse vengono rese disponibili su richiesta tramite una connessione Internet.

\gref{CMM}
\textit{Capability Maturity Model}. Modello di sviluppo creato con lo scopo di migliorare i processi software di un'azienda.

\gref{Code coverage}
Misura utilizzata per descrivere il livello di copertura del codice sorgente a seguito dell'esecuzione di test.

\gref{Commit}
In ambito di \glo{Sistema di controllo di versione}{sistema di controllo di versione}, operazione con cui si aggiungono le modifiche pianificate al \glo{Repository}{repository}.

\gref{Componente}
\begin{enumerate}
	\item React: elemento che fa parte della gerarchia del DOM. L'elemento eredita dalla classe base astratta React.Component.
	\item Unità software dotata di una precisa identità e interfacce ben definite.
\end{enumerate}

\gref{Cross-device}
Possibilità di poter utilizzare lo stesso strumento software su più dispositivi, tipicamente con capacità hardware e dimensioni dello schermo differenti.

\gref{Cross-platform}
Possibilità di poter usare lo stesso strumento software su diversi sistemi operativi.

\gref{CSS}
\textit{Cascading Style Sheets}. Linguaggio che permette di definire lo stile e la formattazione di una pagina \glo{HTML}{HTML}. Permette di mantenere separate presentazione e contenuto. \\
La versione stabile più recente è CSS3.

