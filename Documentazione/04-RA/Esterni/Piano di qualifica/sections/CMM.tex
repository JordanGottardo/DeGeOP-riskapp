
\section{CMM}
\label{appendice A}
%\newcommand{\acp}{area chiave di processo}

Il \glo{CMM}{CMM} (Capability Maturity Model) è un modello che mira a migliorare e rifinire i processi software di un'organizzazione. Il modello descrive un percorso evolutivo a cinque livelli riguardante processi sempre più maturi e organizzati. \\
Il CMM è stato sviluppato dal \glo{SEI}{SEI} e promosso e finanziato dal Dipartimento della Difesa statunitense per valutare la qualità dei processi software delle organizzazioni che collaboravano con esso.

Il modello fornisce:
\begin{itemize}
	\item una base concettuale a cui appoggiarsi per valutare il livello dei processi;
\item un insieme di best practices consolidate negli anni da esperti e utilizzatori;
\item un linguaggio comune e una visione condivisa;
\item un metodo per definire un miglioramento in ambito organizzativo.
\end{itemize}

\subsection{Struttura}
Il modello è costituito da cinque aspetti:
\begin{itemize}
\item  \textbf{livelli di maturità:} il CMM identifica un processo continuo di maturazione a cinque livelli (in cui il maggiore è uno stato ideale dove i processi sono sistematicamente gestiti da una combinazione di ottimizzazione e miglioramento del processo);

\item  \textbf{area chiave di processo:} identifica un insieme di attività correlate che, quando eseguite assieme, raggiungono un insieme di obiettivi considerati importanti;

\item  \textbf{obiettivi:} gli obiettivi di un'area chiave di processo riassumono gli stati che devono sussistere affinché tale area sia implementata in modo efficace e duraturo. La quantità di obiettivi soddisfatti indica il livello di capability raggiunto dall'organizzazione in un dato livello di maturità. Gli obiettivi denotano l’ambito, i limiti e lo scopo di ogni area chiave di processo;

\item  \textbf{caratteristiche comuni:} includono le pratiche che implementano e regolamentano le aree chiave di processo. Esistono cinque tipi di caratteristiche comuni:
\begin{itemize}
	\item impegno nell'esecuzione;
	\item abilità nell'esecuzione;
	\item attività eseguite;
	\item misurazioni e analisi;
	\item verifica e implementazione.
\end{itemize}

\item  \textbf{pratiche chiave}: descrivono gli elementi di infrastruttura e prassi che contribuiscono all'implementazione e regolamentazione dell'area. 
\end{itemize}

\subsection{Livelli}
Sono presenti cinque livelli:
\begin{itemize}
\item \textbf{livello 1 - Iniziale:} i processi in questo livello hanno la tendenza ad essere non documentati e in uno stato di continuo cambiamento. Date queste caratteristiche, l'esito molto spesso dipende dallo sforzo dei singoli e non si considera essere ripetibile. I processi vengono riadattati di volta in volta, risultando caotici e scarsamente controllabili;

\item \textbf{livello 2 - Ripetibile:} I processi di questo livello sono generalmente ripetibili, eventualmente con buoni risultati. La disciplina, se presente, pur non essendo rigorosa, aiuta a sostenere i processi durante i periodi di elevato carico di lavoro;

\item \textbf{livello 3 - Definito:} i processi cominciano ad essere standardizzati, in quanto la disciplina è più rigorosa e la documentazione più completa. Inoltre sono soggetti ad un certo livello di miglioramento nel lungo periodo;

\item \textbf{livello 4 - Gestito:} i processi sono controllati quantitativamente in accordo alle metriche di processo prestabilite. L'amministrazione aziendale può adeguare e adattare i processi a particolari progetti senza perdite sostanziali di qualità o deviazioni dalle specifiche;

\item \textbf{livello 5 - Ottimizzato:} i processi in questo livello hanno come obbiettivo il miglioramento continuo delle loro performance attraverso miglioramenti tecnologici sia incrementali che innovativi.
\end{itemize}
