

\section{PDCA}
		

Il \glo{PDCA}{PDCA} (conosciuto anche come "Ciclo di Deming" o "Ciclo di Shewhart" ) è un metodo per la gestione delle attività di processo ripetibili e misurabili e per la manutenibilità dei processi stessi. È un metodo iterativo suddiviso in quattro \glo{Fase}{fasi} (Plan-Do-Check-Act, da cui l'acronimo) e assicura un non decremento della qualità ad ogni ciclo. Fissati degli obiettivi di miglioramento desiderati si iterano le attività previste dal PDCA fino al raggiungimento degli stessi. I miglioramenti ai quali si fa riferimento sono legati all'efficienza e all'efficacia. Migliorare l'efficienza significa usare meno risorse per fare lo stesso lavoro. Migliorare l'efficacia significa divenire più conformi alle aspettative.
% <<FIGURE ???>>

\subsection{Fasi}
Sono presenti quattro fasi:
\begin{itemize}
	\item  \textbf{Plan:} vengono definiti gli obiettivi di miglioramento, le strategie da utilizzare per perseguire tali obiettivi e il modo in cui queste verranno utilizzate. \\
	Per far ciò si svolgono i seguenti passi:
	\begin{enumerate}
		\item si svolge una prima fase di identificazione del problema (ad esempio un processo da migliorare) nella quale saranno raccolti dei dati in seguito a delle misurazioni; 
		\item viene analizzato il problema e vengono individuati gli aspetti negativi, decidendone la loro importanza e le priorità di intervento;
		\item vengono definiti gli obiettivi di massima in modo chiaro e quantitativo, indicando i benefici ottenibili con il loro raggiungimento. Vengono inoltre specificati i tempi necessari per la loro attuazione, gli indicatori e gli strumenti di controllo necessari.
	\end{enumerate}
		
	
	\item  \textbf{Do:} viene attuato ciò che è stato pianificato per risolvere il problema. Nello stesso tempo si devono anche raccogliere i dati necessari all'analisi che verrà svolta in seguito;
	
	\item  \textbf{Check:} consiste nel verificare i risultati ottenuti (per efficienza ed efficacia) in seguito all'attuazione delle strategie di miglioramento. Essi saranno analizzati e studiati (anche attraverso grafici e tabelle riassuntive) in modo tale da avere una visione chiara di quanto rilevato. Se gli obiettivi sono stati raggiunti, ovvero se è avvenuto un miglioramento, si può passare alla fase successiva; in caso contrario è necessario ripetere il ciclo PDCA sullo stesso problema, analizzando gli stadi del ciclo precedente e individuando le cause del mancato raggiungimento degli obiettivi stabiliti.
	
	L'esito del processo può essere di tre tipi:
	\begin{itemize}
		\item miglioramento secondo le aspettative;
		\item miglioramento superiore alle aspettative;
		\item miglioramento inferiore alle aspettative.
	\end{itemize}


\iffalse


% Check : [...] miglioramento. I risultati verrano analizzati e studiati anche attraverso grafici e tabelle riassuntive, in modo tale da avere una visione chiara di quanto rilevato. Se gli obiettivi [...] 
\fi 


	\item  \textbf{Act:} i miglioramenti individuati vengono regolamentati e integrati nello standard dell'organizzazione e tutti i membri del \glo{Gruppo}{gruppo} vengono informati e conseguentemente formati. Verrà quindi eseguita una nuova iterazione dell'intero ciclo. \\
	
\end{itemize}
