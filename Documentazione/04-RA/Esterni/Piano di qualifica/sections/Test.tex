\newpage

\section{Test}
\label{Test}
	I test, eseguiti tramite analisi dinamica, sono attività che servono a verificare che il software prodotto implementi le funzionalità richieste. Una caratteristica fondamentale dei test è la ripetibilità: i risultati che essi forniscono devono essere deterministici, in modo da eseguire azioni correttive in caso gli esiti non siano quelli attesi.
	%Da modificare in futuro
	Per tracciare i test eseguiti e i risultati ottenuti sarà necessario produrre dei log di facile consultazione.
	
	Le tabelle che descrivono i test utilizzano le seguenti abbreviazioni:
	\begin{itemize}
		\item \textbf{NI:} non implementato;
		\item \textbf{I:} implementato;
		\item \textbf{NS:} non soddisfatto;
		\item \textbf{S:} soddisfatto.
	\end{itemize}
	
	\subsection{Test di validazione}
I test di validazione, eseguiti durante il collaudo finale, servono a verificare che il software soddisfi le richieste del proponente.

\def\arraystretch{1.5}
\rowcolors{2}{D}{P}
\begin{longtable}{p{1.8cm}!{\VRule[1pt]}p{3.5cm}!{\VRule[1pt]}p{5.5cm}!{\VRule[1pt]}p{1cm}!{\VRule[1pt]}p{1cm}}
	\rowcolor{I}
	\color{white} \textbf{Test} & \color{white} \textbf{Descrizione} & \color{white} \textbf{Operazioni} & \color{white} \textbf{Stato} & \color{white} \textbf{Esito} \\ 
	\endfirsthead 
	\rowcolor{I} 
	\color{white} \textbf{Test} & \color{white} \textbf{Descrizione} & \color{white} \textbf{Operazioni} & \color{white} \textbf{Stato} & \color{white} \textbf{Esito} \\ 
	\endhead 
	TVDF24.4 & L'utente intende cambiare la modalità di visualizzazione della mappa & Viene richiesto di: \begin{enumerate} 
		\item aprire l'applicazione; 
		\item cliccare sul pulsante per cambiare la visualizzazione della mappa. 
	\end{enumerate} & N.I & N.S \\ 
	TVDF25 & L'utente intende avviare il tutorial & Viene richiesto di: \begin{enumerate} 
		\item aprire l'applicazione; 
		\item premere sul pulsante relativo al tutorial. 
	\end{enumerate} & N.I & N.S \\ 
	TVFF11.3 & L'utente intende compilare i dati dell'arco di tipo Trasporto. & Viene richiesto di: \begin{enumerate} 
		\item avviare la procedura di aggiunta arco; 
		\item selezionare che l'arco è di tipo Trasporto; 
		\item compilare i dati dell'arco nell'area informativa. 
	\end{enumerate} & N.I & N.S \\ 
	TVFF14.5 & L'utente intende modificare i dati dell'arco di tipo Trasporto. & Viene richiesto di: \begin{enumerate} 
		\item avviare la procedura di modifica arco; 
		\item modificare i dati dell'arco nell'area informativa. 
	\end{enumerate} & N.I & N.S \\ 
	TVFF26 & L'utente intende avviare l'assistente vocale. & Viene richiesto di: \begin{enumerate} 
		\item aprire l'applicazione; 
		\item premere sul pulsante relativo all'assistente vocale. 
	\end{enumerate} & N.I & N.S \\ 
	TVOF1.1 & L'utente intende disegnare il perimetro dell'asset sulla mappa. & Viene richiesto di: \begin{enumerate} 
		\item avviare la procedura di aggiunta asset; 
		\item tracciare il perimetro dell'asset sulla mappa utilizzando gli strumenti a disposizione. 
	\end{enumerate} & I & S \\ 
	TVOF1.2 & L'utente intende compilare i dati dell'asset. & Viene richiesto di: \begin{enumerate} 
		\item avviare la procedura di aggiunta asset; 
		\item compilare i campi richiesti nell'area informativa. 
	\end{enumerate} & I & S \\ 
	TVOF10 & L'utente intende eliminare un nodo. & Viene richiesto di: \begin{enumerate} 
		\item aprire l'applicazione; 
		\item selezionare un nodo per visualizzarne le informazioni; 
		\item premere sul pulsante "Elimina" nell'area informativa; 
		\item confermare l'eliminazione del nodo. 
	\end{enumerate} & I & S \\ 
	TVOF11 & L'utente intende aggiungere un nuovo arco. & Viene richiesto di: \begin{enumerate} 
		\item avviare l'applicazione:; 
		\item inserire un asset se non ne sono presenti; 
		\item inserire almeno due nodi; 
		\item cliccare sul pulsante "+"; 
		\item selezionare "Aggiungi arco"; 
		\item selezionare il nodo di origine dell'arco; 
		\item selezionare il nodo di destinazione dell'arco; 
		\item compilare i dati dell'arco; 
		\item confermare l'aggiunta dell'arco. 
	\end{enumerate} & I & S \\ 
	TVOF11.1 & L'utente intende disegnare un arco. & Viene richiesto di: \begin{enumerate} 
		\item avviare la procedura di aggiunta arco; 
		\item selezionare nodo di origine dell'arco; 
		\item selezionare nodo di destinazione dell'arco. 
	\end{enumerate} & I & S \\ 
	TVOF12 & L'utente intende visualizzare le informazioni di un arco. & Viene richiesto di: \begin{enumerate} 
		\item aprire l'applicazione; 
		\item inserire un arco se non ne sono presenti; 
		\item selezionare l'arco dalla mappa per visualizzarne le informazioni nell'area informativa. 
	\end{enumerate} & I & S \\ 
	TVOF14 & L'utente intende modificare un arco. & Viene richiesto di: \begin{enumerate} 
		\item aprire l'applicazione; 
		\item selezionare un arco dalla mappa per visualizzarne le informazioni; 
		\item cliccare il pulsante "Modifica" nell'area informativa; 
		\item modificare il nodo di destinazione o di origine o i dati dell'arco; 
		\item confermare la modifica. 
	\end{enumerate} & I & S \\ 
	TVOF14.1 & L'utente intende modificare il nodo di origine di un arco. & Viene richiesto di: \begin{enumerate} 
		\item avviare la procedura di modifica arco; 
		\item selezionare un arco dalla mappa per visualizzarne le informazioni; 
		\item cliccare sul pulsante ; 
		\item selezionare un nuovo nodo di origine dalla mappa. 
	\end{enumerate} & I & S \\ 
	TVOF14.2 & L'utente intende modificare il nodo di destinazione di un arco. & Viene richiesto di: \begin{enumerate} 
		\item avviare la procedura di modifica arco; 
		\item selezionare un arco dalla mappa per visualizzarne le informazioni; 
		\item cliccare sul pulsante "Modifica destinazione" nell'area informativa; 
		\item selezionare un nuovo nodo di destinazione dalla mappa. 
	\end{enumerate} & I & S \\ 
	TVOF15 & L'utente intende eliminare un arco. & Viene richiesto di: \begin{enumerate} 
		\item aprire l'applicazione; 
		\item selezionare un arco per visualizzarne le informazioni; 
		\item premere sul pulsante "Elimina" nell'area informativa; 
		\item confermare l'eliminazione dell'arco. 
	\end{enumerate} & I & S \\ 
	TVOF16 & L'utente intende aggiungere un nuovo scenario di danno. & Viene richiesto di: \begin{enumerate} 
		\item aprire l'applicazione; 
		\item premere sul pulsante "+"; 
		\item selezionare "Aggiungi scenario"; 
		\item disegnare lo scenario di danno sulla mappa; 
		\item compilare i dati dello scenario nell'area informativa; 
		\item confermare l'inserimento. 
	\end{enumerate} & I & S \\ 
	TVOF16.1 & L'utente intende compilare le informazioni dello scenario di danno. & Viene richiesto di: \begin{enumerate} 
		\item avviare la procedura di inserimento scenario; 
		\item compilare le informazioni nell'area informativa. 
	\end{enumerate} & I & S \\ 
	TVOF16.1.7 & L'utente intende disegnare lo scenario di danno su mappa. & Viene richiesto di: \begin{enumerate} 
		\item avviare la procedura di inserimento scenario; 
		\item disegnare lo scenario di danno su mappa con gli strumenti a disposizione. 
	\end{enumerate} & I & S \\ 
	TVOF17 & L'utente intende visualizzare le informazioni di uno scenario di danno. & Viene richiesto di: \begin{enumerate} 
		\item aprire l'applicazione; 
		\item selezionare la tab "Scenari" nell'area informativa; 
		\item selezionare lo scenario di danno da visualizzare dall'area informativa. 
	\end{enumerate} & I & S \\ 
	TVOF19 & L'utente intende modificare uno scenario. & Viene richiesto di: \begin{enumerate} 
		\item aprire l'applicazione; 
		\item selezionare uno scenario di danno per visualizzarne le informazioni; 
		\item cliccare sul pulsante "Modifica" nell'area informativa; 
		\item modificare il disegno su mappa o i dati nell'area informativa; 
		\item confermare la modifica dello scenario. 
	\end{enumerate} & I & S \\ 
	TVOF19.1 & L'utente intende modificare le informazioni di uno scenario di danno. & Viene richiesto di: \begin{enumerate} 
		\item avviare la procedura di modifica scenario di danno; 
		\item modificare le informazioni dello scenario nell'area informativa. 
	\end{enumerate} & I & S \\ 
	TVOF19.1.7 & L'utente intende modificare il disegno di uno scenario di danno. & Viene richiesto di: \begin{enumerate} 
		\item avviare la procedura di modifica scenario di danno; 
		\item modificare il disegno dello scenario su mappa con gli strumenti a disposizione. 
	\end{enumerate} & I & S \\ 
	TVOF2 & L'utente intende visualizzare le informazioni di un asset. & Viene richiesto di: \begin{enumerate} 
		\item aprire l'applicazione; 
		\item cliccare sul perimetro di un asset. 
	\end{enumerate} & I & S \\ 
	TVOF20 & L'utente intende eliminare uno scenario di danno. & Viene richiesto di: \begin{enumerate} 
		\item aprire l'applicazione; 
		\item selezionare la tab "Scenari" nell'area informativa; 
		\item selezionare uno scenario nell'area informativa per visualizzarne le informazioni; 
		\item cliccare sul pulsante "Elimina" nell'area informativa; 
		\item confermare l'eliminazione dello scenario. 
	\end{enumerate} & I & S \\ 
	TVOF21 & L'utente intende avviare l'analisi di danno. & Viene richiesto di: \begin{enumerate} 
		\item aprire l'applicazione; 
		\item disegnare il processo produttivo sulla mappa, inserendo almeno un asset; 
		\item cliccare sul tab "Analisi" nell'area informativa; 
		\item selezionare su quali scenari di danno vuole calcolare l'analisi; 
		\item premere il pulsante "Avvia analisi". 
	\end{enumerate} & I & S \\ 
	TVOF22 & L'utente intende visualizzare il risultato di un'analisi di danno precedentemente eseguita. & Viene richiesto di: \begin{enumerate} 
		\item aprire l'applicazione; 
		\item cliccare sul tab "Analisi" nell'area informativa; 
		\item selezionare l'analisi di danno di cui vuole visualizzare i risultati. 
	\end{enumerate} & I & S \\ 
	TVOF24.1 & L'utente intende aumentare il livello di ingrandimento della mappa. & Viene richiesto di: \begin{enumerate} 
		\item aprire l'applicazione; 
		\item cliccare sul pulsante che aumenta l'ingrandimento della mappa oppure scrollare verso l'alto con la rotellina del mouse oppure effettuare la gesture "pinch in" su tablet. 
	\end{enumerate} & I & S \\ 
	TVOF24.2 & L'utente intende diminuire il livello di ingrandimento della mappa. & Viene richiesto di: \begin{enumerate} 
		\item aprire l'applicazione; 
		\item cliccare sul pulsante che diminuisce l'ingrandimento della mappa oppure scrollare verso il basso con la rotellina del mouse oppure effettuare la gesture "pinch out" su tablet. 
	\end{enumerate} & I & S \\ 
	TVOF24.3 & L'utente intende spostarsi sulla mappa. & Viene richiesto di: \begin{enumerate} 
		\item aprire l'applicazione; 
		\item cliccare e trascinare su un punto della mappa oppure effettuare la gesture "pan" su tablet. 
	\end{enumerate} & I & S \\ 
	TVOF4 & L'utente intende modificare un asset. & Viene richiesto di: \begin{enumerate} 
		\item aprire l'applicazione; 
		\item selezionare il perimetro di un asset dalla mappa per visualizzarne le informazioni; 
		\item cliccare sul pulsante "Modifica" nell'area informativa; 
		\item modificare il disegno o i dati dell'asset; 
		\item confermare la modifica. 
	\end{enumerate} & I & S \\ 
	TVOF4.1 & L'utente intende modificare il disegno di un asset. & Viene richiesto di: \begin{enumerate} 
		\item avviare la procedura di modifica asset; 
		\item modificare il perimetro dell'asset con gli strumenti a disposizione. 
	\end{enumerate} & I & S \\ 
	TVOF4.2 & L'utente intende modificare i dati dell'asset. & Viene richiesto di: \begin{enumerate} 
		\item avviare la procedura di modifica asset; 
		\item modificare i dati dell'asset nell'area informativa. 
	\end{enumerate} & I & S \\ 
	TVOF47 & L'utente intende eliminare i risultati di un'analisi di danno precedentemente calcolata. & Viene richiesto di: \begin{enumerate} 
		\item aprire l'applicazione; 
		\item cliccare sul tab "Analisi" nell'area informativa; 
		\item selezionare un'analisi di danno per visualizzarne i risultati; 
		\item cliccare sul pulsante "Elimina" nell'area informativa; 
		\item confermare l'eliminazione dell'analisi. 
	\end{enumerate} & I & S \\ 
	TVOF5 & L'utente intende eliminare un asset. & Viene richiesto di: \begin{enumerate} 
		\item aprire l'applicazione; 
		\item selezionare il perimetro di un asset per modificarne le informazioni; 
		\item cliccare sul pulsante "Elimina" nell'area informativa; 
		\item confermare di voler eliminare un asset. 
	\end{enumerate} & I & S \\ 
	TVOF6 & L'utente intende inserire un nuovo nodo. & Viene richiesto di: \begin{enumerate} 
		\item aprire l'applicazione; 
		\item inserire almeno un asset, se non ne sono presenti; 
		\item cliccare sul pulsante "+"; 
		\item selezionare "Aggiungi nodo"; 
		\item selezionare l'asset di appartenenza del nodo sulla mappa; 
		\item posizionare il nodo all'interno dell'asset; 
		\item compilare i dati del nodo; 
		\item confermare l'aggiunta del nodo. 
	\end{enumerate} & I & S \\ 
	TVOF6.1 & L'utente intende selezionare l'asset di appartenenza del nodo. & Viene richiesto di: \begin{enumerate} 
		\item avviare la procedura di aggiunta nodo; 
		\item cliccare sul perimetro di un asset sulla mappa. 
	\end{enumerate} & I & S \\ 
	TVOF6.12 & L'utente intende compilare i dati del nodo. & Viene richiesto di: \begin{enumerate} 
		\item avviare la procedura di aggiunta nodo; 
		\item selezionare l'asset di appartenenza del nodo e posizionarlo all'interno del perimetro; 
		\item compilare i dati del nodo nell'area informativa. 
	\end{enumerate} & I & S \\ 
	TVOF6.2 & L'utente intende posizionare un nodo all'interno di un asset. & Viene richiesto di: \begin{enumerate} 
		\item avviare la procedura di aggiunta nodo; 
		\item selezionare l'asset di appartenenza del nodo; 
		\item posizionare il nodo all'interno dell'asset. 
	\end{enumerate} & I & S \\ 
	TVOF7 & L'utente intende visualizzare le informazioni di un nodo. & Viene richiesto di: \begin{enumerate} 
		\item avviare l'applicazione; 
		\item selezionare un nodo dalla mappa. 
	\end{enumerate} & I & S \\ 
	TVOF9 & L'utente intende modificare un nodo. & Viene richiesto di: \begin{enumerate} 
		\item aprire l'applicazione; 
		\item selezionare un nodo dalla mappa per visualizzarne le informazioni; 
		\item modificare l'asset di appartenenza o i dati del nodo; 
		\item confermare le modifiche. 
	\end{enumerate} & I & S \\ 
	TVOF9.1 & L'utente intende modificare l'asset di appartenenza del nodo. & Viene richiesto di: \begin{enumerate} 
		\item avviare la procedura di modifica nodo; 
		\item cliccare sul pulsante "Modifica asset di appartenenza"; 
		\item selezionare il perimetro di un nuovo asset dalla mappa; 
		\item posizionare il nodo all'interno dell'asset. 
	\end{enumerate} & I & S \\ 
	TVOF9.12 & L'utente intende modificare i dati del nodo. & Viene richiesto di: \begin{enumerate} 
		\item avviare la procedura di modifica nodo; 
		\item modificare i dati nell'area informativa. 
	\end{enumerate} & I & S \\ 
	\rowcolor{white}
	\caption{Riepilogo test di validazione}
\end{longtable}

	\subsection{Test di sistema}
		I test di sistema servono a verificare il corretto funzionamento delle componenti dell'intero sistema.
		
		\def\arraystretch{1.5}
		\rowcolors{2}{D}{P}
		\begin{longtable}{p{1.5cm}!{\VRule[1pt]}p{5cm}!{\VRule[1pt]}p{1cm}!{\VRule[1pt]}p{1cm}}
			\rowcolor{I}
			\color{white} \textbf{Test} & \color{white} \textbf{Descrizione} & \color{white} \textbf{Stato} & \color{white} \textbf{Esito} \\ 
			\endfirsthead 
			\rowcolor{I} 
			\color{white} \textbf{Test} & \color{white} \textbf{Descrizione} & \color{white} \textbf{Stato} & \color{white} \textbf{Esito} \\ 
			\endhead 
			TSFF11.3 & Viene verificato che il sistema permetta di compilare i dati dell'arco di tipo Trasporto. & N.I & N.S \\ 
			TSFF14.5 & Viene verificato che il sistema permetta la modifica dei dati dell'arco di tipo Trasporto. & N.I & N.S \\ 
			TSFF16 & Viene verificato che il sistema permetta l'aggiunta di un nuovo scenario di danno. & I & S \\ 
			TSFF16.1 & Viene verificato che il sistema permetta la compilazione delle informazioni di uno scenario di danno durante l'inserimento di uno scenario. & I & S \\ 
			TSFF16.1.7 & Viene verificato che il sistema permetta il disegno di uno scenario di danno su mappa. & I & S \\ 
			TSFF17 & Viene verificato che il sistema permetta la visualizzazione di uno scenario di danno. & I & S \\ 
			TSFF19 & Viene verificato che il sistema permetta la modifica di uno scenario di danno precedentemente inserito. & I & S \\ 
			TSFF19.1 & Viene verificato che il sistema permetta la modifica delle informazioni dello scenario di danno. & I & S \\ 
			TSFF19.1.7 & Viene verificato che il sistema permetta la modifica del disegno dello scenario di danno. & I & S \\ 
			TSFF20 & Viene verificato che il sistema permetta l'eliminazione di uno scenario di danno precedentemente inserito. & N.I & N.S \\ 
			TSFF21 & Viene verificato che il sistema permetta l'avvio di un'analisi di danno. & N.I & N.S \\ 
			TSFF22 & Viene verificato che il sistema permetta la visualizzazione del risultato di un'analisi di danno precedentemente eseguita su mappa. & I & S \\ 
			TSFF24.4 & Viene verificato che il sistema permetta di cambiare la modalità di visualizzazione della mappa. & N.I & N.S \\ 
			TSFF25 & Viene verificato che il sistema permetta l'avvio e la fruizione del tutorial. & N.I & N.S \\ 
			TSFF26 & Viene verificato che il sistema permetta l'avvio e la fruizione dell'assistente vocale. & I & S \\ 
			TSFF47 & Viene verificato che il sistema permetta l'eliminazione dei risultati di un'analisi di danno precedentemente calcolata. & I & S \\ 
			TSOF1 & Viene verificato che il sistema permetta l'aggiunta di un nuovo asset. & I & S \\ 
			TSOF1.1 & Viene verificato che il sistema permetta di disegnare il perimetro dell'asset su mappa durante l'aggiunta di un nuovo asset. & I & S \\ 
			TSOF1.2 & Viene verificato che il sistema permetta di compilare i dati dell'asset durante l'aggiunta dell'asset. & I & S \\ 
			TSOF10 & Viene verificato che il sistema permetta l'eliminazione di un nodo precedentemente inserito. & I & S \\ 
			TSOF11 & Viene verificato che il sistema permetta l'aggiunta di un nuovo arco. & I & S \\ 
			TSOF11.1 & Viene verificato che il sistema permetta di disegnare un arco, scegliendo nodo di origine e di destinazione. & I & S \\ 
			TSOF12 & Viene verificato che il sistema permetta la visualizzazione delle informazioni di un arco precedentemente inserito. & I & S \\ 
			TSOF14 & Viene verificato che il sistema permetta la modifica di un arco precedentemente inserito. & I & S \\ 
			TSOF14.1 & Viene verificato che il sistema permetta la modifica del nodo di origine dell'arco. & I & S \\ 
			TSOF14.2 & Viene verificato che il sistema permetta la modifica del nodo di destinazione dell'arco. & I & S \\ 
			TSOF15 & Viene verificato che il sistema permetta l'eliminazione di un arco precedentemente inserito. & I & S \\ 
			TSOF2 & Viene verificato che il sistema permetta la visualizzazione di un asset precedentemente inserito. & I & S \\ 
			TSOF24.1 & Viene verificato che il sistema permetta l'aumento del livello di ingrandimento della mappa. & I & S \\ 
			TSOF24.2 & Viene verificato che il sistema permetta la diminuzione del livello di ingrandimento della mappa. & I & S \\ 
			TSOF24.3 & Viene verificato che il sistema permetta lo spostamento sulla mappa. & I & S \\ 
			TSOF4 & Viene verificato che il sistema permetta la modifica di un asset precedentemente inserito. & I & S \\ 
			TSOF4.1 & Viene verificato che il sistema permetta la modifica del perimetro di un asset precedentemente inserito. & I & S \\ 
			TSOF4.2 & Viene verificato che il sistema permetta la modifica dei dati di un asset precedentemente inserito. & I & S \\ 
			TSOF5 & Viene verificato che il sistema permetta l'eliminazione di un asset precedentemente inserito. & I & S \\ 
			TSOF6 & Viene verificato che il sistema permetta l'aggiunta di un nuovo nodo. & I & S \\ 
			TSOF6.1 & Viene verificato che il sistema permetta la selezione di un asset di appartenenza durante l'aggiunta di un nuovo nodo. & I & S \\ 
			TSOF6.12 & Viene verificato che il sistema permetta la compilazione dei dati del nodo durante l'aggiunta di un nuovo nodo. & I & S \\ 
			TSOF6.2 & Viene verificato che il sistema permetta il posizionamento di un nodo all'interno del perimetro di un asset durante l'aggiunta di un nuovo nodo. & I & S \\ 
			TSOF7 & Viene verificato che il sistema permetta la visualizzazione delle informazioni di un nodo precedentemente inserito. & I & S \\ 
			TSOF9 & Viene verificato che il sistema permetta la modifica di un nodo precedentemente inserito. & I & S \\ 
			TSOF9.1 & Viene verificato che il sistema permetta di modificare l'asset di appartenenza di un nodo. & I & S \\ 
			TSOF9.12 & Viene verificato che il sistema permetta la modifica dei dati del nodo. & I & S \\ 
			\rowcolor{white}
			\caption{Riepilogo test di sistema}
		\end{longtable}
		
	\subsection{Test di integrazione}
		I test di integrazione servono a verificare il corretto funzionamento di più unità. Più precisamente, l'obiettivo è quello di testare i vari \glo{Package}{package}, sia singolarmente che nel loro insieme.
\def\arraystretch{1.5}
\rowcolors{2}{D}{P}
\begin{longtable}{p{1cm}!{\VRule[1pt]}p{3cm}!{\VRule[1pt]}p{5cm}!{\VRule[1pt]}p{1.5cm}!{\VRule[1pt]}p{1.5cm}}
	\rowcolor{I}
	\color{white} \textbf{Test} & \color{white} \textbf{Package} & \color{white} \textbf{Descrizione} & \color{white} \textbf{Stato} & \color{white} \textbf{Esito} \\ 
	\endfirsthead
	\color{white} \textbf{Test} & \color{white} \textbf{Package} & \color{white} \textbf{Descrizione} & \color{white} \textbf{Stato} & \color{white} \textbf{Esito} \\ 
	\endhead
	TI1 & DeGeOP & Viene verificato che l’applicazione Web carichi correttamente le librerie JavaScript utilizzate. & N.I & N.S\\
	TI2 & CallManagerPkg & Viene verificato che sia funzionante il collegamento del gestore delle chiamate con il server RiskApp & I & S\\
	TI3 & CallManagerPkg & Viene verificato che i dati inviati dal gestore delle chiamate siano salvati corretamente sul server RiskApp & I & S\\
	TI4 & ActionCreatorsPkg & Viene verificato che l’Action Creators riceva correttamente gli input dalla View e crei le action ad essi associate. & I & S\\
	TI5 & ActionCreatorsPkg & Viene verificato che il gestore delle chiamate invii una richiesta all'Action Creators per una creazione di un'azione. & I & S\\
	TI6 & ActionCreatorsPkg & Viene verificato che l'Action Creators carichi correttamente lo store & I & S\\
	TI7 & StorePkg & Viene verificato che lo Store comunichi con il Reducer & I & S\\
	TI8 & ReducerPkg & Viene verificato che il Reducer gestisca correttamente tutte le azioni inviate allo store. & I & S\\
	TI9 & ViewPkg & Viene verificato che la View funzioni correttamente permettendo il caricamento e la visualizzazione della pagina & I & S\\
	\rowcolor{white}
	\caption{Riepilogo test di integrazione}
\end{longtable}

	
		
		
	\subsection{Test di unità}
		I test di unità servono a verificare il corretto funzionamento della singola unità, ovvero della più piccola parte di lavoro realizzabile dal singolo programmatore.
		
	% v: 2
	\def\arraystretch{1.5}
	\rowcolors{2}{D}{P}
	\begin{longtable}{p{1cm}!{\VRule[1pt]}p{9cm}!{\VRule[1pt]}p{1cm}!{\VRule[1pt]}p{1cm}}
		\rowcolor{I}
		\color{white} \textbf{Test} & \color{white} \textbf{Descrizione e componenti} & \color{white} \textbf{Stato} & \color{white} \textbf{Esito} \\ 
		\endfirsthead
		\color{white} \textbf{Test} & \color{white} \textbf{Descrizione e componenti} & \color{white} \textbf{Stato} & \color{white} \textbf{Esito} \\ 
		\endhead
		TU1 & Viene verficato che il concretePolygon sia creato correttamente.
		\vskip 0.6em Metodi testati:
		\begin{itemize}[leftmargin=*,noitemsep,nosep]
			\item PolygonPkg::ConcretePolygonFactory.createPolygon().
		\end{itemize}
		& I & S\\
		TU2 & Viene verificato che i campi di un oggetto Asset vengano validati o meno.
		\vskip 0.6em Metodi testati:
		\begin{itemize}[leftmargin=*,noitemsep,nosep]
			\item ProcessPkg::Asset.assetIsValid();
			\item ProcessPkg::Asset.testDescriptionValidation();
			\item ProcessPkg::Asset.testNameValidation();
			\item ProcessPkg::Asset.testSurfaceValidation();
			\item ProcessPkg::Asset.testUnitValueValidation().
		\end{itemize}
		& I & S\\
		TU3 & Viene verificato che i campi di un oggetto Node vengano validati o meno.
		\vskip 0.6em Metodi testati:
		\begin{itemize}[leftmargin=*,noitemsep,nosep]
			\item ProcessPkg::Node.nodeIsValid();
			\item ProcessPkg::Node.testNameValidation().
		\end{itemize}
		& I & S\\
		TU4 & Viene verificato che i campi di un oggetto ResourceNode vengano validati o meno.
		\vskip 0.6em Metodi testati:
		\begin{itemize}[leftmargin=*,noitemsep,nosep]
			\item ProcessPkg::ResourceNode.nodeIsValid().
		\end{itemize}
		& I & S\\
		TU5 & Viene verificato che i campi di un oggetto ExitNode vengano validati o meno.
		\vskip 0.6em Metodi testati:
		\begin{itemize}[leftmargin=*,noitemsep,nosep]
			\item ProcessPkg::ExitNode.nodeIsValid().
		\end{itemize}
		& I & S\\
		TU6 & Viene verificato che i campi di un oggetto MachineNode vengano validati o meno.
		\vskip 0.6em Metodi testati:
		\begin{itemize}[leftmargin=*,noitemsep,nosep]
			\item ProcessPkg::MachineNode.nodeIsValid();
			\item ProcessPkg::MachineNode.testCapacityValidation();
			\item ProcessPkg::MachineNode.testProcessingTimeValidation();
			\item ProcessPkg::MachineNode.testValueValidation().
		\end{itemize}
		& I & S\\
		TU7 & Viene verificato che i campi di un oggetto QueueNode vengano validati o meno.
		\vskip 0.6em Metodi testati:
		\begin{itemize}[leftmargin=*,noitemsep,nosep]
			\item ProcessPkg::QueueNode.nodeIsValid();
			\item ProcessPkg::QueueNode.testCapacityValidation().
		\end{itemize}
		& I & S\\
		TU8 & Viene verificato che i campi di un oggetto SourceNode vengano validati o meno.
		\vskip 0.6em Metodi testati:
		\begin{itemize}[leftmargin=*,noitemsep,nosep]
			\item ProcessPkg::SourceNode.nodeIsValid();
			\item ProcessPkg::SourceNode.testLeadTimeValidation().
		\end{itemize}
		& I & S\\
		TU9 & Viene verificato che l'AssetReducer esegua correttamente l'azione ricevuta sulla lista di Asset.
		\vskip 0.6em Metodi testati:
		\begin{itemize}[leftmargin=*,noitemsep,nosep]
			\item ReducerPkg::AssetReducer.assetReducer().
		\end{itemize}
		& I & S\\
		TU10 & Viene verificato che l'NodeReducer correttamente l'azione ricevuta sulla lista di Node.
		\vskip 0.6em Metodi testati:
		\begin{itemize}[leftmargin=*,noitemsep,nosep]
			\item ReducerPkg::NodeReducer.nodeReducer().
		\end{itemize}
		& I & S\\
		TU11 & Viene verificato che l'EdgeReducer esegua correttamente l'azione ricevuta sulla lista di Edge.
		\vskip 0.6em Metodi testati:
		\begin{itemize}[leftmargin=*,noitemsep,nosep]
			\item ReducerPkg::EdgeReducer.edgeReducer().
		\end{itemize}
		& I & S\\
		TU12 & Viene verificato che l'OptionReducer esegua correttamente l'azione ricevuta sull'oggetto Options.
		\vskip 0.6em Metodi testati:
		\begin{itemize}[leftmargin=*,noitemsep,nosep]
			\item ReducerPkg::OptionReducer.optionReducer().
		\end{itemize}
		& I & S\\
		TU13 & Viene verificato che il Reducer invochi il reducer corretto rispetto all'azione ricevuta.
		\vskip 0.6em Metodi testati:
		\begin{itemize}[leftmargin=*,noitemsep,nosep]
			\item ReducerPkg::Reducer.generalReducer().
		\end{itemize}
		& I & S\\
		TU14 & Viene verificato che l'ActionCreator crei correttamente le azioni relative ad un oggetto Asset.
		\vskip 0.6em Metodi testati:
		\begin{itemize}[leftmargin=*,noitemsep,nosep]
			\item ActionCreatorsPkg::AssetActionCreator.deleteAsset();
			\item ActionCreatorsPkg::AssetActionCreator.editAsset();
			\item ActionCreatorsPkg::AssetActionCreator.insertAsset().
		\end{itemize}
		& I & S\\
		TU15 & Viene verificato che l'NodeCreator crei correttamente le azioni relative ad un oggetto Node.
		\vskip 0.6em Metodi testati:
		\begin{itemize}[leftmargin=*,noitemsep,nosep]
			\item ActionCreatorsPkg::NodeActionCreator.deleteNode();
			\item ActionCreatorsPkg::NodeActionCreator.editNode();
			\item ActionCreatorsPkg::NodeActionCreator.insertNode().
		\end{itemize}
		& I & S\\
		TU16 & Viene verificato che l'EdgeCreator crei correttamente le azioni relative ad un oggetto Edge.
		\vskip 0.6em Metodi testati:
		\begin{itemize}[leftmargin=*,noitemsep,nosep]
			\item ActionCreatorsPkg::EdgeActionCreator.deleteEdge();
			\item ActionCreatorsPkg::EdgeActionCreator.editEdge();
			\item ActionCreatorsPkg::EdgeActionCreator.insertEdge().
		\end{itemize}
		& I & S\\
		TU17 & Viene verificato che l'OptionCreator crei correttamente le azioni relative all'oggetto Options.
		\vskip 0.6em Metodi testati:
		\begin{itemize}[leftmargin=*,noitemsep,nosep]
			\item ActionCreatorsPkg::OptionActionCreator.insertOptions().
		\end{itemize}
		& I & S\\
		TU19 & Viene verificato che DeGeOPView modifichi correttamente il suo stato tramite il metodo handleChange.
		\vskip 0.6em Metodi testati:
		\begin{itemize}[leftmargin=*,noitemsep,nosep]
			\item DeGeOPViewPkg::DeGeOPView.handleChange().
		\end{itemize}
		& I & S\\
		TU20 & Viene verificato che DeGeOPView esegua le chiamate allo store con gli oggetti corretti.
		\vskip 0.6em Metodi testati:
		\begin{itemize}[leftmargin=*,noitemsep,nosep]
			\item DeGeOPViewPkg::DeGeOPView.emitGeneralAction().
		\end{itemize}
		& I & S\\
		TU21 & Viene verificato che DeGeOPView crei la sidebar corretta rispetto al suo stato.
		\vskip 0.6em Metodi testati:
		\begin{itemize}[leftmargin=*,noitemsep,nosep]
			\item DeGeOPViewPkg::DeGeOPView.changeSidebarType().
		\end{itemize}
		& I & S\\
		TU22 & Viene verificato che AssetSidebar venga renderizzata con le proprietà corrette.
		\vskip 0.6em Metodi testati:
		\begin{itemize}[leftmargin=*,noitemsep,nosep]
			\item SidebarPkg::AssetSidebar.render().
		\end{itemize}
		& I & S\\
		TU23 & Viene verificato che NodeSidebar venga renderizzata con le proprietà corrette.
		\vskip 0.6em Metodi testati:
		\begin{itemize}[leftmargin=*,noitemsep,nosep]
			\item SidebarPkg::NodeSidebar.render().
		\end{itemize}
		& I & S\\
		TU24 & Viene verificato che EdgeSidebar venga renderizzata con le proprietà corrette.
		\vskip 0.6em Metodi testati:
		\begin{itemize}[leftmargin=*,noitemsep,nosep]
			\item SidebarPkg::EdgeSidebar.render().
		\end{itemize}
		& I & S\\
		TU25 & Viene verificato che ScenarioSidebar venga renderizzata con le proprietà corrette.
		\vskip 0.6em Metodi testati:
		\begin{itemize}[leftmargin=*,noitemsep,nosep]
			\item SidebarPkg::ScenarioSidebar.render().
		\end{itemize}
		& I & S\\
		TU26 & Viene verificato che AnalysisSidebar venga renderizzata con le proprietà corrette.
		\vskip 0.6em Metodi testati:
		\begin{itemize}[leftmargin=*,noitemsep,nosep]
			\item SidebarPkg::AnalysisSidebar.render().
		\end{itemize}
		& I & S\\
		TU28 & Viene verificato che TwoButtons venga renderizzata con le proprietà corrette.
		\vskip 0.6em Metodi testati:
		\begin{itemize}[leftmargin=*,noitemsep,nosep]
			\item ButtonsPkg::TwoButtons.render().
		\end{itemize}
		& I & S\\
		TU29 & Viene verificato che ThreeButtons venga renderizzata con le proprietà corrette.
		\vskip 0.6em Metodi testati:
		\begin{itemize}[leftmargin=*,noitemsep,nosep]
			\item ButtonsPkg::ThreeButtons.render().
		\end{itemize}
		& I & S\\
		\rowcolor{white}
		\caption{Riepilogo test di unità}
	\end{longtable}
	
		