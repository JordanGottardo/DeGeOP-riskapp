\documentclass[a4paper,11pt]{article}
\usepackage{../../../Template/zemplateVerb}
%\includeGlossario

%\externaldocument{path/to/glossario}
\docVerbV{6}
\docTitle{Verbale esterno - \getdocVerbV}
\docVersion{1.0.0}
\docCreationDate{\frmdata{07}{08}{2017}}
\docLastUpdateDate{\frmdata{07}{08}{2017}}
\docStatus{Approvato}
\docEditors{Giovanni Prete}
\docVerificators{Leonardo Brutesco}
\docApprovers{Marco Pasqualini}
\docUse{Esterno}
\docDestination{\Tullio \\ & \Cardin \\ & \zephyrus \\ & \riskapp}
\docJournal{
	1.0.0 & \frmdata{07}{08}{2017} & Marco Pasqualini & \responsabile  & Approvazione \\
	0.1.0 & \frmdata{07}{08}{2017} & Leonardo Brutesco & \verificatore & Verifica documento\\
	0.0.1 & \frmdata{07}{08}{2017} & Giovanni Prete & \programmatore & Stesura documento\\
}

\begin{document}
	\section{Estremi della riunione}
	\begin{itemize}
		\item \textbf{data:} \frmdata{07}{08}{2017};
		\item \textbf{ora inizio:} \frmora{16}{00};
		\item \textbf{ora fine:} \frmora{17}{00};
		\item \textbf{luogo:} chat condivisa con il proponente su Slack;
		\item \textbf{segretario:} Giovanni Prete;
		\item \textbf{partecipanti:}
		\begin{itemize}
			 \item Jordan Gottardo;
			 \item Giovanni Prete;
			 \item Giulia Petenazzi;
			 \item Daniel De Gaspari;
			 \item Leonardo Brutesco;
			 \item Marco Pasqualini;
			 \item Pierpaolo Toniolo (proponente).
		\end{itemize}
		\item \textbf{assenti:}
			\begin{itemize}
				\item nessuno.
			\end{itemize}
	\end{itemize}
	\section{Ordine del giorno}
		\begin{itemize}
			\item discussione riguardante la chiamata REST di analisi di scenari.
		\end{itemize}
	\section{Verbale della riunione}
	\begin{itemize}
		\item illustrazione dei problemi sorti nella chiamata di analisi di scenari \riskapp{} utilizzando le nuove API fornite da \riskapp;
		\item discusse possibili soluzioni con il proponente.
	\end{itemize}
	\section{Decisioni prese}
	\begin{itemize}
		\itemVE accordata la sostituzione della chiamata REST al server di \riskapp{} per l'avvio delle analisi con una chiamata a una funzione mock che ne simuli il funzionamento. Il mock in particolare consentirà alle altre parti dell'applicazione di agire su risultati di analisi fittizi in maniera trasparente.
	\end{itemize}
\end{document}
