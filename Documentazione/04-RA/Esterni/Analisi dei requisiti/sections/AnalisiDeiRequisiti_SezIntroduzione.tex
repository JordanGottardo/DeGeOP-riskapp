\section{Introduzione}
	\subsection {Scopo del documento}
	Lo scopo del documento è definire i casi d'uso e i requisiti del prodotto emersi durante lo studio del capitolato C3 e dalle riunioni con \riskapp.
	\subsection {Scopo del prodotto}
	\introScopo
	\subsection {Glossario}
	\introGlossario\\
    Nelle sezioni riguardanti i casi d'uso, i requisiti ed il loro tracciamento i termini del glossario troppo frequenti non saranno evidenziati per non compromettere la leggibilità del documento.
	\subsection {Riferimenti}
		\subsubsection{Riferimenti normativi}
			\begin{itemize}
				\item \ndpv.
			\end{itemize}
		\subsubsection{Riferimenti informativi}
			\begin{itemize}
				\item \textbf{\glo{Capitolato}{capitolato} d'appalto C3:} \progetto: A Designer and Geo-localizer Web App for Organizational Plants. Reperibile all'indirizzo:\\ \url{http://www.math.unipd.it/~tullio/IS-1/2016/Progetto/C3.pdf} ;
				\item \sdfv;
				\item \textbf{Guide to the Software Engineering Body of Knowledge: IEEE Computer Society. Software Engineering Coordinating Committee (Versione 2004):}
				\begin{itemize}
					\item \textbf{Chapter 2:} Software Requirements.
				\end{itemize}
				\item \textbf{slide del corso di Ingegneria del Software - Diagrammi dei casi d'uso:}\\
				\url{http://www.math.unipd.it/~tullio/IS-1/2016/Dispense/E01b.pdf}.
			\end{itemize}
