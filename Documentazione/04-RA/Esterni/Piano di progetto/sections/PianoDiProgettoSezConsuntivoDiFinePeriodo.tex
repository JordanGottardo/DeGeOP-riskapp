%\section {Consuntivo}

\section {Consuntivo}
	\subsection {Introduzione}
	In questa sezione viene presentato il bilancio orario ed economico del progetto a consuntivo. Per ogni \glo{Periodo}{periodo} viene steso un consuntivo che mostra il quantitativo di ore rendicontate investite durante quel periodo e i totali (rendicontati, non rendicontati e complessivi) delle ore spese fino al termine del periodo preso in esame. Al termine del progetto verrà presentato un consuntivo finale.
	\\Il bilancio orario può essere:
	\begin{itemize}
	\item \textbf{positivo:} il preventivo orario ha superato il consuntivo orario;
	\item \textbf{negativo:} il consuntivo orario ha superato il preventivo orario;
	\item \textbf{in pari:} il preventivo orario conincide con il consuntivo orario.
	\end{itemize}
	Il bilancio economico può essere:
	\begin{itemize}
	\item \textbf{positivo:} il preventivo economico ha superato il consuntivo economico;
	\item \textbf{negativo:} il consuntivo economico ha superato il preventivo economico;
	\item \textbf{in pari:}  il preventivo economico conincide con il consuntivo economico.
	\end{itemize}
	
	\newpage
	\subsection {Dettaglio periodi}
		\subsubsection {Periodo: An - Analisi}
			\paragraph{Consuntivo orario}
			Si ricorda che le ore di questo periodo figurano come non rendicontate.
%---------------------------------------------------------------------
								\begin{table}[H] \begin{center} \begin{tabular}{llllllll}
								\toprule
								\textbf{Nominativo}	&	\textbf{Re}		&	\textbf{Am}		&	\textbf{At}		&	\textbf{Pj}		&	\textbf{Pr}		&	\textbf{Ve}		&	\textbf{Tot}		 \\
								\midrule																						 
								Brutesco	&	-		&	-		&	13	(+1)	&	-		&	-		&	12		&	25	(+1)\\
								Damo		&	-		&	9		&	16	(+1)	&	-		&	-		&	-		&	25	(+1)\\
								De Gaspari	&	-		&	-		&	13	(+1)	&	-		&	-		&	12		&	25	(+1)\\
								Gottardo	&	14		&	-		&	11	(+1)	&	-		&	-		&	-		&	25	(+1)\\
								Pasqualini	&	-		&	-		&	13	(+1)	&	-		&	-		&	12		&	25	(+1)\\
								Petenazzi	&	7		&	-		&	18	(+1)	&	-		&	-		&	-		&	25	(+1)\\
								Prete		&	-		&	10		&	15	(+1)	&	-		&	-		&	-		&	25	(+1)\\
								\midrule																					
								Tot in ore	&	21		&	19		&	99	(+7)	&	0		&	0		&	36		&	175	(+7)\\
								\bottomrule
								\end{tabular} \end{center} \caption{Prospetto orario a consuntivo per il periodo di analisi}\label{tab:oreAnalisiCons}
									\end{table}
%---------------------------------------------------------------------
			\paragraph{Consuntivo economico}
			La differenza tra le ore a consuntivo e preventivo di questo periodo non sono a carico del proponente e quindi saranno considerate solamente come ore di investimento non rendicontate.
%-----------------------------------------------------------------


								\begin{table}[H] \begin{center} \begin{tabular}{llllllll}
								\toprule
									&	\textbf{Re}	&	\textbf{Am}	&	\textbf{At}	&	\textbf{Pj}	&	\textbf{Pr}	&	\textbf{Ve}	&	\textbf{Tot}	 \\
								\midrule
								Tot in ore	&	21		&	19		&	99 (+7)	&	0		&	0		&	36		&	175	(+7)\\
								Tot in €	&	 € 630 		&	 € 380 		&	 € 2.475 		&	 € -   		&	 € -   		&	 € 540 		&	 € 4.025 	\\
								\bottomrule
								\end{tabular} \end{center} \caption{Prospetto economico - Periodo:
								An
								}\label{tab:sAnCons} \end{table}

%-----------------------------------------------------------------
			\paragraph{Conclusioni}
			Il lavoro degli \analisti{} ha richiesto più tempo di quello preventivato in quanto lo studio del dominio si è dimostrato più difficile di quanto previsto. Come si vede dalla tabella \ref{tab:oreAnalisiCons} il bilancio orario risulta negativo in quanto eccede di 7 ore rispetto a quanto pianificato.
			Come si vede dalla tabella \ref{tab:sAnCons} il bilancio economico è negativo per un importo pari a -175€.
			Queste variazioni rispetto al preventivo non avranno impatto sul costo finale in quanto le ore aggiuntive sono considerate di investimento.

%---------------------------------------------------------------------------------
	\newpage
	\subsubsection{Pl - Progettazione logica}
	\paragraph{Consuntivo orario}
%-----------------------------------------------------------------------------------
								\begin{table}[h] \begin{center} \begin{tabular}{llllllll}																						
								\toprule																						
									&	Re		&	Am		&	An		&	Pj		&	Pr		&	Ve		&	Tot	 \\ 	
								\midrule																		
								Brutesco	&	5		&	-		&	-		&	21		&	-		&	-		&	26	\\
								Damo		&	5		&	-		&	-		&	30		&	-		&	-		&	35	\\
								De Gaspari	&	-		&	-		&	-		&	6	(-4)	&	-		&	20	(+4)	&	26	\\
								Gottardo	&			&	-		&	-		&	10		&	-		&	16		&	26	\\
								Pasqualini	&	-		&	5		&	5		&	16		&	-		&	-		&	26	\\
								Petenazzi	&	-		&	4		&	5		&	17		&	-		&	-		&	26	\\
								Prete	&	-		&	-		&	-		&	14	(+4)	&	-		&	12	(-4)	&	26	\\
								\midrule																					
								Tot in ore	&	10		&	9		&	10		&	114(+0)		&	0		&	48(+0)		&	191	\\
								
								\bottomrule
								\end{tabular} \end{center} \caption{Prospetto a consuntivo orario per il periodo di																						
									Progettazione logica																						
									}\label{tab:orePl} \end{table}	
%-----------------------------------------------------------------------------------
	\paragraph{Consuntivo economico}
%-----------------------------------------------------------------------------------
								\begin{table}[H] \begin{center} \begin{tabular}{llllllll}																						
							\toprule	
								&	Re		&	Am		&	An		&	Pj		&	Pr		&	Ve		&	Tot	 \\ 	
							\midrule																		
							Tot in ore	&	10		&	9		&	10		&	114(+0)		&	0		&	48(+0)		&	191	\\
							Tot in €	&	 € 300 		 & 	 € 180 		 & 	 € 250 		 & 	 € 2.508 		 & 	 € -   		 & 	 € 720 		 & 	 € 3.958 	\\
							\bottomrule																						
							\end{tabular} \end{center} \caption{Prospetto a consuntivo economico per il periodo di																						
							Progettazione logica																						
							}\label{tab:sPl} \end{table}
	
	%----------------------------------------------------------------------------	
				\paragraph{Conclusioni}
				E’ stato necessario un ricalibro delle ore a causa di problemi personali di un membro, come indicato
				dall’analisi dei rischi. Nonostante ciò, il consuntivo orario presenta gli stessi totali esposti nel preventivo orario di questo periodo.
%---------------------------------------------------------------------------------
\newpage
\subsubsection{PCV - Progettazione Codifica Validazione}
\paragraph{Consuntivo orario}
%-----------------------------------------------------------------------------------
							\begin{table}[H] \begin{center} \begin{tabular}{llllllll}
										\toprule
										\textbf{Nominativo}	&	\textbf{Re}	&	\textbf{Am}	&	\textbf{At}	&	\textbf{Pj}	&	\textbf{Pr}	&	\textbf{Ve}	&	\textbf{Tot}\\
										\midrule
										Brutesco	&	-		&	4		&	-		&	5		&	14	(-1)&	29	&	52	\\
										Damo		&	-		&	-		&	1	(+1)&	9		&	20	(-2)&	39	&	69	\\
										De Gaspari	&	10		&	-		&	-		&	8		&	24	(-1)&	13	&	55	\\
										Gottardo	&	-		&	6		&	-		&	28		&	21	(-1)&	-	&	55	\\
										Pasqualini	&	-		&	4		&	-		&	29		&	15	(-1)&	7	&	55	\\
										Petenazzi	&	-		&	5		&	-		&	12		&	19	(-1)&	19	&	55	\\
										Prete		&	10		&	-		&	-		&	6		&	20	(-1)&	16	&	52	\\
										\midrule																				
										Tot in ore	&	20		&	19		&	1	(+1)&	97		&	133	(-8)&	123	&	393	(-7)\\
																				
										\bottomrule
									\end{tabular} \end{center} \caption{Consuntivo orario - Periodo:
									PCV
								} \end{table}
%-----------------------------------------------------------------------------------
\paragraph{Consuntivo economico}
%-----------------------------------------------------------------------------------
\begin{table}[H] \begin{center} \begin{tabular}{llllllll}																						
			\toprule	
			&	Re		&	Am		&	An		&	Pj		&	Pr		&	Ve		&	Tot	 \\ 	
			\midrule																		
			Tot in ore	&	20		&	19		&	1	(+1)&	97		&	133	(-8)&	123	&	393	(-7)\\
			Tot in €	&	 € 600 		 & 	 € 380 		 & 	 € 25 		 & 	 € 2.134 		 & 	 € 1.995 		 & 	 € 1.845 		 & 	 € 6.979 	\\
			\bottomrule																						
		\end{tabular} \end{center} \caption{Prospetto a consuntivo economico per il periodo PCV																					
	}\label{tab:s_PCV_c} \end{table}

%----------------------------------------------------------------------------	

				
	\paragraph{Conclusioni}
	A livello di pianificazione, i codificatori hanno impiegato più tempo di quanto previsto, anche a causa di un rischio non preventivato che si è verificato: non sono state seguite le norme di codifica stabilite. La codifica è stata, specialmente nei momenti iniziali poco organizzata e alquanto indisciplinata. Il \responsabile{} ha cercato di arginare il fenomeno richiamando gli amministratori in modo che stabilissero un maggior rigore specialmente per quanto riguarda :
	\begin{itemize}
		\item il controllo di configurazione;
		\item l'integrazione degli strumenti e delle tecnologie.
	\end{itemize}
	Le conseguenze principali sono:
	\begin{itemize}
		\item gli incrementi 5, 6, 7, 8 pianificati (legati a requisiti facoltativi) non sono stati svolti;
		\item negli incrementi 1, 2, 3, 4, si è:
				\begin{itemize}
					\item progettato a basso livello l'incremento;
					\item parzialmente codificato l'incremento;
					\item parzialmente effettuati i test di unita dell'incremento.
				\end{itemize}
			non effettuando alcuna validazione del prodotto, e portando di fatto a uno sviluppo di carattere sequenziale.
	\end{itemize}
	Il sorgere di alcuni problemi col proponente come indicato dall'analisi dei rischi, ha portato alla formazione di un'ora di analisi, necessaria per la modifica dell' \adr{} e a una diminuzione delle ore di codifica, anche grazie al fatto che non si sono svolti gli incrementi opzionali.
	Il consuntivo orario ed economico presenta quindi delle variazioni rispetto a quanto era stato preventivato nelle tabelle \ref{tab:h_PCV} e \ref{tab:s_PCV}. In particolare sono state impiegate 7 ore in meno diminuendo la spesa di 95€.
	Per cercare di risolvere questi problemi, è stato steso un adeguato preventivo a finire.
	\newpage
	
	
	\subsubsection{Va - Validazione}
				\paragraph{Consuntivo orario}
				\begin{table}[H] \begin{center} \begin{tabular}{llllllll}
							\toprule
							\textbf{Nominativo}	&	\textbf{Re}	&	\textbf{Am}	&	\textbf{At}	&	\textbf{Pj}	&	\textbf{Pr}	&	\textbf{Ve}	&	\textbf{Tot}	 \\
							\midrule
							Brutesco	&	-		&	-		&	-		&	-		&	6		&	21(+1)		&	27	\\
							Damo		&	-		&	-		&	-		&	-		&	-		&	-		&	0	\\
							De Gaspari	&	-		&	10		&	-		&	-		&	-		&	14(+1)		&	24	\\
							Gottardo	&	-		&	-		&	-		&	-		&	6		&	18(+1)		&	24	\\
							Pasqualini	&	13		&	-		&	-		&	-		&	-		&	11(+1)		&	24	\\
							Petenazzi	&	-		&	-		&	-		&	-		&	3(-4)	&	21(+5)	&	24	\\
							Prete		&	1(+1)	&	-		&	-		&	-		&	15(+4)	&	11(-4)	&	27	\\
							\midrule																					
							Tot in ore	&	14	(+1)&	10		&	0		&	0		&	30(+0)	&	96	(+6)&	144(+6)	\\
							
							\bottomrule
						\end{tabular} \end{center} \caption{Prospetto orario consuntivo - Periodo:
						Va
					}\end{table}
					\paragraph{Consuntivo economico}
					\begin{table}[H] \begin{center} \begin{tabular}{llllllll}
								\toprule
								&	\textbf{Re}	&	\textbf{Am}	&	\textbf{At}	&	\textbf{Pj}	&	\textbf{Pr}	&	\textbf{Ve}	&	\textbf{Tot}	 \\
								
								\midrule
								Tot in ore	&	14		&	10		&	0		&	0		&	30		&	90		&	144	\\
								Tot in €	&	 € 420 		 & 	 € 200 		 & 	 € -   		 & 	 € -   		 & 	 € 450 		 & 	 € 1.350 		 & 	 € 2.420 	\\
								\bottomrule
							\end{tabular} \end{center} \caption{Prospetto economico consuntivo - Periodo:
							Va
						}\end{table}
						
		\paragraph{Conclusioni}
		Il responsabile ha speso un'ora in più rispetto a quanto previsto per la gestione dei problemi con il proponente. Sono state spese sei ore di verifica in più per effettuare controlli più minuziosi sull'attività di codifica per arrivare ad avere una codifica quanto più coordinata, ordinata e attinente alle norme di progetto, anche alla luce dei problemi sorti durante lo scorso periodo. Da notare che vengono contate ore aggiuntive di progettazione o codifica per la correzione e il miglioramento tecnico di quanto consegnato in Revisione di Qualifica. Tali ore sono state conteggiate come non rendicontate.
									\newpage
%-----------------------------------------------------------------
	\subsection{Totale non rendicontato}
		\subsubsection{Consuntivo orario}
%-----------------------------------------------------------------
						\begin{table}[h] \begin{center} \begin{tabular}{llllllll}																						
						\toprule
									&	Re		&	Am		&	An		&	Pj		&	Pr		&	Ve		&	Tot	 \\ 	
						\midrule					
						Brutesco	&	2		&	3		&	14		&	4(+2)		&	4(+2)		&	14	&	41(+4)	\\
						Damo		&	2		&	12		&	17		&	2		&	2		&	2		&	37	\\
						De Gaspari	&	2		&	3		&	14		&	4(+2)		&	4(+2)		&	14	&	41(+4)	\\
						Gottardo	&	16		&	3		&	12		&	4(+2)		&	4(+2)		&	2		&	41(+4)	\\
						Pasqualini	&	2		&	3		&	14		&	4(+2)		&	4(+2)		&	14		&	41(+4)	\\
						Petenazzi	&	9		&	3		&	19		&	4(+2)		&	4(+2)		&	2		&	41(+4)	\\
						Prete	&	2		&	13		&	16		&	4(+2)		&	4(+2)		&	2		&	41(+4)	\\
						\midrule																Tot in ore	&	35		&	40		&	106		&	26(+12)		&	26(+12)		&	50		&	283(+24)	\\						
				
					
					\bottomrule																					
																							
						\end{tabular} \end{center} \caption{Prospetto orario a consuntivo totale non rendicontato													
						}\label{tab:oreNonRend} \end{table}							
%-----------------------------------------------------------------
		\subsubsection{Consuntivo economico}
%-----------------------------------------------------------------
						\begin{table}[H] \begin{center} \begin{tabular}{llllllll}
						\toprule
							&	\textbf{Re}	&	\textbf{Am}	&	\textbf{At}	&	\textbf{Pj}	&	\textbf{Pr}	&	\textbf{Ve}	&	\textbf{Tot}\\
			
						\midrule
						Tot in ore	&	35		&	40		&	106		&	26		&	26		&	50		&	283	\\
						Tot in €	&	 € 1.050 		 & 	 € 800 		 & 	 € 2.650 		 & 	 € 572 		 & 	 € 390 		 & 	 € 750 		 & 	 € 6.212 	\\

						\bottomrule
						\end{tabular} \end{center} \caption{Prospetto economico a consuntivo totale non rendicontato 
						} \end{table}
					
			\paragraph{Conclusioni}
			I progettisti e i programmatori hanno speso 24 ore in più del previsto per effettuare alcuni miglioramenti tecnici al prodotto consegnato in Revisione di Qualifica, come descritto nel Verbale Interno 9. Questo ha portato ad avere una variazione sul preventivo del totale non rendicontato di 24 ore, portando a spendere 444 euro in più rispetto al previsto nelle tabelle di preventivo \ref{tab:h_TotaleNonRendicontato} e  \ref{tab:s_TotaleNonRendicontato_prev}. Tali variazioni sono da considerarsi non a carico del proponente.
%-----------------------------------------------------------------
	
	\newpage
	\subsection{Totale complessivo}
		\subsubsection{Consuntivo orario}

%-----------------------------------------------------------------------------------------																					
						\begin{table}[h] \begin{center} \begin{tabular}{llllllll}																					
						\toprule		
							&	Re		&	Am		&	An		&	Pj		&	Pr		&	Ve		&	Tot	\\
							\midrule								
							Brutesco	&	7		&	7		&	14		&	30(+2)		&	24(+1)		&	64(+1)	&	146(+4)	\\
							Damo		&	7		&	12		&	18(+1)	&	41			&	22(-2)		&	41		&	141(-1)	\\
							De Gaspari	&	12		&	13		&	14		&	18(-2)			&	28(+1)		&	61(+5)	&	146(+4)	\\
							Gottardo	&	16		&	9		&	12		&	42(+2)		&	31(+1)		&	36(+1)	&	146(+4)	\\
							Pasqualini	&	15		&	12		&	19		&	49(+2)		&	19(+1)		&	32(+1)	&	146(+4)	\\
							Petenazzi	&	9		&	12		&	24		&	33(+2)		&	26(-3)		&	42(+5)	&	146(+4)	\\
							Prete		&	13(+1)	&	13		&	16		&	24(+6)		&	39(+5)		&	41(-8)	&	146(+4)	\\
							\midrule																					
							Tot in ore	&	79(+1)		&	78		&	117(+1)		&	237(+12)		&	189(+4)		&	317(+5)		&	1017(+23)	\\

					
						\bottomrule																					
						\end{tabular} \end{center} \caption{Prospetto orario a consuntivo totale complessivo							
						} \end{table}
%-----------------------------------------------------------------------------------------																					
		\subsubsection{Consuntivo economico}
%-----------------------------------------------------------------
						\begin{table}[H] \begin{center} \begin{tabular}{llllllll}
						\toprule
							&	\textbf{Re}	&	\textbf{Am}	&	\textbf{At}	&	\textbf{Pj}	&	\textbf{Pr}	&	\textbf{Ve}	&	\textbf{Tot}\\
						\midrule																					
							Tot in ore	&	79		&	78		&	117		&	237		&	189		&	317		&	1017	\\
							Tot in €	&	 € 2.370 		 & 	 € 1.560 		 & 	 € 2.925 		 & 	 € 5.214 		 & 	 € 2.835 		 & 	 € 4.755 		 & 	 € 19.659 	\\




						\bottomrule			
						\end{tabular} \end{center} \caption{Prospetto economico a consuntivo totale complessivo
						} \end{table}
					
						\subsection{Conclusioni}
						Il totale complessivo a consuntivo (e di conseguenza anche quello complessivo) presenta delle variazioni rispetto a quanto preventivato; in particolare sono state spese complessivamente 23 ore in più (che figurano tutte come non rendicontate) rispetto a quanto preventivato nelle tabelle  \ref{tab:h_TotaleComplessivo} e  \ref{tab:s_TotaleComplessivo}.
						Sono stati spesi quindi 454 euro in più (di cui 10 rendicontati e 444 non rendicontati).
%-----------------------------------------------------------------
	
	\newpage
	\subsection{Totale rendicontato}
		\subsubsection{Consuntivo orario}
		
		%-----------------------------------------------------------------------------------------																					
		\begin{table}[h] \begin{center} \begin{tabular}{llllllll}																					
					\toprule																					
					&	Re		&	Am		&	An		&	Pj		&	Pr		&	Ve		&	Tot	\\
					\midrule													Brutesco	&	5		&	4		&	0		&	26		&	20(-1)		&	50(+1)		&	105	\\
					Damo	&	5		&	0		&	1(+1)		&	39		&	20(-2)		&	39		&	104(-1)	\\
					De Gaspari	&	10		&	10		&	0		&	14(-4)		&	24(-1)		&	47(+5)		&	105	\\
					Gottardo	&	0		&	6		&	0		&	38		&	27(-1)		&	34(+1)		&	105	\\
					Pasqualini	&	13		&	9		&	5		&	45		&	15(-1)		&	18(+1)		&	105	\\
					Petenazzi	&	0		&	9		&	5		&	29		&	22(-5)		&	40(+5)		&	105	\\
					Prete	&	11(+1)		&	0		&	0		&	20(+4)		&	35(+3)		&	39(+8)		&	105	\\
					\midrule																					
					Tot in ore	&	44(+1)		&	38		&	11(+1)		&	211		&	163(-8)		&	267(+5)		&	734(-1)	\\
																						
																				
				\end{tabular} \end{center} \caption{Prospetto orario a consuntivo totale rendicontato											
			} \end{table}
			%-----------------------------------------------------------------------------------------																					
			\subsubsection{Consuntivo economico}
			%-----------------------------------------------------------------
			\begin{table}[H] \begin{center} \begin{tabular}{llllllll}
						\toprule
						&	\textbf{Re}	&	\textbf{Am}	&	\textbf{At}	&	\textbf{Pj}	&	\textbf{Pr}	&	\textbf{Ve}	&	\textbf{Tot}\\
							\midrule																					
						Tot in ore	&	44		&	38		&	11		&	211		&	163		&	267		&	734	\\
						Tot in €	&	 € 1.320 		 & 	 € 760 		 & 	 € 275 		 & 	 € 4.642 		 & 	 € 2.445 		 & 	 € 4.005 		 & 	 € 13.447 	\\

						\bottomrule			
					\end{tabular} \end{center} \caption{Prospetto economico a consuntivo totale rendicontato
				}\label{tab:s_TotaleNonRendicontato} \end{table}
			%-----------------------------------------------------------------

	\subsection{Conclusioni}
	\label{consid_cons_pl}
	Il totale rendicontato a consuntivo presenta delle variazioni rispetto a quanto preventivato; in particolare il bilancio orario si chiude in positivo di un'ora, mentre quello economico si chiude in negativo di 10€ rispetto a quanto preventivato nelle tabelle \ref{tab:h_TotaleRendicontato} e  \ref{tab:s_TotaleRendicontato}.