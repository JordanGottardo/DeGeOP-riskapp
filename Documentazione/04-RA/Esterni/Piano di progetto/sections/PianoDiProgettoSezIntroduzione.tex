% section {Introduzione}
\section {Introduzione}
	\subsection {Scopo del documento}
	Con il seguente documento si intende pianificare il modo e i tempi in cui il \glo{Gruppo}{gruppo} intende sviluppare il progetto \progetto. In particolare, gli scopi del documento sono:
	\begin{itemize}
	 \item {analizzare e gestire i fattori di rischio;}
	 \item {collocare le attività da svolgere nel tempo;}
	 \item {assegnare le attività da svolgere ai membri del gruppo;}
	 \item {definire il preventivo dei costi;}
	 \item {definire il consuntivo di periodo;}
	 \item {definire il preventivo a finire.}
	\end{itemize}
	\subsection {Scopo del prodotto}
	\introScopo
	\subsection {Glossario}
	\introGlossario
	\subsection {Riferimenti}
		\subsubsection{Riferimenti normativi}
		\begin{itemize}
				\item \textbf{regole del progetto didattico:}\\
                \url{http://www.math.unipd.it/~tullio/IS-1/2016/Dispense/L09.pdf};
				\item {\ndpv};
				\item \textbf{\glo{Capitolato}{capitolato} d'appalto:}\\
                \url{http://www.math.unipd.it/~tullio/IS-1/2016/Progetto/C3.pdf}
		\end{itemize}
		\subsubsection{Riferimenti informativi}
		\begin{itemize}
				\item \textbf{slides del corso di Ingegneria del Software:}\\
                \url{http://www.math.unipd.it/~tullio/IS-1/2016/};
                \item \textbf{SWEBok - Chapter 8:Software Engineering Management:}\\
                \url{http://www.math.unipd.it/~tullio/IS-1/2007/Approfondimenti/SWEBOK.pdf};
				\item {\sdfv};
				\item {\pdqv};
				\item {\adrv};
				\item \textbf{regolamento di Organigramma:}
				\url{http://www.math.unipd.it/~tullio/IS-1/2016/Progetto/PD01b.html}.
		\end{itemize}
	\subsection {Scadenze e altri vincoli}
	\label{subsec:scadenze}
	\begin{itemize}
	\item
		È stato deciso di rispettare le seguenti scadenze di consegna del materiale:
		\begin{itemize}
			 \item {\revereq}: \frmdata{11}{01}{2017};
			 \item {\revprog}: \frmdata{06}{03}{2017};
			 \item {\revaqual}: \frmdata{11}{04}{2017};
			 \item {\revacc}: \frmdata{08}{05}{2017}.
		\end{itemize}
	\item
		Ogni componente del gruppo deve dimostrare di aver impiegato, per lo svolgimento del progetto, un numero di ore maggiore (o pari) a 85 e inferiore (o pari) a 105.
		Un componente del gruppo, Giovanni Damo, dovrà lasciare lo svolgimento del progetto in data \frmdata{05}{04}{2017}. All'uscita dal progetto il componente dovrà aver comunque adempiuto all'obbligo sopra descritto.
	\end{itemize}
	%subsection {Scelta RP min/max} io la metterei nella pianificazione
