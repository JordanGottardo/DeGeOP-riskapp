%\section {Pianificazione}
\section {Pianificazione}
\subsection {Introduzione}
	\subsubsection{Scelta modello di sviluppo}
	Il modello di \glo{Ciclo di vita}{ciclo di vita} scelto per il prodotto è il modello incrementale.\\
	Lo sviluppo del progetto viene suddiviso in quattro periodi significativi, descritti nella sezione \ref{periodi_e_milestones}.
	\paragraph{Strutturazione}
	Questo modello consiste, nei primi periodi di sviluppo del prodotto, nel:
	\begin{itemize}
			\item fissare i requisiti principali;
			\item definire gli incrementi;
			\item assegnare i requisiti agli incrementi;
			\item stabilire l'architettura di massima del sistema.
	\end{itemize}
	Il \glo{Gruppo}{gruppo} svolgerà questo lavoro nei primi due periodi. Il terzo periodo si compone di un ciclo in cui ogni iterazione è data da:
	\begin{itemize}
		\item sviluppo dell'incremento (l'incremento viene progettato a basso livello, codificato, testato e validato);
		\item integrazione dell'incremento;
		\item validazione dell'intero prodotto creato fino a quel momento.
	\end{itemize}
	Una volta terminati tutti gli incrementi, il ciclo terminerà.
	Durante l'ultimo periodo si effettueranno i test di accettazione e il collaudo del sistema.
	
	\paragraph{Vantaggi}
	I vantaggi di questo modello sono:
	\begin{itemize}
		\item la presenza di rilasci multipli e successivi, che aggiungono valore tangibile al prodotto;
		\item la possibilità del proponente di prendere visione periodicamente della parte di prodotto creata e di rilasciare un parere;
		\item la riduzione del livello generale di rischio del progetto.
	\end{itemize}

	\subsubsection {Scelta consegna RP}
	Il gruppo ha scelto di consegnare alla \revprog{} i risultati della progettazione ad alto livello. Questo per identificare tempestivamente errori concettuali, riducendone quindi il costo di correzione.
	%\subsubsection{Dipendenze fra le attività}
	\subsubsection{Periodi e milestones}
	\label{periodi_e_milestones}
	Si è deciso di fissare una \glo{Milestone}{milestone} in corrispondenza di ogni scadenza stabilita nella sezione \ref{subsec:scadenze}.
	Ogni periodo termina 4 giorni prima di una milestone; il tempo in eccesso potrà essere usato in caso di ritardi nel completamento delle attività.\\

	\begin{table}[H]
		\footnotesize\setlength{\tabcolsep}{6pt}
		\begin{center}
			\begin{tabular}{lllll}
				\toprule
				ID periodo & Nome periodo                                             & Inizio                        & Fine                          & Relativa milestone                            \\
				\midrule
				An      & Analisi                                               & \frmdata{01}{12}{2016}  & \frmdata{07}{01}{2017}  & RR - \frmdata{11}{01}{2017}                  \\
				Pl      & Progettazione logica                                  & \frmdata{12}{01}{2017}  & \frmdata{02}{03}{2017}  & RP - \frmdata{06}{03}{2017}                  \\
				PCV   & Prog. dettaglio Codifica Validazione  & \frmdata{07}{03}{2017}  & \frmdata{07}{03}{2017} & RQ - \frmdata{11}{04}{2017} \\
				Va      & Finalizzazione                                           & \frmdata{12}{04}{2017} & \frmdata{04}{05}{2017}  & RA - \frmdata{08}{05}{2017}  \\
				\bottomrule
			\end{tabular}
		\end{center}
		\caption{Periodi e milestones}
		\label{tab:periodi}
	\end{table}\mbox{}\\
	
\clearpage
\subsection{Assegnazione delle attività ai periodi}
	I processi e le attività assegnate ad ogni periodo nelle seguenti sezioni del documento sono state descritte nelle \ndpv:

	\subsubsection {Analisi}
		\textbf{Periodo}: da \frmdata{01}{12}{2016} a \frmdata{07}{01}{2017} (38 giorni) \\
		Durante questo periodo vengono valutati i vari capitolati proposti e dopo aver scelto un determinato \glo{Capitolato}{capitolato}, vengono gestite principalmente la contrattazione e la raccolta di requisiti principali.
		\paragraph{Processi e attività coinvolte}
			\begin{table}[H]
				\centering
				\begin{tabular}{ll}
					\toprule
					\textbf{Processo}                           & \textbf{Attività}              \\
					\midrule
					\multirow{2}{*}{\textbf{Fornitura}}         & Studio di Fattibilità          \\
					& Contrattazione                 \\
					\midrule
					\textbf{Sviluppo}          & Analisi dei requisiti          \\
					\midrule
					\textbf{Documentazione}            & Produzione dei documenti       \\
					\midrule
					\textbf{Verifica}                  & Verifica                       \\
					\midrule
					\textbf{Gestione processi} 					& Gestione processi              \\
					\midrule
					\textbf{Gestione infrastrutture}				& Gestione infrastrutture        \\
					\midrule
					\textbf{Gestione della configurazione}				& Gestione della configurazione        \\
					\midrule
					\textbf{Apprendimento} 						& Apprendimento                 \\
					\bottomrule
				\end{tabular}
				\caption{Processi e relative attività}
				\label{An-ProcessiAttività}
			\end{table}
		\paragraph{Diagramma di Gantt}
		\begin{figure}[H]
			\centering
			\includegraphics[width=\textwidth]{img/Gantt/g1c.png}
			\caption{Diagramma di Gantt - \glo{Periodo}{Periodo} di analisi}
		\end{figure}
		
	\newpage
	\subsubsection {Progettazione logica}
		\textbf{Periodo}: da \frmdata{12}{01}{2017} a \frmdata{02}{03}{2017} (50 giorni) \\
		Durante questo periodo:
		\begin{itemize}
			\item viene ultimata la raccolta di requisiti;
			\item vengono definiti gli incrementi;
			\item vengono assegnati i requisiti agli incrementi;
			\item viene stabilita l'architettura di massima del sistema.
		\end{itemize}
		Nel corso della progettazione logica verranno quindi individuati gli incrementi, che saranno riportati nel diagramma di Gantt del terzo periodo a partire dalla versione 2.0.0 del \pdp.
		\paragraph{Incrementi individuati}
		Durante il secondo periodo sono stati individuati i seguenti incrementi:
		\begin{itemize}
			\item incremento 1 - mappa;
			\item incremento 2 - \glo{Asset}{asset};
			\item incremento 3 - \glo{Nodo}{nodi};
			\item incremento 4 - \glo{Arco}{archi};
			\item incremento 5 - scenari;
			\item incremento 6 - analisi;
			\item incremento 7 - tutorial;
			\item incremento 8 - assistente vocale.
		\end{itemize}
		\paragraph{Assegnazione dei requisiti agli incrementi}
		Durante il secondo periodo i requisiti presenti nel documento di \adr sono stati assegnati agli incrementi elencati nel paragrafo precedente.
			\begin{table}[H]
				\centering
				\begin{tabular}{ll}
					\toprule
					\textbf{Requisiti}                           & \textbf{Incrementi}              \\
					\midrule
					ROF1 & 2 - asset \\
					ROF2 & 2 - asset \\
					ROF3 & 2 - asset \\
					ROF4 & 2 - asset \\
					ROF5 & 2 - asset \\
					\midrule
					ROF6 & 3 - nodi \\
					ROF7 & 3 - nodi \\
					ROF8 & 3 - nodi \\
					ROF9 & 3 - nodi \\
					ROF10 & 3 - nodi \\
					\midrule
					ROF11 & 4 - archi \\
					ROF12 & 4 - archi \\
					ROF13 & 4 - archi \\
					ROF14 & 4 - archi \\
					ROF15 & 4 - archi \\
					\midrule
					RFF16 & 5 - scenari \\
					RFF17 & 5 - scenari \\
					RFF18 & 5 - scenari \\
					RFF19 & 5 - scenari \\
					RFF20 & 5 - scenari \\
					\midrule
					RFF21 & 6 - analisi \\
					RFF22 & 6 - analisi \\
					RFF23 & 6 - analisi \\
					\midrule
					ROF24 & 1 - mappa \\
					\midrule
					RFF25 & 7 - tutorial \\
					\midrule
					RFF6  & 8 - assistente vocale \\
					\bottomrule
				\end{tabular}
				\caption{Assegnazione dei requisiti agli incrementi}
			\end{table}
		I requisiti non presenti in questa tabella sono da considerarsi implicitamente assegnati ad ogni incremento.
		\paragraph{Processi e attività coinvolte}
			\begin{table}[H]
				\centering
				\begin{tabular}{ll}
					\toprule
					\textbf{Processo}                           & \textbf{Attività}              \\
					\midrule
					\multirow{2}{*}{\textbf{Sviluppo}}          & Analisi dei requisiti          \\
					& Progettazione ad alto livello  \\
					\midrule
					\textbf{Documentazione}            & Produzione dei documenti       \\
					\midrule
					\textbf{Verifica}                  & Verifica                       \\
					\midrule
					\textbf{Gestione processi} 					& Gestione processi              \\
					\midrule
					\textbf{Gestione infrastrutture}				& Gestione infrastrutture        \\
					\midrule
					\textbf{Gestione della configurazione}				& Gestione della configurazione        \\
					\midrule
					\textbf{Apprendimento} 						& Apprendimento                 \\
					\bottomrule
				\end{tabular}
				\caption{Processi e relative attività}
				\label{Pl-ProcessiAttività}
			\end{table}
		\paragraph{Diagramma di Gantt}
		\begin{figure}[H]
			\centering
			\includegraphics[width=\textwidth]{img/Gantt/g2c.png}
			\caption{Diagramma di Gantt - Periodo di progettazione logica}
		\end{figure}
	
	\newpage
	\subsubsection {Progettazione di dettaglio, codifica e validazione}
		\textbf{Periodo}: da \frmdata{07}{03}{2017} a \frmdata{07}{04}{2017} (31 giorni)\\
		Questo periodo consiste in un ciclo in cui ogni iterazione è data da progettazione, codifica, testing, integrazione del singolo incremento e dalla validazione dell'intero prodotto creato fino a quel momento.
		\\Una volta terminati tutti gli incrementi, il ciclo terminerà.
		\\Le linee verticali presenti nel diagramma indicano i rilasci del sistema validato alla fine dei vari incrementi.
		\paragraph{Processi e attività coinvolte}
			\begin{table}[H]
				\centering
				\begin{tabular}{ll}
					\toprule
					\textbf{Processo}                           & \textbf{Attività}              \\
					\midrule
					\multirow{2}{*}{\textbf{Sviluppo}}          & Progettazione ad basso livello \\
					& Codifica e test \\
					\midrule
					\textbf{Documentazione}            & Produzione dei documenti       \\
					\midrule
					\textbf{Verifica}                  & Verifica                       \\
					\midrule
					\textbf{Validazione}               & Validazione                    \\
					\midrule
					\textbf{Gestione processi} 					& Gestione processi              \\
					\midrule
					\textbf{Gestione infrastrutture}				& Gestione infrastrutture        \\
					\midrule
					\textbf{Gestione della configurazione}				& Gestione della configurazione        \\
					\midrule
					\textbf{Apprendimento} 						& Apprendimento                 \\
					\bottomrule
				\end{tabular}
				\caption{Processi e relative attività}
				\label{Pdrob-ProcessiAttività}
			\end{table}
		\paragraph{Diagramma di Gantt}
		\begin{figure}[H]
			\centering
			\includegraphics[width=\textwidth]{img/Gantt/g3c.png}
			\caption{Diagramma di Gantt - Periodo di Progettazione di dettaglio Codifica Validazione}
		\end{figure}
		
	\newpage
	\subsubsection {Validazione}
		\textbf{Periodo}: da \frmdata{12}{04}{2017} a \frmdata{04}{05}{2017} (23 giorni) \\
		Durante questo periodo si effettueranno i test di accettazione e il collaudo del sistema.
		\paragraph{Processi e attività coinvolte}
			\begin{table}[H]
				\centering
				\begin{tabular}{ll}
					\toprule
					\textbf{Processo}                           & \textbf{Attività}              \\
					\midrule
					\textbf{Documentazione}            & Produzione dei documenti       \\
					\midrule
					\textbf{Verifica}                  & Verifica                       \\
					\midrule
					\textbf{Validazione}               & Validazione                    \\
					\midrule
					\textbf{Gestione processi} 					& Gestione processi              \\
					\midrule
					\textbf{Gestione infrastrutture}				& Gestione infrastrutture        \\
					\midrule
					\textbf{Gestione della configurazione}				& Gestione della configurazione        \\
					\bottomrule
				\end{tabular}
				\caption{Processi e relative attività}
				\label{Va-ProcessiAttività}
			\end{table}
		\paragraph{Diagramma di Gantt}
		\begin{figure}[H]
			\centering
			\includegraphics[width=\textwidth]{img/Gantt/g4c.png}
			\caption{Diagramma di Gantt - Periodo di validazione}
		\end{figure}
		
			\newpage
			\subsubsection {Validazione - Pianificazione correttiva}
			\textbf{Periodo}: da \frmdata{12}{04}{2017} a \frmdata{18}{08}{2017} (128 giorni) \\
			In seguito all'esito della Revisione di Qualifica, si sono tenuti alcuni incontri tra i membri del gruppo e sono state prese importanti decisioni riguardo la consegna del progetto (vedi Verbale interno 6). In particolare si è deciso di consegnare il prodotto finale nella revisione di avanzamento del 22 agosto. Durante questo periodo dovranno essere effettuate opportune manovre correttive appositamente scelte per migliorare sul piano tecnico il prodotto consegnato in sede di Revisione di Qualifica, per poi passare a concludere i test di unità, integrazione e sistema mancanti. Infine come da piani precedentemente fissati, il gruppo si occuperà dei test di accettazione e il collaudo del sistema. La correzione di quanto presentato in Revisione di Qualifica ha portato alla reintroduzione del processo di sviluppo in questo periodo ed è da considerarsi una iterazione che figurerà nelle ore non rendicontate.
			\paragraph{Processi e attività coinvolte}
			\begin{table}[H]
				\centering
				\begin{tabular}{ll}
					\toprule
					\textbf{Processo}                           & \textbf{Attività}              \\
					\midrule
					\multirow{2}{*}{\textbf{Sviluppo}}          & Progettazione ad basso livello \\
					& Codifica e test \\
					\midrule
					\textbf{Documentazione}            & Produzione dei documenti       \\
					\midrule
					\textbf{Verifica}                  & Verifica                       \\
					\midrule
					\textbf{Validazione}               & Validazione                    \\
					\midrule
					\textbf{Gestione processi} 					& Gestione processi              \\
					\midrule
					\textbf{Gestione infrastrutture}				& Gestione infrastrutture        \\
					\midrule
					\textbf{Gestione della configurazione}				& Gestione della configurazione        \\
					\midrule
					\textbf{Apprendimento} 						& Apprendimento                 \\
					\bottomrule
				\end{tabular}
				\caption{Processi e relative attività}
				\label{VaCorr-ProcessiAttività}
			\end{table}
			\paragraph{Diagramma di Gantt}
			\begin{figure}[H]
				\centering
				\includegraphics[width=\textwidth]{img/Gantt/FiAprile.png}
				\caption{Diagramma di Gantt - Periodo di Validazione - Pianificazione correttiva Aprile}
			\end{figure}
			\newpage
			\begin{figure}[H]
				\centering
				\includegraphics[width=\textwidth]{img/Gantt/FiMaggio.png}
				\caption{Diagramma di Gantt - Periodo di Validazione - Pianificazione correttiva Maggio}
			\end{figure}
			\newpage
			\begin{figure}[H]
				\centering
				\includegraphics[width=\textwidth]{img/Gantt/FiGiugno.png}
				\caption{Diagramma di Gantt - Periodo di Validazione - Pianificazione correttiva Giugno}
			\end{figure}			\newpage
			\begin{figure}[H]
				\centering
				\includegraphics[width=\textwidth]{img/Gantt/FiLuglio.png}
				\caption{Diagramma di Gantt - Periodo di Validazione - Pianificazione correttiva Luglio}
			\end{figure}
			\newpage
			\begin{figure}[H]
				\centering
				\includegraphics[width=\textwidth]{img/Gantt/FiAgosto.png}
				\caption{Diagramma di Gantt - Periodo di Validazione - Pianificazione correttiva Agosto}
			\end{figure}
