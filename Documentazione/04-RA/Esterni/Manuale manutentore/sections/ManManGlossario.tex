\section{Glossario}
	\subsection{A}
	
		\mgref{Action}
		In Redux, oggetto che descrive un cambiamento di stato.
		
		\mgref{API}
		\textit{Application Programming Interface}. Insieme di procedure utilizzabili per interfacciarsi con un programma o un sistema informatico in modo standard. Spesso si intendono le librerie software disponibili in un certo linguaggio di programmazione.
	
		\mgref{Arco}
		Collegamento orientato tra due nodi.
		
		\mgref{Asset}
		Fabbricato con importanza strategica per il processo produttivo di un'azienda. Un asset può contenere uno o più nodi.
		
		\mgref{Asus}
		Azienda produttrice di dispositivi tecnologici di varia tipologia.
		
	\subsection{B}

		\mgref{Browser}
		Il web browser, o più semplicemente browser, è un'applicazione per il recupero, la presentazione e la navigazione di risorse web. Tali risorse (ad esempio pagine web, immagini, video) sono a disposizione sul World Wide Web su una rete locale o sullo stesso computer dove il browser è in esecuzione.
		
	\subsection{C}
		\mgref{Chrome}		
		\mglo{Browser}{Browser} web sviluppato da Google.
		
		\mgref{Componente}
		\begin{enumerate}
			\item React: elemento che fa parte della gerarchia del DOM. L'elemento eredita dalla classe base astratta React.Component.
			\item Unità software dotata di una precisa identità e interfacce ben definite.
		\end{enumerate}
		
		\mgref{Cross-platform}
		Possibilità di poter usare lo stesso strumento software su diversi sistemi operativi.
		
		\mgref{CSS}
		\textit{Cascading Style Sheets}. Linguaggio che permette di definire lo stile e la formattazione di una pagina \glo{HTML}{HTML}. Permette di mantenere separate presentazione e contenuto. \\
		La versione stabile più recente è CSS3.
		
	\subsection{D}
	
	\mgref{Design pattern}
	Soluzione progettuale generale per la risoluzione di un problema ricorrente. \\
	I design pattern orientati agli oggetti tipicamente mostrano relazioni ed interazioni tra classi o oggetti.
	Ad un livello più alto si trovano invece i pattern architetturali, con ambito più ampio. Essi descrivono un pattern complessivo adottato dall'intero sistema.
	
	\mgref{Dispatch}
	In Redux, metodo per inviare un'azione allo store.
	
	\mgref{DOM}
	\textit{Document Object Model}. Forma di rappresentazione dei documenti strutturati come modello orientato agli oggetti, definita dal \mglo{W3C}{W3C}.
	
	\subsection{E}
	
	\mgref{ESLint}
	Tool per l'analisi statica del codice \mglo{JavaScript}{JavaScript}.
	
	\subsection{F}
		\mgref{Firefox}
		\mglo{Browser}{Browser} web  libero e multipiattaforma, mantenuto da Mozilla Foundation.
		
		\mgref{Framework}
		Insieme di classi cooperanti che forniscono lo scheletro di un'applicazione riusabile per uno specifico dominio applicativo. Delinea l'architettura delle applicazioni in cui viene usato.
		
	\subsection{G}
		\mgref{Gesture}
		Combinazione di movimenti e click del dispositivo di puntamento (ad esempio il mouse) che vengono riconosciuti come comandi specifici.
		
		\mgref{Git}
		\mglo{Sistema di controllo di versione}{Sistema di controllo di versione} distribuito e \mglo{Open source}{open source}, creato da Linus Torvalds nel 2005.
		Vari progetti software usano Git per il controllo del versionamento, principalmente il kernel \mglo{Linux}{Linux}.
		
		\mgref{GitHub}
		Servizio Web per il controllo di versione basato su \mglo{Git}{Git}. Offre diversi piani per \mglo{Repository}{repository} privati sia a pagamento che gratuiti, molto utilizzati per lo sviluppo di progetti \mglo{Open source}{open source}.
		
		\mgref{Gruppo}
		Componenti che fanno parte del gruppo \zephyrus.
	\subsection{H}
	
	\mgref{HTML}
	\textit{HyperText Markup Language}. Linguaggio usato per la definizione di pagine Web; la sua sintassi è stabilita dal \mglo{W3C}{W3C}.
	HTML5 è l'ultima versione stabile.
	
	\subsection{I}
		\mgref{IDE}
		\textit{Integrated Development Enviroment}. Software utilizzato per la scrittura di codice sorgente. Spesso aiuta il programmatore segnalando errori di sintassi, oltre a tutta una serie di strumenti e funzionalità di supporto allo sviluppo.
	
		\mgref{iOS}
		Sistema operativo sviluppato da Apple Inc.
	
		\mgref{Ipad}
		Tablet prodotto dall'azienda Apple Inc.
	\subsection{J}
	
	\mgref{JavaScript}
	Linguaggio di scripting orientato agli oggetti e agli eventi. È utilizzato prevalentemente nella programmazione Web lato client per la creazione di effetti dinamici interattivi.
	
%	\subsection{K}
	\subsection{L}
	
		\mgref{Libreria}
		Insieme di funzioni e strutture dati predefinite e predisposte per lo sviluppo di software.
		
		\mgref{Linguaggio di markup}
		Insieme di regole che descrivono i meccanismi di rappresentazione (strutturali, semantici o presentazionali) di un testo che, utilizzando convenzioni standardizzate, sono utilizzabili su più supporti.
		
		\mgref{Linux}
		Linux è una famiglia di sistemi operativi di tipo Unix-like, rilasciati sotto varie distribuzioni, aventi la caratteristica comune di utilizzare come nucleo il kernel Linux. Ubuntu è la distribuzione di Linux più utilizzata.
	\subsection{M}
		\mgref{MacOS}
		Sistema operativo sviluppato da Apple.
	\subsection{N}
		\mgref{Node.js}
		Runtime \mglo{JavaScript}{JavaScript} costruito sul motore V8 di Chrome.
	
		\mgref{Nodo}
		Oggetto che fa parte dei processo produttivo aziendale. È contenuto all'interno di un asset.
	\subsection{O}
		\mgref{Open source}
		Software di cui i detentori dei diritti rendono pubblico il codice sorgente, permettendo ad altri programmatori di apportarvi modifiche. Questa possibilità è regolata tramite l’applicazione di apposite licenze d’uso.
		
		\mgref{OpenStreetMap}
		Servizio di mappe liberamente modificabili dell'intero pianeta.
	
	\subsection{P}
	
	\mgref{Package}
	Costrutto per organizzare classi logicamente correlate o che forniscono servizi simili, all'interno di sottogruppi ordinati. I package possono essere compressi permettendo la trasmissione di più classi in una sola volta. In \mglo{UML}{UML}, analogamente, è un raggruppamento arbitrario di elementi in una unità di livello più alto.
		
	\mgref{Proprietà}
	Input con i quali sono costruite le componenti React.
	
%	\subsection{Q}
	\subsection{R}
	
		\mgref{React}
		Libreria \mglo{JavaScript}{JavaScript} per la creazione di interfacce grafiche.
		
		\mgref{Reducer}
		In Redux, funzione che restituisce un nuovo stato dello store in seguito ad un'azione.
		
		\mgref{Render}
		Metodo richiesto da React.Component per la visualizzazione delle componenti sul DOM virtuale.
		
		\mgref{Repository}
		Ambiente di un sistema informativo in cui vengono gestiti i metadati attraverso
		tabelle relazionali. Il repository utilizzato dal \mglo{Gruppo}{gruppo} \zephyrus{} è fornito
		dalla piattaforma \mglo{GitHub}{GitHub}.
		
		\mgref{REST}
		Stile architetturale che offre la possibilità di manipolare rappresentazioni testuali di risorse Web utilizzando un set predefinito di operazioni.
	
	\subsection{S}
	
		\mgref{Safari}
		\mglo{Browser}{browser} web sviluppato dalla Apple.
		
		\mgref{Sistema di controllo di versione}
		Strumento che consente di tracciare le modifiche a cui viene sottoposto un insieme di file, consentendo di accedere alle vecchie versioni e di lavorare in più persone contemporaneamente.
		
		\mgref{Store}
		In Redux, ilcontenitore degli stati dell'applicazione.
		
%	\subsection{T}
%	\subsection{U}
	\subsection{V}
		\mgref{VirtualBox}
		Software open source per l'esecuzione di macchine virtuali.
	\subsection{W}
	
		\mgref{W3C}
		\textit{World Wide Web Consortium}. Organizzazione non governativa internazionale che ha come scopo quello di sviluppare tutte le potenzialità del World Wide Web. Al fine di riuscire nel proprio intento, la principale attività svolta dal W3C consiste nello stabilire standard tecnici che riguardino sia i linguaggi di markup che i protocolli di comunicazione.
		
		\mgref{WebStorm}
		\mglo{IDE}{IDE} che offre supporto allo sviluppo di applicazioni \mglo{JavaScript}{JavaScript} sia client che server. Offre inoltre supporto a \mglo{Node.js}{Node.JS}, \mglo{HTML}{HTML}, \mglo{CSS}{CSS} e frameworks come AngularJS e a librerie JavaScript come React.
		
		\mgref{Windows}
		Sistema operativo sviluppato da Microsoft.
%	\subsection{X}
%	\subsection{Y}
%	\subsection{Z}