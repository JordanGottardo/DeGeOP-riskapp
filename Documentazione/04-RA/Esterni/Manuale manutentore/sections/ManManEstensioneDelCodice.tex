\newpage

\section{Estensione del codice}

\subsection{Creazione di una nuova categoria di sidebar}
Le sidebar presentano, come richiesto da React una struttura altamente modulare.
Per aggiungere una nuova sidebar di seguito denominata per comodità \textit{NewSidebar} si devono effettuare le seguenti operazioni:
\begin{itemize}
	\item aggiungere \textit{ViewPkg}::\textit{DeGeOPView}::\textit{DeGeOPView}, nella funzione \textit{sidebarFactory()} il \textit{conditional render} della nuova sidebar;
	\item aggiungere in \textit{ViewPkg}::\textit{DeGeOPView} la componente React  \textit{NewSidebar} che renderizza la sidebar. Essendo ogni sidebar divisa in due sezioni (contenuto e bottoni), per renderizzarla dovranno venire a sua volta richiamate le renderizzazioni:
	\begin{itemize}
		\item della nuova componente React (da creare anch'essa) \textit{ViewPkg}::\textit{DeGeOPView}::\textit{ContentPkg}::\textit{NewSidebarContent};
		\item di una delle componenti a scelta tra \textit{ViewPkg}::\textit{DeGeOPView}::\textit{ButtonsPkg}::\textit{threeButtons}, \textit{ViewPkg}::\textit{DeGeOPView}::\textit{ButtonsPkg}::\textit{twoButtons}.
	\end{itemize}
\end{itemize}

\subsection{Aggiornamenti dei campi dati di asset e nodi}
Visto che \riskapp{} utilizza il modello di sviluppo Agile, il loro server e i file JSON che esso ritorna, sono soggetti a continui e rapidi cambiamenti. Di seguito viene descritto come estendere il codice in caso venissero aggiunti nuovi campi dati per asset o nodi.

\subsubsection{Asset}In caso venissero aggiunti nuovi campi dati per un asset seguire i seguenti passi:
\begin{itemize}
	\item aggiungere in \textit{StorePkg}::\textit{ProcessPkg}::\textit{Asset} il campo dati e le funzioni per gestire le sue eventuali validazioni;
	\item aggiungere nello \textit{state} \textit{ViewPkg}::\textit{DeGeOPView}::\textit{DeGeOPView} il campo dati e le sue validazioni e in		\textit{ViewPkg}::\textit{SidebarPkg}::\textit{ContentPkg}::\textit{InsertAssetContent} e \textit{ViewPkg}::\textit{SidebarPkg}::\textit{ContentPkg}::\textit{ViewAssetContent} aggiungere il campo dati di interesse utilizzando le componenti di React-Toolbox, come descritto nella \textit{Specifica Tecnica}.
\end{itemize}

\subsubsection{Nodo}In caso venissero aggiunti nuovi campi dati per un  nodo seguire i seguenti passi:
\begin{itemize}
	\item aggiungere in \textit{StorePkg}::\textit{ProcessPkg}::\textit{Node} nelle sue derivate il campo dati e le funzioni per gestire le sue eventuali validazioni;
	\item aggiungere nello \textit{state} \textit{ViewPkg}::\textit{DeGeOPView}::\textit{DeGeOPView} il campo dati e le sue validazioni e in		\textit{ViewPkg}::\textit{SidebarPkg}::\textit{ContentPkg}::\textit{InsertNodeContent} e \textit{ViewPkg}::\textit{SidebarPkg}::\textit{ContentPkg}::\textit{ViewNodeContent} aggiungere il campo dati di interesse utilizzando le componenti di React-Toolbox, come descritto nella \textit{Specifica Tecnica}.
\end{itemize}

%Se cambiano le API -> aggiungere campi dati agli asset, nodi, archi
%supporto al multilingua
%nuovo tipo di nodo
%aggiungere foto all'asset (campi dati)
%nuovi tipi di azioni operabili sul modello (Reducer->nuovo blocco)
%sile css inline + react-toolbox-themr per modificare la presentazione dei componenti toolbox
%gestore chiamate???
%marker nodi 
%react-dev tools
%redux-dev tools 
%nel sisstema di build + nel chrome store -> sisitea di build in sistema di development
%aiuto per trovare bug