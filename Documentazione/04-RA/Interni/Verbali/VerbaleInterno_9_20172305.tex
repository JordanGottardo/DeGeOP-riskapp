\documentclass[a4paper,11pt]{article}
\usepackage{../../../Template/zemplateVerb}

%\externaldocument{path/to/glossario}
\docVerbV{9}
\docTitle{Verbale interno - \getdocVerbV}
\docVersion{1.0.0}
\docCreationDate{\frmdata{19}{05}{2017}}
\docLastUpdateDate{\frmdata{19}{05}{2017}}
\docStatus{Approvato}
\docEditors{Giovanni Prete}
\docVerificators{Leonardo Brutesco}
\docApprovers{Marco Pasqualini}
\docUse{Interno}
\docDestination{\Tullio \\ & \Cardin \\ & \zephyrus}
\docJournal{
1.0.0 & \frmdata{23}{05}{2017} & Marco Pasqualini &  \responsabile & Approvazione\\
0.1.0 & \frmdata{23}{05}{2017} & Leonardo Brutesco &  \verificatore & Verifica\\
0.0.1 & \frmdata{23}{05}{2017} & Giovanni Prete &  \programmatore & Stesura documento\\
}

\begin{document}
	\section{Estremi della riunione}
	\begin{itemize}
		\item \textbf{data:} \frmdata{23}{05}{2017};
		\item \textbf{ora inizio:} \frmora{09}{30};
		\item \textbf{ora fine:} \frmora{10}{30};
		\item \textbf{luogo:} Torre Archimede - Padova;
		\item \textbf{segretario:} Giovanni Prete;
		\item \textbf{partecipanti:}
			\begin{itemize}
				\item Daniel De Gaspari;
				\item Giovanni Prete;
				\item Giulia Petenazzi;
				\item Jordan Gottardo;
				\item Leonardo Brutesco;
				\item Marco Pasqualini.
			\end{itemize}
		\item \textbf{assenti:}
			\begin{itemize}
			 \item nessuno.
			\end{itemize}
	\end{itemize}
	\section{Ordine del giorno}
		Discussione post revisione di qualifica.
	\section{Verbale della riunione}
		\begin{itemize}
			\item identificazione dei punti carenti di quanto consegnato in revisione di qualifica;
			\item discusse possibili soluzioni per miglioramenti tecnici del prodotto consegnato
			\item votazione sull'introduzione di TypeScript (4 a favore e 2 contrari);
			\item discusse possibili soluzioni per organizzare al meglio  il lavoro da svolgere a livello temporale (scadenze di consegna) e personale.
		\end{itemize}
	\section{Decisioni prese}
		\begin{itemize}
			\itemVI il \responsabile{} stenderà un adeguato piano di lavoro nel \pdp{} che guidi i prossimi avanzamenti del progetto alla luce di quanto discusso in questa riunione;
			\itemVI il \responsabile{} terrà conto dei periodi di stage di cinque membri del gruppo (alcuni certi fin da data del presente verbale), e delle esigenze personali dei membri del gruppo. Si prevederanno momenti di lavoro più intensi durante il mese di agosto. L'obbiettivo deve essere quello di consentire un avanzamento del progetto, seppur in maniera rallentata;
			\itemVI il prodotto finale verrà consegnato nella revisione del 29 agosto;
			\itemVI per alleggerire il carico di responsabilità della componente \textit{DeGeOPView} e per sfruttare al meglio la tecnologia offerta da \textit{Redux}, verranno rimossi dalla  componente \textit{DeGeOPView} i dati riguardanti il comportamento della \textit{View}, che vengono spostati nella componente \textit{Store};
			\itemVI la componente \textit{DeGeOPView} in ogni dato momento conterrà alternativamente i dati relativi a soltanto un asset, nodo, arco, scenario o risultato di analisi, alleggerendo ulteriormente il carico di responsabilità di questa componente;
			\itemVI viene eliminato il livello intermedio di renderizzazione della \textit{Sidebar}. La componente \textit{DeGeOPView} si occuperà direttamente della renderizzazione del contenuto della \textit{sidebar} e dei suoi \textit{bottoni};
			\itemVI la \textit{sidebar} riguardante gli scenari sarà \textit{sidebar} di default;
			\itemVI introdotta la \textit{sidebar} riguardante le analisi;
			\itemVI le \textit{sidebar}, che prima erano divise sia in base alla funzionalità (inserimento o modifica) sia in base all'oggetto (asset, nodo...) vengono accorpate in componenti suddivise solamente per oggetto, per eliminare la ripetizione di importanti parti di codice;
			\itemVI introdotte le tecnologie TypeScript (per una maggiore aderenza del codice ai documenti) e Axios (per la gestione delle chiamate al server di \riskapp{}).
			\itemVI fissare un incontro con il proponente per definire i dettagli riguardanti le API di scenari e analisi.
			%c) introdotta classe analysisMock. PROSSIMO VERBALE
			%\itemVI introduzione della tecnologia Axios a supporto delle chiamate REST. PROSSIMO VERBALE
		\end{itemize}
\end{document}
